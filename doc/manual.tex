\documentclass{article}

\usepackage{manual-preamble}

\title{Classe \texttt{livroabertoem}}
\author{Tarso Boudet Caldas}
\date{\today}


\begin{document}

\begin{abstract}
	Este documento descreve os comandos e ambientes criados para o projeto do
	Livro Aberto de Matemática do Ensino Médio, da Associação Livro Aberto em
	parceria com o IMPA.\footnote{Para mais informações sobre o projeto, acesse
		\url{https://www.umlivroaberto.org}}

	O principal diferencial da classe \verb|livroaberto| é a possibilidade
	da criação de um livro para o professor e para o aluno a partir dos
	mesmos arquivos, bastando escolher a opção \verb|professor| durante a
	invocação.
\end{abstract}

\section{Introdução}

A classe \verb|livroabertoem| serve como template para a criação dos
livros do Ensino Médio da Associação Livro Aberto, isto é, além de
contar com a interface para a geração do material do aluno e do
professor, possui também um design gráfico próprio elaborado por Enzo
Esberard, bolsista da Escola de Belas Artes da UFRJ\@. Os detalhes do
modelo gráfico serão explicitados em um documento à parte, atualmente
em desenvolvimento.

Aqui iremos apenas descrever os comandos criados para o desenvolvimento
de capítulos seguindo a filosofia e sintaxe do projeto da coleção do
ensino médio.

\section{Pré-requisitos}

Usamos como base a classe \verb|memoir| para o desenvolvimento do
modelo. Além disso, utilizamos a seguinte lista de pacotes para o a
criação da interface: \latexinline{multicol}, \latexinline{xstring},
\latexinline{comment}, \latexinline{xpatch},
\latexinline{paracol}, \latexinline{xcolor},
\latexinline{graphicx}, \latexinline{calc},
\latexinline{tikz}, \latexinline{tcolorbox} e \latexinline{pagecolor}.


\section{Divisão do documento}

Os livros do Ensino médio possuem a seguinte hierarquia na divisão dos documentos: \dirtree{%
  .1 Volume.
  .2 Capítulo.
  .3 Seção
  .4 Explorando.
  .4 Organizando.
  .4 Praticando.
  .4 Saiba Mais.
  .3 Exercícios.
  .3 Material Suplementar.
  .3 Bibliografia.
}

\subsection{Volume}

É criado com o comando \volume{}

\end{document}
