\section{Referências Bibliográficas}\label{sec:referencias}

Para os referências Bibliográficas, estamos utilizando Bib\LaTeX\ em
conjunto com o programa \verb|biber|, que são uma evolução do antigo
Bib\TeX. Enquanto o
\href{https://linorg.usp.br/CTAN/macros/latex/contrib/biblatex/doc/biblatex.pdf}{\texttt{biblatex}}
é o pacote de estilo que faz a formatação da bibliografia no documento,
o
\href{http://mirrors.ctan.org/biblio/biber/base/documentation/biber.pdf}{\texttt{biber}}
é um programa que processa o arquivo \latexinline/.bib/ para que o
Bib\LaTeX\ possa ler e utilizar a sua interface para lidar com
Bibliografias de acordo com a configuração desejada. Isto é, o pacote
Bib\LaTeX\ pode ser utilizado a partir de uma bibliografia processada
usando o \textbf{programa} Bib\TeX, enquanto uma bibliografia usando o
\verb|biber| precisa do pacote Bib\LaTeX\ para ser lida corretamente
pelo \LaTeX.

As bibliografias utilizando Bib\LaTeX\ seguem a mesma sintaxe das que
usam o Bib\TeX, porém com uma grande expansão dos tipos de referências
e das entradas de dados, além de ser mais fácil de configurar.

\begin{description}
	\item [Se você já conhece o sistema Bib\TeX] Não terá grandes dificuldades em montar
	      as bibliografias no formato Bib\LaTeX. Para uma referência dos diferentes tipos de entradas
	      e de campos de dados, além dos comandos de referencias base, consulte o documento
        \empht{\href{https://tug.ctan.org/info/biblatex-cheatsheet/biblatex-cheatsheet.pdf}{Bib\LaTeX\ Cheatsheet}.}

	\item [Caso não tenha muita familiaridade com o Bib\TeX] Uma boa introdução sobre como montar
	      uma bibliografia é a
	      \href{https://linorg.usp.br/CTAN/macros/latex/contrib/biblatex-contrib/biblatex-abnt/doc/biblatex-abnt.pdf}{%
		      documentação do estilo \texttt{biblatex-abnt}%
	      }, que, apesar de não ser o estilo de bibliografia utilizado neste projeto, as Seções
	      5 e 6 dão uma explicação básica dos comandos e dos tipos de entradas mais utilizados.

	\item [Se você quer uma referência para se aprofundar no Bib\LaTeX] O documento
	      \href{https://docs.google.com/viewer?url=https://github.com/PaulStanley/biblatex-tutorial/releases/download/0.2/biblatex-tutorial.pdf}{%
		      Bib\LaTeX, An Easier Read%
	      }, de Paul Stanley, traz um tutorial bem completo e de fácil leitura, e se difere
	      da documentação do pacote, que tem um caráter mais técnico. Porém, é importante
	      ressaltar que a \textbf{versão mais recente} do documento \textbf{é de 2017}, e por isto
	      \textbf{está bastante desatualizada}. Devemos portanto, ter isto em mente, já que
	      o Bib\LaTeX\ é um pacote em constante desenvolvimento.

	      De toda forma, as informações trazidas neste tutorial são, em sua maior
	      parte, ainda válidas, é podem ser úteis caso se esteja buscando fazer
	      algo mais específico com a bibliografia e os comandos de citação. Seu texto pode
        ser aproveitado tanto por iniciantes na criação de bibliografias no \LaTeX, quanto
        por usuários experientes do Bib\TeX\ e do Bib\LaTeX.

  \item [As entradas devem estar no formato \enquote{stanley2017} ou 
        \enquote{stanley2017easier-reading}] Isto é, primeiro o nome do autor, depois o ano,
        seguido de parte do título da obra\footnote{%
          Opcional no caso de um trabalho do autor na bibliografia, mas 
          obrigatório no caso de múltiplas obras de um mesmo autor
        }

	\item [Adicione a bibliografia com o comando \latexinline/\addbibresource/ no preâmbulo]
	      Preferencialmente dentro do arquivo de estilo do capítulo.

	\item [Insira a bibliografia com o comando \latexinline/\printbibliography/] Este comando
	      irá criar uma nova seção com a bibliografia, usando apenas os itens citados pelos
	      pelo documento usando os comandos de citação (\cref{sec:citacoes}). Caso deseje
	      incluir alguma entrada que não tenha sido citada, basta usar o comando
	      \latexinline/\nocite/, com as entradas desejadas no argumento, separadas por vírgulas,
	      ou usando um \enquote{\texttt{*}} dentro argumento (\latexinline/\nocite{*}/),
	      que irá botar todas as entradas do arquivo de bibliografia.

	\item [Pode-se utilizar mais de uma bibliografia] É possível dividir o conteúdo
	      bibliográfico em diversos arquivos e imprimir mais de uma bibliografia. Isto
	      pode ser útil, por exemplo, para criar uma bibliografia diferente para o material
	      do aluno e do professor, podendo a segunda ser introduzida em alguma página
	      do professor. Isto pode ser feito incluindo um \emph{label} como opção do comando
	      \latexinline/\addbibresource/. Para mais informações, consulte as Seções
	      3.8.1 e 3.8.2 da documentação Bib\LaTeX\ ou o capítulo 7 do tutorial
	      \emph{Bib\LaTeX, An Easier Read}, citado anteriormente.

	\item [Não é necessário configurar a bibliografia] As configurações criadas pelo pacote
	      \verb|livroabertoem-preamble| já são suficientes para criar a maior parte dos tipos
	      de entrada, e as configurações globais não devem ser alteradas. Entretanto, o
	      Bib\LaTeX\ possui uma grande flexibilidade para criar diferentes tipos de entradas,
	      sendo possível, por exemplo, criarmos uma para lidar apenas com leis brasileiras.%
	      \footnote{Não iremos dar exemplos disso, apenas queremos constatar que isto é possível.}

	\item [Faça a configuração em um arquivo \texttt{biblatex.cfg}] Se houver na pasta
	      principal do projeto um arquivo com este nome, será automaticamente lido como
	      um arquivo de configurações do Bib\LaTeX. Sendo assim comandos de citações, tipos e
	      campos de entradas do banco de dados podem ser modificados ou criados dentro
	      deste arquivo, sem a necessidade de adicionar qualquer outro comando nos arquivos
	      principais do projeto. Mas \textbf{atenção}, o Overleaf não lê arquivos com a
	      extensão \verb|.cfg|, por isto, caso queira editar este arquivo, é recomendado que
	      mude o nome para \verb|biblatex.tex| e depois altere para o nome original.
\end{description}

\subsection{Citações}\label{sec:citacoes}

As citações devem ser feitas utilizando os comandos do Bib\LaTeX, sendo
os principais
\begin{description}
	\item [\latexinline/\textcite/] Citações no meio do texto. Ficam no formato
	      Autor (Ano).
	\item [\latexinline/\parencite/] Citações em parênteses, normalmente utilizadas
	      no final de uma frase. Ficam no formato (Autor, Ano).
	\item [\latexinline/\footcite/] Citação na nota de rodapé.
\end{description}
Também há o comando \latexinline/\cite/, mas no estilo de citação APA, possui
a mesma função do \latexinline/\textcite/, porém sem parênteses por volta do ano.
Este é um comando mais genérico, e que costuma ter funções diferentes em cada estilo de bibliografia escolhido. Por exemplo,
no estilo \verb|abnt-numeric| insere o número da citação na página de referências.
Isto deve ser levado em conta na hora de usar este comando, pois caso venha a ser
escolhido outro formato de bibliografia, o formato da citação também irá mudar.


Todos os comandos de citação possuem três argumentos, dois opcionais e
um obrigatório. O obrigatório é a entrada da bibliografia desejada, e
pode ser feito uma lista separada por vírgulas com todas as entradas
que se deseja citar em um comando só (caso se esteja citando duas ou
mais obras, por exemplo). O primeiro argumento opcional insere texto
antes da citação, enquanto o segundo insere texto após. Além disso, o
segundo argumento opcional pode ser utilizado para inserir o número das
páginas, necessidanto apenas inserir o número. Caso seja utilizado
apenas um argumento opcional, o comando irá considerar como o elemento
pós-citação, podendo ser o número de páginas. Exemplificando:

\begin{itemize}
	\item \latexinline/\textcite{stanley2017}/ resulta em \enquote{\textcite{stanley2017}}.
	\item \latexinline/\textcite[41-44]{stanley2017}/ resulta em \enquote{\textcite[41-44]{stanley2017}}.
	\item \latexinline/\textcite[C.f.][41-44]{stanley2017}/ resulta em \enquote{\textcite[C.~f][41-44]{stanley2017}}.
\end{itemize}

Para citações em parênteses:

\begin{itemize}
	\item \latexinline/\parencite{stanley2017}/ resulta em \enquote{\parencite{stanley2017}}.
	\item \latexinline/\parencite[41-44]{stanley2017}/ resulta em \enquote{\parencite[41-44]{stanley2017}}.
	\item \latexinline/\parencite[C.f.][41-44]{stanley2017}/ resulta em \enquote{\parencite[C.~f.][41-44]{stanley2017}}.
\end{itemize}

Para ilustrar as diferenças entre os comandos \latexinline/\cite/ e
\latexinline/\textcite/ no esilo APA\@:

\begin{itemize}
	\item \latexinline/\cite{stanley2017}/ resulta em \enquote{\cite{stanley2017}}.
	\item \latexinline/\cite[41-44]{stanley2017}/ resulta em \enquote{\cite[41-44]{stanley2017}}.
	\item \latexinline/\cite[C.f.][41-44]{stanley2017}/ resulta em \enquote{\cite[C.~f.][41-44]{stanley2017}}.
\end{itemize}


\subsection{\texttt{csquotes}}\label{sec:csquotes}

Além dos comandos para citações já presentes no Bib\LaTeX, usamos o
pacote
\href{https://linorg.usp.br/CTAN/macros/latex/contrib/csquotes/csquotes.pdf}{\texttt{csquotes}}
para integrar esta interface ao ambiente \verb|quote|, além de outros
comandos úteis para lidar com citações. Já tratamos do comando
\latexinline/\enquote/ (Veja \cref{sec:formatacao}), vamos agora falar
dos outros comandos e também dos ambientes do pacote \verb|csquotes|.

\begin{description}
	\item [Para citações no meio de um parágrafo, use \latexinline/\textquote/\footnote{%
          Assim como \latexinline/\enquote/, possui uma versão alternativa 
          \latexinline/\textquote*/ que começa usando aspas simples
	      }] Este comando possui três argumentos, o obrigatório, com o texto da citação,
        e dois opcionais, um com o autor original e outro com a pontuação no final da 
        citação.\footnote{Isto está presente na interface do pacote para que seja possível
          alterar a ordem da pontuação em relação ao autor} Caso somente um dos argumentos
          opcionais for usado, será entendido como o de autor da citação.
        (\citetitle[Veja][Seção 3.3]{csquotes}). A seguir, temos alguns exemplos:
        \begin{itemize}
          \item \latexinline/\textquote{texto}/: \textquote{texto}
          \item \latexinline/\textquote[][.]{texto}/: \textquote[][.]{texto}
          \item \latexinline/\textquote[Stanley 2017, 41--44]{texto}/: 
               \textquote[Stanley 2017, 41--44]{texto}
          \item \latexinline/\textquote[{\cite[41-44]{stanley2017}}][.]{texto}/:\footnote{%
              Caso queira incluir um comando de citação no autor, o mesmo deve estar
              entre chaves, porém, na maior parte dos casos, utilizaremos a integração
              explicada na \cref{sec:integracao-csquotes}.
            } \textquote[{\cite[41-44]{stanley2017}}]{texto}.
        \end{itemize}

  \item [Para citações grandes (i.~e. maiores que 4 linhas) utilize o ambiente 
        \texttt{displayquote}] Este ambiente possui os mesmos argumentos do comando
        \latexinline/\textquote/, portanto, os exemplos anteriores servem para este 
        ambiente. A diferença entre os dois comandos é que enquanto um utiliza o
        comando \latexinline/\enquote/ como base, este utiliza o ambiente \verb|quote| 
        como base. Segue um exemplo do resultado deste ambiente:

\end{description}
\begin{center}
  \begin{minipage}{.4\linewidth}
\begin{minted}[linenos=false]{latex}
  \begin{displayquote}[{\cite[41-44]{stanley2017}}][.]
    There is a big difference in the essential 
    way in which Bib\LaTeX\ and traditional 
    Bib\TeX-based systems work. In particular,
    in Bib\LaTeX\ the external program is used 
    only to prepare data: the formatting and 
    output of that data is largely handled 
    using \LaTeX\ code, whereas in Bib\TeX, most 
    of the formatting is done by Bib\TeX, 
    producing a bibliography which is more-or-less 
    ready to be typeset as it is.
  \end{displayquote}
\end{minted}
\end{minipage}
\hfill
\begin{minipage}{.4\linewidth}
  \begin{displayquote}[{\cite[41-44]{stanley2017}}][.]
    There is a big difference in the essential 
    way in which Bib\LaTeX\ and traditional 
    Bib\TeX-based systems work. In particular,
    in Bib\LaTeX\ the external program is used 
    only to prepare data: the formatting and 
    output of that data is largely handled 
    using \LaTeX\ code, whereas in Bib\TeX, most 
    of the formatting is done by Bib\TeX, 
    producing a bibliography which is more-or-less 
    ready to be typeset as it is
  \end{displayquote}
\end{minipage}
\end{center}
\begin{description}
  \item [Caso a citação esteja em um intermediário entre quatro e cinco linhas, 
        use \latexinline/\blockquote/] Este comando funciona da mesma forma
        que os dois anteriores, porém, ele detecta o número de linhas ocupadas
        pela citação, e caso seja maior do que o definido, passará a usar o ambiente
        \texttt{displyquote}
\end{description}


Além dos comandos descritos acima, é válido notar que este pacote possui ainda
comandos para inserção de texto e para elipses, e textos omitidos, sento estes,
respectivamente, \latexinline{\textins}, \latexinline{\textelp} e \latexinline{\textdel}.
Todos eles requerem um argumento, sendo que, no caso do primeiro, pode estar vazio.

\begin{description}
  \item [Para indicarmos a inserção de texto, usamos \latexinline/\textins/] Se for 
        modificada apenas parte da palavra, o texto inserido deve estar na variante
        \latexinline/\textins*/. Como exemplo: \latexinline/\textins{texto}/ gera
        \textins{texto}.
  \item [O comando \latexineline/\textelp/ não requer texto, mas deve estar acompanhado de
        chaves, mesmo que vazias] No caso de elipses, o argumento vazio irá apenas inserir
        \enquote{\textelp{}}, mas se houver algum texto, será tratado como uma inserção
        logo após a elipse. Caso se queira a inserção \textbf{antes} da elipse, use
        \latexineline{\textelp*}
  \item [Para omitir parte de uma palavra, use \latexinline/\textdel/] Este comando faz com que
        o conteúdo do argumento seja omitido e substituído por chaves vazias. Isto é,
        \latexinline{texto\textdel{s}} nos dá \enquote{text\textdel{s}}
\end{description}

\subsubsection{A integração de \texttt{csquotes} com o Bib\LaTeX}%
\label{sec:integracao-csquotes}

O principal motivo de usarmos o \verb|csquotes| é que os comandos que tratamos 
anteriormente possuem versões alternativas, respectivamente, \latexinline/\textcquote/,
\verb|displaycquote| e \latexinline/\blockquote/. Veja que a diferença no nome dos comandos
é que em vez de terminarem com \enquote{\texttt{quote}}, passam a usar 
\enquote{\texttt{cquote}} (possível lembrar como \emph{cite quote}). Estes comandos
substituiem no primeiro argumento opcional, os argumentos de citação, isto é,
\latex/\textcquote[pre][pos]{entrada}[pontuação]/
onde as entradas usadas são as do arquivo de bibliografia usado.

Alterando o exemplo anterior para usar \verb|displaycquote|, temos

\begin{center}
  \begin{minipage}{.4\linewidth}
\begin{minted}[linenos=false]{latex}
  \begin{displaycquote}[41-44]{stanley2017}[.]
    There is a big difference in the essential 
    way in which Bib\LaTeX\ and traditional 
    Bib\TeX-based systems work. In particular,
    in Bib\LaTeX\ the external program is used 
    only to prepare data: the formatting and 
    output of that data is largely handled 
    using \LaTeX\ code, whereas in Bib\TeX, most 
    of the formatting is done by Bib\TeX, 
    producing a bibliography which is more-or-less 
    ready to be typeset as it is
  \end{displaycquote}
\end{minted}
\end{minipage}
\hfill
\begin{minipage}{.4\linewidth}
  \begin{displaycquote}[41-44]{stanley2017}[.]
    There is a big difference in the essential 
    way in which Bib\LaTeX\ and traditional 
    Bib\TeX-based systems work. In particular,
    in Bib\LaTeX\ the external program is used 
    only to prepare data: the formatting and 
    output of that data is largely handled 
    using \LaTeX\ code, whereas in Bib\TeX, most 
    of the formatting is done by Bib\TeX, 
    producing a bibliography which is more-or-less 
    ready to be typeset as it is
  \end{displaycquote}
\end{minipage}
\end{center}
