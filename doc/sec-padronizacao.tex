\section{Orientações para a padronização}%
\label{sec:padronizacao}

A seguir temos algumas recomendações gerais para facilitar um código
limpo que ajude na colaboração entre muitos autores, além de padronizar
o estilo de elementos comuns a todos os livros, como bibliografia,
referências, uso de unidades, etc.

\subsection{Formatação do código}\label{sec:formatacao}

É bom que o código siga algumas orientações para sua padronização, facilitando
a leitura por diferentes colaboradores. Estas não são regras, apenas recomendações,

\begin{description}
	\item[Tente deixar as linhas com cerca de 72 caracteres] Alguns editores de
		texto, como Vim e Neovim, não quebram a linha automaticamente. Por
		isso, para facilitar diferentes formas de edição em diversos tamanhos
		de tela, e também facilitar o alinhamento de elementos do código, tente
		limitar o número de caracteres em cada linha sempre que possível. Isto
		pode ser feito quebrando linhas a partir de frases, por exemplo.

	\item[Utilize indentação para os ambientes] Assim, como nos exemplos
		presentes neste manual, os ambientes devem possuir identação sempre
		aumentando um grau pra cada aninhamento.

	\item[Modo matemático na sintaxe do \LaTeX] A sintaxe \latexinline/$ $/ e
		\latexinline/$$ $$/ são para entrar no modo matemático no \TeX. O
		formato do \LaTeX\ utiliza \latexinline/\( \)/ e \latexinline/\[ \]/
		respectivemente. No primeiro caso, não há diferença na função, mas o
		seu uso é recomendado para entender melhor onde se inicia o modo
		matemático na linha, pois, por exemplo, no caso de se listar diversas
		variáveis, usamos \latexinline/\(a\), \(b\), \(c\)/, que é melhor para
		a leitura do que \latexinline/$a$, $b$,  $c$/, especialmente em uma
		lista grande de variáveis.\footnote{Aqui não usamos
			\latexinline/\(a, b, c\)/, pois isso impediria a quebra de linha entre
			as vírgulas. } Já no segundo caso, há uma diferença de espaçamento no
		modo matemático de \emph{display}, por isso, \textbf{não utilize}
		\latexinline/$$ $$/. %chktex 45%%chktex 45

	\item [Notas de rodapé devem ficar após a pontuação no texto] Nos casos em que a nota de rodapé
	      é inserida em uma palavra seguida imediatamente de uma pontuação no texto, a nota de rodapé
	      fica sempre após a pontuação.

  \item [Para aspas, utilize \latexinline/\enquote{...}/] O pacote \verb|csquotes|
        possui este comando que pode ser usado em ninho, e irá automatizar
        a troca de aspas duplas para simples, dependendo do contexto. É recomendado
        em relação às aspas padrão do \LaTeX\ (escrito com \verb|``...''|) por
        trazer mais clareza ao código, para facilitar a troca para citações com 
        referência (Veja \cref{sec:citacoes}), e também elimina a confusão
        da diferença entre \verb|``...''| e \verb|"..."|.

	\item[Coloque \enquote{\texttt{\%}} no final da linha, caso termine com
	\enquote{\texttt{\{}}] A quebra de linha no código é lida pelo \LaTeX\
		como um espaço, por isso
		\begin{latexcode}
\enquote{
 ...
}
       \end{latexcode}
		irá criar um espaço indesejado após o início do comando, sendo assim, é
		recomendado colocar \texttt{\%} após o início do comando, para que
		espaço seja ignorado:
		\begin{latexcode}
\enquote{%
 ...
}
       \end{latexcode}
	\item[Use \texttt{\textbackslash} após comandos de texto, caso sejam seguidos
	de espaço] Comandos no \LaTeX\ ignoram o espaço que vem após o mesmo.
		Um exemplo disso é
		\begin{itemize}
			\item \latexinline/\LaTeX ignora o espaço/: \LaTeX ignora o espaço %chktex 1
			\item \latexinline/\LaTeX\ insere o espaço corretamente/: \LaTeX\ insere o espaço
			      corretamente.
		\end{itemize}
		Isto é chamado de \emph{espaço de controle}, e também deve ser usado após os pontos
		de uma abreviação, isto é, \enquote{ex.\ tem um espaço de controle} é escrito com
		\latexinline{ex.\ tem um espaço de controle}.\footnote{O \TeX\ adiciona um espaço
		maior após o ponto para separar melhor as frases, mas isto não é desejado quando
		o ponto é usado para indicar uma abreviação. }

	\item [Coloque \enquote{\latexinline/\@/} antes do ponto final de frases encerradas
	      com siglas e semelhantes] Quando o ponto final vem após uma palavra apenas com letras
	      maiúsculas, o \TeX\ entende como uma abreviação, e por isso não insere o espaço maior
	      (isto acontece, por exemplo, com algarismos romanos). Para corrigir estes casos,
	      use \enquote{\latexinline/\@/} antes do ponto.

	\item [Se atente ao tipo do traço (curto, médio ou travessão)] O \LaTeX\ possui
	      três tipos de traço. Um traço curto (\enquote{\texttt{-}}), usado como hífen no meio
	      das palavras como composição (\enquote{salve-se} é um exemplo);
	      traço médio (\enquote{\texttt{--}}), usado principalmente para denotar intervalos,
	      elementos em série, etc (1999--2001 e A--Z são exemplos).
	      Por fim, há um traço longo que representa o travessão (\enquote{\texttt{---}}),
	      e deve ser segundo \href{https://pt.wikipedia.org/wiki/Travessão}%
	      {as devidas regras linguísticas}. Nos primeiros dois casos não deve haver espaço entre
	      o traço, apenas no terceiro.
\end{description}


\subsection{Referências cruzadas}\label{sec:referencias}

Para nos referirmos a elementos dentro do texto, utilizamos o pacote
\href{https://linorg.usp.br/CTAN/macros/latex/contrib/cleveref/cleveref.pdf}{\texttt{cleveref}},
que possui uma interface para \enquote{descobrir} qual o tipo de
referência desejada, tornando desnecessário escrever algo do tipo
\latexinline/Tabelas~\ref{tab:divisoes-comandos}/.\footnote{O comando
	\enquote{\latexinline/~/} insere um \emph{espaço rígido}, isto é, o
	\LaTeX\ não irá quebrar a linha neste espaço. Deve ser utilizado para
	juntar iniciais, pronomes de tratamento, números, etc. Por exemplo,
	\protect\latexinline/Sr.~Tarso B.~Caldas/,
	\protect\latexinline/Século~XXI/, e outras situações semelhantes. }
Isto é feito através do comando \latexinline/\cref{label}/. Você pode
colocar vários \emph{labels} em um comando só, que ele irá criar uma
lista um um alcance de referências, por exemplo,
\latexinline/\cref{fig1,fig2,fig3}/ irá inserir \enquote{Figuras 1 a
	3}, o que também pode ser feito com
\latexinline/\crefrange{fig1}{fig3}/. Já com
\latexinline/\cref{fig1,fig2,fig4}/, teremos \enquote{Figuras 1, 2 e
	4}.%

Todas as divisões e ambientes que possuem título podem ser
referenciadas, porém, aquelas que não são numeradas podem ter apenas o
seu título referenciado, que pode ser feito a partir de
\latexinline/\namecref{label}/.

Por fim, caso queira se referir à página, utilize o comando
\latexinline/\cpageref{label}/.

\begin{description}
	\item [Não deixe espaço entre o \emph{label} e o elemento ao qual ele se refere] Para
	      que o \LaTeX\ identifique exatamente qual o ponto correto da referência, é
	      necessário que não haja espaços entre, por exemplo, a seção e o \emph{label}:
	      \latexinline/\section{titulo}\label{sec:titulo}/. Se o label for muito grande
	      e for desejável quebrar a linha, coloque \enquote{\texttt{\%}}
	      (vide \cref{sec:formatacao}).
	\item [Utilize nomes para os \emph{labels}, não números] As referências devem ser
	      idendificadas pelo que representam, e não por sua ordem, visto que a ordem
	      de figuras, tabelas, seções,~etc.\ pode mudar. Assim como no caso das figuras
	      (Ver \cref{sec:figuras}), utilize nomes de fácil reconhecimento do elemento,
	      de preferência o título ou as primeiras palavras da legenda.
	\item [Adicione sempre um prefixo aos \emph{labels}] Podemos dividir os \emph{labels} usando
	      \enquote{\texttt{:}} e, portanto, usar isto para facilitar o reconhecimento da
	      referência, por exemplo, \latexinline/\label{fig:figura}/ para referenciar figuras.
	      A \cref{tab:prefixos} apresenta os prefixos escolhidos para cada elemento de texto.
	      \begin{table}[H]\label{tab:prefixos}
		      \centering
		      \begin{threeparttable}
			      \begin{tabular}
				      {>{\ttfamily}p{.2\linewidth}p{.3\linewidth}}
				      \toprule
				      \normalfont Prefixo & Elemento                \\
				      \midrule
				      fig                 & Figura                  \\
				      tab                 & Tabelas                 \\
				      cap                 & Capítulos               \\
				      sec                 & Seções                  \\
				      exp                 & Explorando              \\
				      org                 & Organizando             \\
				      prac                & Praticando              \\
				      know                & Saiba Mais              \\
				      exrc                & Exercícios              \\
				      task                & Atividades              \\
				      exmp                & Exemplo                 \\
				      obs                 & Observação              \\
				      def                 & Definição               \\
				      thm                 & Teorema                 \\
				      rfl                 & Refletindo\tnote{*}     \\
				      kwl                 & Você Sabia?\tnote{*}    \\
				      rsc                 & Para pesquisar\tnote{*} \\
				      \bottomrule
			      \end{tabular}
			      \begin{tablenotes}
				      \footnotesize
				      \item[*] Por estes comandos não possuírem nem título nem numeração, só
				      podem ser referenciados por suas páginas usando \latexinline/\cpageref/.
			      \end{tablenotes}
			      \caption{Prefixos a serem usados na criação de \emph{labels}}
		      \end{threeparttable}
	      \end{table}
	\item [Os \emph{labels} de exercícios seguem outra regra] Os exercícios,
	      caso façam parte de uma seção, devem levar o nome da seção (\verb|exrc:secao|),
	      e se for uma seção própria, use o nome do capítulo (\verb|exrc:capitulo|).
\end{description}


\subsection{Cores}\label{sec:cores}

Os livros todos utilizam uma paleta de cores definida para o projeto, e
não se deve utilizar outras cores na produção de figuras em \TikZ, por
exemplo. Todas as cores possuem nomes com relação ao seu uso, por
exemplo, as cores de cada volume recebem o nome do volume utilizado.
Isso é feito a partir de apelidos para as cores.\footnote{Boa parte
	destes apelidos deriva de versões antigas do modelo, e servem apenas
	para evitar conflitos de versão.} A \cref{tab:cores} lista todas as
cores definidas, seus códigos e os apelidos dados.

\subsubsection{Comandos de cores}

Apesar de termos estas opções todas de cores, recomendamos que priorize
o uso das cores do volume, da seção atual e da coleção. Nos dois
primeiros casos, é possível fazer isso com os comandos
\latexinline/\currentcolor/ e \latexinline/\currentvolcolor/, que devem
ser utilizados no mesmo lugar onde fosse ser inserida a cor. Já a
terceira é a primeira cor definida, o azul, também chamado de
\enquote{\texttt{primario} ou \enquote{\texttt{principal}}}.

\begin{table}[H]\label{tab:cores}
	\centering
	\setlength{\tabcolsep}{1em}
	\begin{tabular}
		{*{2}{>{\ttfamily}c}>{\ttfamily}c>{\centering\arraybackslash}m{3cm}}
		\toprule
		\centering\normalfont Nome da cor & \normalfont Apelidos & \normalfont Código da cor (HTML) & \centering\arraybackslash Exemplo \\
		\midrule
		azul                              &
		\makecell[l]{%
		principal                                                                                                                       \\
		cor1                                                                                                                            \\
		primario                                                                                                                        \\
		penazul
		}                                 & \#2747CC             & \colorexample{primario}                                              \\
		\addlinespace
		cinza                             &
		\makecell[l]{%
		secundario                                                                                                                      \\
		cor2                                                                                                                            \\
		grafite                                                                                                                         \\
		boxbackground
		}                                 & \#F2F2F2             & \colorexample{cinza}                                                 \\
		\addlinespace
		azulclaro                         &
		\makecell[l]{%
		session1                                                                                                                        \\
		atento                                                                                                                          \\
		terciario                                                                                                                       \\
		linkscolor                                                                                                                      \\
		explore
		}                                 & \#0890FF             & \colorexample{azulclaro}                                             \\
		\addlinespace
		verde                             &
		\makecell[l]{%
		session2                                                                                                                        \\
		penverde                                                                                                                        \\
		praticando
		}                                 & \#00C642             & \colorexample{verde}                                                 \\
		\addlinespace
		vinho                             &
		\makecell[l]{%
		session3                                                                                                                        \\
		destacado                                                                                                                       \\
		penvinho                                                                                                                        \\
		know
		}                                 & \#FF0000             & \colorexample{vinho}                                                 \\
		\addlinespace
		laranja                           &
		\makecell[l]{%
		session4                                                                                                                        \\
		penlaranja                                                                                                                      \\
		arrange
		}                                 & \#FF6D00             & \colorexample{laranja}                                               \\
		\addlinespace
		rosa                              &
		\makecell[l]{%
		volume1                                                                                                                         \\
		topicos
		}                                 & \#EF117B             & \colorexample{rosa}                                                  \\
		\addlinespace
		roxo                              &
		\makecell[l]{%
		volume2                                                                                                                         \\
		estatistica
		}                                 & \#B83DF9             & \colorexample{roxo}                                                  \\
		\addlinespace
		verdeagua                         &
		\makecell[l]{%
		volume3                                                                                                                         \\
		penaqua                                                                                                                         \\
		geometria
		}                                 & \#51CEBC             & \colorexample{verdeagua}                                             \\
		\addlinespace
		amarelo                           &
		\makecell[l]{%
		volume4                                                                                                                         \\
		funcoes
		}                                 & \#FFC200             & \colorexample{amarelo}                                               \\
		\addlinespace
		cinzaescuro                       &
		\makecell[l]{%
		volume5                                                                                                                         \\
		computacional
		}                                 & \#939393             & \colorexample{cinzaescuro}                                           \\
		\bottomrule
	\end{tabular}
	\caption{Paleta de cores do projeto}
\end{table}


\subsection{Figuras}\label{sec:figuras}

As figuras correspondentes ao capítulo devem ficar na pasta
\verb|figuras| e ter todas nomes únicos, em letras maiúsculas, com
palavras separados por \enquote{\texttt{-}} e preferencialmente em
PDF.\@ Por exemplo, \verb|categoria-mulheres-na-maratona.pdf|.

\begin{description}
	\item [Dê nome às figuras que tenham alguma relação ao que elas representam] Para
	      que seja fácil identificar a figura no texto e nos arquivos, é importante que
	      o nome delas tenha algum significado atrelado à imagem. Por isso, evite figuras
	      com apenas números, ou nomeadas pela seção. Caso haja dificuldade na nomeação, use
	      o conteúdo como referência (título da atividade, por exemplo).
	\item [Dê legenda às figuras sempre que possível] É importante que as figuras possuam
	      uma breve descrição, especialmente por questões de acessibilidade.
	\item [Converta suas figuras para PDF] O melhor formato para se inserir figuras no
	      \LaTeX\ é PDF, pois ele simplesmente inclui a imagem (isto acontece também com
	      JPEG, porém PDF é preferível por ser vetorial). No caso de PNG, o programa
	      tem que converter a figura \textbf{toda vez que o documento é procesado}, por isso
	      \textbf{não utilize figuras em PNG}. Caso deseje converter suas figuras, há
	      diversas de se fazer isso, mas a mais ágil é com o programa
	      \href{https://github.com/myollie/img2pdf}{\texttt{img2pdf}}, que pode ser instalado%
	      \footnote{Requer python e outras dependências e
		      bibliotecas relacionadas. Consulte a documentação}
	      usando \mint{bash}/pip3 install img2pdf/
	\item [Priorize o uso de medidas relativas para tamanho de figuras] Para que as figuras
	      consigam ser modificadas com facilidade e para se ter referências para a
	      padronização é melhor o uso de medidas relativas no lugar de absolutas. Algumas
	      medidas relativas são \verb|em| (altura da letra \enquote{M} na fonte atual),
	      \verb|ex| (altura da letra \enquote{x} na fonte atual) e \latexinline/\linewidth/
	      (largura da linha atual). Também é possível somar e multiplicar por escalar estas
	      unidades, por exemplo, \latexinline/.8\linewidth/ utiliza \qty{80}{\percent}
	      da linha atual.
	\item [Não utilize \latexinline/\textwidth/ como medida] O material do aluno e do
	      professor possuem tamanhos de texto diferentes. Caso queira utilizar
	      exatamente a largura da página do aluno como medida, use o comando
	      \latexinline/\stutextwidth/.
  \item []
	\item [Não dê tanta importância ao posicionamento das figuras] durante a
	      edição, o \LaTeX\ as posiciona sozinha, e já está definido o alinhamento
	      \verb|!htb| como padrão, além de estarem já centralizadas (não é necessário
	      o comando\latexinline/\centering/). Sempre que possível deixe as figuras
	      flutuarem a partir do próprio \LaTeX, e utilize os comandos de referência
	      para citar as imagens. O posicionamento selecionado pelo \LaTeX\ é feito
	      a fim de deixar as páginas mais uniformes, a menos que seja muito importante
	      que a figura fique em um ponto específico no texto,\footnote{Neste caso, pode ser
		      usado o alinhando \texttt{H} definido pelo pacote
		      \href{https://linorg.usp.br/CTAN/macros/latex/contrib/float/float.pdf}{\texttt{float}}.}
	      o que em muitos casos não é imperativo.
	\item [Legendas devem ficar em cima da figura, usando o comando \latexinline/\caption/]
	      Já a fonte das figuras (ou descrições complementares) deve ser inserida com
	      o comando \latexinline/\legend/ logo abaixo da figura.
\end{description}

Figuras podem ser colocadas simplesmente como, por exemplo:

\begin{latexcode}
\begin{figure}
  \caption{Legenda}%
  \includegraphics[width=.6\linewidth]{nomedafigura}%
  \label{fig:nomedafigura}
  \legend{Fonte e/ou texto complementar à legenda}
\end{figure}
\end{latexcode}

\subsubsection{Figuras em \TikZ}

É recomendado que figuras em \TikZ\ devem estar em arquivos separados na pasta
\latexinline/tikz/ e devem ter a extensão \latexinline/.tex/, isto para facilitar
o processo de \emph{externalização}.\footnote{Incluindo as figuras no meio do texto,
	o \LaTeX\ irá ter que recalcular a figura toda vez que o texto é processado,
	o que aumenta bastante o tempo de processamendo com figuras mais complexas, por isso
	é recomendado que as figuras sejam feitas separadamente, tenham o PDF gerado e seja
	incluído o PDF em si.}

Caso queira produzir as figuras individualmente no Overleaf, use a
classe
\href{https://linorg.usp.br/CTAN/macros/latex/contrib/standalone/standalone.pdf}{\texttt{standalone}}
junto do pacote \verb|livroabertoem-preamble|, que irá incluir os
comandos de cores, fontes e as configurações básicas de padronização.%

\begin{latexcode}
  \documentclass[tikz]{standalone}
  \usepackage{livroabertoem-preamble}

  \begin{document}
    \begin{tikzpicture}
      codigo da figura
    \end{tikzpicture}
  \end{document}
\end{latexcode}

Concluída a figura, faça o \emph{download} do PDF e inclua como as
demais imagens, com o comando \latexinline/\includegraphics/.

\subsection{Tabelas}\label{sec:tabelas}

As tabelas\footnote{Para mais informações sobre como criar tabelas no
	\LaTeX, consulte \url{https://en.wikibooks.org/wiki/LaTeX/Tables}}
devem ser simples e criadas usando a interface do pacote
\href{https://linorg.usp.br/CTAN/macros/latex/contrib/booktabs/booktabs.pdf}{\latexinline/booktabs/}.
Elas seguem o padrão:%

\begin{minipage}{.4\linewidth}
	\begin{latexcode}
\begin{table}
  \begin{tabular}{ccc}
    \toprule
    A & B & C \\
    \midrule
    1 & 2 & 3 \\
    4 & 5 & 6 \\
    \bottomrule
  \end{tabular}
\end{table}
\end{latexcode}
\end{minipage}
\hfill
\begin{minipage}{.4\linewidth}
	\begin{table}[H]
		\centering
		\begin{tabular}{ccc}
			\toprule
			A & B & C \\
			\midrule
			1 & 2 & 3 \\
			4 & 5 & 6 \\
			\bottomrule
		\end{tabular}
	\end{table}
\end{minipage}

\begin{description}
	\item [Não utilize linhas verticais] Elas tornam a tabela muito distrativa e fogem do
	      padrão de tabelas científicas. Caso queira separar o conteúdo das colunas, aumente
	      o espaçamento, ou crie uma segunda tabela ao lado.
	\item [Evite o abuso de linhas horizontais] As linhas verticais têm a função de delimitar
	      o início, o fim e separar o cabeçalho, por isso, utilize apenas os comandos
	      \latexinline/\toprule/ para a linha no topo; para a debaixo,
	      \latexinline/\bottomrule/; para separar o cabeçalho da tabela,
	      \latexinline/\midrule/.

	      Também é possível linhas horizontais em apenas algumas colunas
	      utilizando o comando \latexinline/\cmidrule{a-b}/, onde \latexinline/a/
	      e \latexinline/b/ são as colunas que iniciam e terminam a linha,
	      respectivamente.\footnote{Mais informações na documentação do pacote
		      \texttt{booktabs}.}
	\item [Se precisar de espaço extra entre linhas, use o comando
        \latexinline/\addlinespace/] Este é um comando que insere uma linha
        horizontal \enquote{invisível}, e serve como uma separação adicional 
        caso haja textos longos.
	\item O espaço entre colunas pode ser ajustado com o comando
	      \latexinline/\setlength\tabcolsep{medida}/, onde a \verb|medida| padrão
	      é \qty{5}{pt}
	\item [O espaço geral entre as linhas pode ser alterado com o comando
        \latexinline/\renewcommand{\arraystretch}{n}/] Onde \verb|n| multiplica
	      o valor padrão do espaçamento (útil para tabelas que tenham muito texto
	      em todas as linhas, mas deve ser dada preferência ao comando
	      \latexinline/\addlinespace/).
\end{description}

Além dos tipos padrão, foram definidos os seguintes alinhamentos de coluna para
facilitar a criação de tabelas.

\begin{table}[H]
	\centering
	\small
	\begin{tabular}{cm{.25\linewidth}l}
		\toprule
		Tipo de coluna & Descrição                                                            & Definição                                                             \\
		\midrule
		\verb|C|       & Coluna \verb|m|, mas centralizada, em vez de alinhamento justificado & \latexinline/\newcolumntype{C}[1]{>{\centering\arraybackslash}m{#1}}/ \\
		\addlinespace
		\verb|e|       & Coluna \verb|c| no modo matemático                                   & \latexinline/\newcolumntype{e}{>{\(}c<{\)}}/                          \\
		\addlinespace
		\verb|E|       & Coluna \verb|C| no modo matemático                                   & \latexinline/\newcolumntype{E}[1]{>{\(}E{#1}<{\)}}/                   \\
		\bottomrule
	\end{tabular}
	\caption{Tipos de alinhamento definidos pelo pacote \texttt{livroabertoem-preamble}.}
\end{table}

\subsubsection{Tabelas longas}\label{sec:longtable}

O pacote \verb|longtable| não possui uma boa interação com o pacote
\verb|paracol|, por conta disso, é recomendado o uso do pacote
\href{https://linorg.usp.br/CTAN/macros/latex/contrib/supertabular/supertabular.pdf}{\texttt{supertabular}}.%

Para facilitar a criação desse tipo de tabelas, foi criado o comando
\latexinline/\supertabularheading/, que requer dois argumentos: o
número de colunas e a linha do cabeçalho, e deve ser usado antes do
início da tabela. No exemplo a seguir temos uma tabela com sete
colunas.

\begin{latexcode}
\begin{center}
  \supertabularheading{7}{%
    n_1 & n_2 & n_3 & n_4 & n_5 & n_6 & \text{Configuração}     \\
  }
  \topcaption{Exemplo de tabela longa}
  \begin{supertabular}{*{6}{e}@{\hspace{1em}}m{.15\linewidth}}
    3   & 7   & 42  &     &     &     & 3-7-42         \\
    3   & 8   & 24  &     &     &     & 3-8-24         \\
    3   & 9   & 18  &     &     &     & 3-9-18         \\
    3   & 10  & 15  &     &     &     & 3-10-15        \\
    3   & 12  & 12  &     &     &     & 3-12-12        \\
    4   & 5   & 20  &     &     &     & 4-5-20         \\
    4   & 6   & 12  &     &     &     & 4-6-12         \\
    4   & 8   & 8   &     &     &     & 4-8-8          \\
    5   & 5   & 10  &     &     &     & 5-5-10         \\
    6   & 6   & 6   &     &     &     & 6-6-6          \\
    3   & 3   & 4   & 12  &     &     & 3-3-4-12       \\
    3   & 3   & 6   & 6   &     &     & 3-3-6-6        \\
    3   & 4   & 4   & 6   &     &     & 3-4-4-6        \\
    4   & 4   & 4   & 4   &     &     & 4-4-4-4        \\
    3   & 3   & 3   & 3   & 6   &     & 3-3-3-3-6      \\
    3   & 3   & 3   & 4   & 4   &     & 3-3-3-4-4      \\
    3   & 3   & 3   & 3   & 3   & 3   & 3-3-3-3-3-3    \\
  \end{supertabular}
\end{center}
\end{latexcode}

\begin{center}
	\supertabularheading{7}{%
		n_1 & n_2 & n_3 & n_4 & n_5 & n_6 & \text{Configuração}     \\
	}
	\topcaption{Exemplo de tabela longa}
	\begin{supertabular}
		{*{6}{e}@{\hspace{1em}}m{.15\linewidth}}
		3   & 7   & 42  &     &     &     & 3-7-42         \\ %chktex 8
		3   & 8   & 24  &     &     &     & 3-8-24         \\ %chktex 8
		3   & 9   & 18  &     &     &     & 3-9-18         \\ %chktex 8
		3   & 10  & 15  &     &     &     & 3-10-15        \\ %chktex 8
		3   & 12  & 12  &     &     &     & 3-12-12        \\ %chktex 8
		4   & 5   & 20  &     &     &     & 4-5-20         \\ %chktex 8
		4   & 6   & 12  &     &     &     & 4-6-12         \\ %chktex 8
		4   & 8   & 8   &     &     &     & 4-8-8          \\ %chktex 8
		5   & 5   & 10  &     &     &     & 5-5-10         \\ %chktex 8
		6   & 6   & 6   &     &     &     & 6-6-6          \\ %chktex 8
		3   & 3   & 4   & 12  &     &     & 3-3-4-12       \\ %chktex 8
		3   & 3   & 6   & 6   &     &     & 3-3-6-6        \\ %chktex 8
		3   & 4   & 4   & 6   &     &     & 3-4-4-6        \\ %chktex 8
		4   & 4   & 4   & 4   &     &     & 4-4-4-4        \\ %chktex 8
		3   & 3   & 3   & 3   & 6   &     & 3-3-3-3-6      \\ %chktex 8
		3   & 3   & 3   & 4   & 4   &     & 3-3-3-4-4      \\ %chktex 8
		3   & 3   & 3   & 3   & 3   & 3   & 3-3-3-3-3-3    \\ %chktex 8
	\end{supertabular}
\end{center}


É válido ressaltar que não se deve usar o ambiente \verb|table|, mas sim \verb|center|,
e a legenda deve ser inserida com os comandos \latexinline/\topcaption/ e/ou
\latexinline/\bottomcaption/, também antes do início da tabela.

Nesta tabela é feito o uso de \verb|*{n}{a}| para gerar várias colunas
com o mesmo alinhamento, onde \texttt{n} é o número o multiplicador e
\texttt{a} é o alinhamento desejado. Também é usado
\verb|@{\hspace{1em}}| para definir um espaço de tamanho \qty{1}{em},
isto porque \verb|@{s}|, onde \verb|s| é o separador da coluna (é
permitido usar frases, símbolos matemáticos, espaçamento, etc.).

%TODO: \subsubsection{Quadros}



\subsubsection{Externalização das tabelas}

Se uma tabela for muito grande e/ou conter muito texto, ela pode tornar
mais difícil a navegação e visualização do resto do conteúdo do
arquivo, por isso, nestes casos, é recomendado colocar a tabela em um
arquivo separado (dentro de umas pasta \verb|tabelas|, a ser criada no
diretório \verb|figuras|). Feito isso, inclua a tabela com o comando
\latexinline/\input/ no local desejado. A edição da tabela pode ser,
assim como na figura em \TikZ, feita usando a classe \verb|standalone|,
só que neste caso não é necessário baixar o PDF e incluir como figura.
Se carregado o pacote \verb|standalone| (que acompanha a classe de
mesmo nome) dentro do arquivo \verb|capitulo-preamble.sty|, basta
utilizar o comando \latexinline/\input/ que o preâmbulo do arquivo será
ignorado. Para mais esclarecimentos, consulte a
\href{https://linorg.usp.br/CTAN/macros/latex/contrib/standalone/standalone.pdf}{documentação}
da(o) classe\slash pacote. % chktex 36

\subsection{Unidades}\label{sec:unidades}

O pacote
\href{https://linorg.usp.br/CTAN/macros/latex/contrib/siunitx/siunitx.pdf}{\texttt{siunitx}}
possui uma interface para lidar com números no meio do texto, bem como
unidades do Sistema Internacional, e outras ferramentas que são úteis
para a padronização do estilo, já que as alterações podem ser feitas
globalmente.%


\begin{description}
	\item [Utilize \latexinline/\num/ para inserir números com casas decimais
	      e/ou com mais de quatro dígitos] O \TeX\ insere um espaço após a vírgula
	      no modo matemático, o que é indesejado quando estamos escrevendo números
	      com casas decimais, pois diferente do inglês (e outras línguas) que utilizam
	      o \enquote{\texttt{.}} como separador de casa decimáis, a vírgula é quem cumpre
	      esse papel. Além disso, números que possuem mais de quatro algarismos devem
	      possuir um \enquote{separador}, que neste caso usamos \enquote{\texttt{.}}.
	      O comando \latexinline/\num/ faz essa conversão para nós, por isto, deve ser
	      utilizado na insersão de números nestes dois casos.

	      A diferença pode ser observada no seguinte exemplo:
	      \begin{itemize}
		      \item \latexinline/\(10000,23\)/ insere \(10000,23\);
		      \item \latexinline/\num{10000,23}/ insere \num{10000,23}.
	      \end{itemize}

	\item [Há variações de \latexinline/\num/ que podem ser usadas para montar listas, 
        intervalos e produtos]\label{dsc:variacoes-num} Os comandos
	      \latexinline/\numlist{1;2;3;4}/, \latexinline/\numrange{1}{4}/ e
	      \latexinline/\numproduct{1 x 2 x 3}/ nos dão, respectivamente,
	      \enquote{\numlist{1;2;3;4}}; \enquote{\numrange{1}{4}};
	      \enquote{\numproduct{1 x 2 x 3}}.

	\item [Para escrever números em notação científica, pode-se usar \enquote{\texttt{e}} para
	      a potência de 10]
	      Escrevendo \latexinline/\num{3e4}/, temos como resultado \num{3e4}. Para potências
	      negativas, utilize \latexinline/d/ em vez de \latexinline/e/.

	\item [Ângulos devem ser escritos o comando \latexinline/\ang{angulo}/] Para minutos e
	      segundos, pode-se dividir tanto usando \enquote{\texttt{;}}, quanto \enquote{\texttt{,}}:
	      \latexinline/\ang{3;2;1}/ nos dá \ang{3;2;1}

	\item [Unidades de medida devem ser escritas com \latexinline/\unit{unidade}/] As unidades
	      mais comuns já estão definidas como comandos intuitivos (\latexinline/\mm/
	      para milímetros \latexinline/\s/ para segundos, \ldots). Tabelas com todas as macros
	      definidas para unidades do pacote estão presentes nas páginas \numrange{7}{13}
	      da documentação.

	\item [As unidades podem ser modificadas por outros comandos definidos no pacote]
	      Para levar uma unidade ao quadrado utilize \latexinline/\square/,
	      \latexinline/\cubic/ para ao cubo, e \latexinline/\per/ para dividir
	      uma unidade por outra. Comandos do módo matemático também funcionam
	      (Estão disponíveis exemplos na seção 3.3 da documentação)

	\item [Declare unidades não definidas pela classe] Novas unidades podem ser declaradas
	      a partir do comando \latexinline/\DeclareSIUnit/, para que evitemos possíveis
	      conflitos. O comando \latexinline/\unit/ não exige que seu argumento seja um
	      outro comando, mas a criação destes nos ajuda na padronização.

	\item [Para escrever quantidades, use o comando \latexinline/\qty{num}{unidade}/] O comando
	      \latexinline/\qty/ é uma combinação de \latexinline/\num/ e \latexinline/\unit/, e
	      pode ser utilizado tanto no meio do texto quanto no modo matemático. Este é o
	      principal motivo de usarmos o pacote \verb|siunitx|, por isso, sempre que for
	      falar de alguma quantidade atrelada à unidade, isto deve ser feito com
	      \latexinline/\qty/. Como exemplo, \latexinline/\qty{3}{\mm\square\per\second}/
	      gera \qty{3}{\mm\square\per\second}.

	\item [As variantes de \latexinline/\num/ possuem correspondentes com o comando
	      \latexinline/\qty/] São definidos \latexinline/\qtylist/, \latexinline/\qtyrange/ e
	      \latexinline/\qtyproduct/, e funcionam da mesma maneira do que os comandos
	      equivalentes de número. A diferença é que deve ser adicionado um argumento
	      a mais com a unidade desejada.
\end{description}


\subsubsection{Valores monetários}

O pacote \href{https://linorg.usp.br/CTAN/macros/latex/contrib/currency/currency_doc.pdf}%
{\latexinline/currency/}, que utiliza o \verb|siunitx| como base. Com ele definimos comandos
padrão para lidar com números atrelados a uma determinada moeda. Até o
momento, estão definidos os comandos para Real, Dólar americano e Euro.
Os Comandos seguem a lógica \latexinline/\vXYZ{valor}/, onde
\texttt{XYZ} representa o código ISO da moeda (BRL, USD e EUR,
respectivamente). Para exemplificar o funcionamento, usaremos a seguir
o real como unidade, isto é, o código BRL, mas vale para
qualquer outro código pré-definido.

\begin{description}
	\item [Alterne entre o uso por extenso ou por símbolo através de opções]
	      É possível alternar entre diversas formas de inserir o valor, isto é, a opção
        \enquote{\texttt{@na}} irá digitar usando o nome da moeda, e \enquote{\texttt{@pl}}
	      fará o mesmo, só que no plural.\footnote{Há também a opção \texttt{@iso}, que usa
        o código ISO da moeda, mas seu uso não é recomendado.} Com isso, é possível trazer 
        variedades no texto para que não fique repetitivo. Exemplificando:
	      \begin{itemize}
		      \item \latexinline/\vBRL{10003,28}/ gera \vBRL{10003,28}.
		      \item \latexinline/\vBRL[@na]{1,28}/ gera \vBRL{1,28}.
		      \item \latexinline/\vBRL[@pl]{10003,28}/ gera \vBRL[@pl]{10003,28}.
	      \end{itemize}

  \item [Para inserir apenas o símbolo ou o nome da moeda, use \latexinline/\cBRL/]
        Quando estiver falando da moeda, use este comando com as mesmas opções
        do anterior. Também é útil em combinação com as
        \hyperref[dsc:variacoes-num]{variações de \latexinline/\num/}.
\end{description}

Uma forma de lembrar qual é a diferença entre os dois comandos é ler o \verb|v| de
\latexinline/\vBRL/ como \emph{value} (valor) e o \verb|c| de 
\latexinline/\cBRL/ como \emph{currency} (moeda).

