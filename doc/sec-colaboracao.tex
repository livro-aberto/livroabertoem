\section{Colaboração no projeto}\label{sec:producao-capitulo}

Nesta seção iremos detalhar a estrutura padrão dos projetos criados no
Overleaf para a colaboração entre os autores. Isto envolve

\subsection{Estrutura da pasta}\label{sec:estrutura-da-pasta}

Os arquivos têm a seguinte estrutura.

\dirtree{%
	.1 capitulo/.
	.2 class-options/.
	.3 [arquivos de configuração de opções].
	.2 definitions/.
	.3 [arquivos de configuração da classe e do pacote].
	.2 figuras/.
	.3 tikz/.
	.4 [arquivos das figuras em TikZ].
	.3 tabelas/.
	.4 [arquivos de tabelas grandes].
	.3 [figuras do capítulo].
	.2 resources/.
	.3 figures/.
	.4 [imagens padrão da classe].
	.3 fonts/.
	.4 [arquivos das fontes usadas].
	.2 main.tex.
	.2 capitulo-bibliografia.bib.
	.2 capitulo-preamble.sty.
	.2 livroabertoem.cls.
	.2 livroabertoem-preamble.sty.
	.2 apresentacao.tex.
	.2 sec1-nome-da-secao1.tex.
	.2 sec2-nome-da-secao2.tex.
	.2 sec3-nome-da-secao3.tex.
	.2 sec4-exercicios.tex.
	.2 sec-encerramento.tex.
}%%

\begin{description}
	\item[Divida o texto em vários arquivos] Crie um arquivo novo no início de
		cada seção, isto é, no começo de cada explorando, ou em um comando
		\latexinline/\section/ de introdução. Isto facilita o processo de
		edição.
	\item[Separe o preâmbulo do arquivo principal] Todos os pacotes e comandos
		que forem usados especificamente no capítulo devem estar em um arquivo
		de estilo separado, \verb|capitulo-preamble.sty|, contido na pasta do
		projeto. Isto é para deixar o \verb|main.tex| o mais limpo possível.
	\item[Evite o acúmulo de pacotes] Se o pacote não está sendo usado, não há
		necessidade de ser carregado, isto apenas aumenta o tempo de
		processamento.
	\item As figuras em \latexinline/tikz/ devem estar em arquivos separados
	      dentro na pasta \latexinline/tikz/. Mais detalhes na seção sobre
	      \protect\hyperlink{figuras}{figuras}.
\end{description}


\subsection{Arquivo \texttt{main.tex}}\label{sec:main-tex}

O arquivo \verb|main.tex| é o que será processado, então é onde
declaramos a classe, os pacotes, os comandos descritos nas
\cref{sec:pre-volume,sec:pre-capitulo}, e os \verb|input| das seções
individuais. Sendo assim, começamos este arquivo com as seguintes
linhas:

\begin{latexcode}
\documentclass{livroabertoem}

\usepackage{livroabertoem-preamble}
\usepackage{capitulo-preamble}
\end{latexcode}

Para acessar o conteúdo do professor, basta inserir a seguinte opção:

\begin{latexcode}
\documentclass[professor]{livroabertoem}
\end{latexcode}

As seções podem ser inclusas com o comando
\latexinline/\input{arquivo}/ após os créditos.

A seguir temos um exemplo de como é a estrutura básica do arquivo
\verb|main.tex|:

\latexfile{exemplo.tex}


\subsection{Conteúdo do aluno}\label{sec:conteudo-aluno}

O conteúdo do aluno é dividido usando os comandos descritos nas
\cref{sec:comandos-divisao,sec:comandos-ambientes}. No material do
aluno, temos as seguintes possibilidades de ambientes com título:

\begin{itemize}
	\item \latexinline/task/ Atividades;
	\item \latexinline/example/ Exemplo;
	\item \latexinline/theorem/ Teorema;
	\item \latexinline/definition/ Definição;
	\item \latexinline/observation/ Observação;
\end{itemize}

Eles são da forma

\begin{latexcode}
\begin{task}{título}
  ...
\end{task}
\end{latexcode}

Temos também ambientes sem título, sendo eles:

\begin{itemize}
	\item \latexinline/reflection/ Refletindo.
	\item \latexinline/knowledge/ Você sabia?;
	\item \latexinline/research/ Notas;
	\item \latexinline/project/ Projeto aplicado;
\end{itemize}

sendo da forma

\begin{latexcode}
\begin{knowledge}
  ...
\end{knowledge}
\end{latexcode}

\subsection{Conteúdo do professor}\label{sec:conteudo-professor}

O conteúdo do professor deve estar sempre presente no ambiente
\latexinline/teacher/. Há três ambientes de conteúdo do professor:

\begin{itemize}
	\item \latexinline/objectives/ Objetivos específicos;
	\item \latexinline/sugestions/ Sugestões e discussões (pode ser
	      usado o comando \latexinline/\sugestion{titulo}/ para criar um
	      título para a sugestão);
	\item \latexinline/answer/ Solução.
\end{itemize}

Sendo assim, o conteúdo do professor fica na estrutura:

\begin{latexcode}
\begin{teacher}
  ...

  \begin{objectives}{titulo}
    ...
  \end{objectives}

  \begin{sugestions}{titulo}
    \sugestion{sugestao} ...
  \end{sugestions}

  \begin{answer}{titulo}
    ...
  \end{answer}
\end{teacher}
\end{latexcode}

Estes ambientes são mais comumente usados em atividades, por isso o
ambiente \latexinline/teacher/ deve estar dentro do ambiente
\verb|task| neste caso. Se for colocar conteúdo para o professor dentre
de qualquer um dos outros ambientes do aluno, coloque antes do início
do ambiente (não haverá nenhum erro, apenas será mais fácil para a
transição no modelo final, em que isto não é permitido).

\subsubsection{O ambiente \texttt{teacher*}}\label{sec:teacher-estrela}

O pacote \verb|paracol| nos permite criar textos assíncronos em duas
colunas diferentes, isto é, \emph{colunas paralelas}. Por conta disso,
o texto da coluna do professor possui uma quebra própria, e nem sempre
irá estar sincronizado corretamente com o texto do aluno. A solução
para isto está no ambiente \verb|teacher*|, que sincroniza a coluna do
professor no local em que o ambiente é inserido. Portanto, se quiser
que o texto do professor comece na mesma linha de uma atividade, pode
ser usado este ambiente logo após o início da atividade:

\begin{latexcode}
  \begin{task}
    \begin{teacher*}
      ...
    \end{teacher*}
    ...
  \end{task}
\end{latexcode}

\subsubsection{Habilidades da BNCC}\label{sec:habilidades}

Um último ambiente para o professor é o ambiente
\latexinline/habilities/, que deve ser usado dentro de uma seção do
professor (o ambiente \latexinline/teacher/), para descrever as
habilidades da BNCC trabalhadas neste capítulo. Ele funciona da
seguinte forma:

\begin{latexcode}
\begin{habilities}[titulo]
  \hability{###}
  \efhability{##}{##}
\end{habilities}
\end{latexcode}

O ambiente cria um título \enquote{Habilidades da BNCC}, mas que pode
ser mudado a partir de uma opção entre colchetes com o título desejado.
Dentro deste ambiente, os comandos \latexinline/\hability{###}/ e
\latexinline/\efhability{##}{##}/ inserem o texto da habilidade
desejada do ensino médio e do ensino fundamental, respectivamente, onde
\verb|#| é um dígito, e o argumento obrigatório deve ser o número da
habilidade. Por exemplo, \latexinline/\habiliy{101}/ corresponde à
habilidade \textbf{EM13MAT101}, e \latexinline/\efhability{01}{02}/
corresponde à habilidade \textbf{EF01MA02}.

Estes comandos possuem um funcionamento diferente caso usados fora do
ambiente \verb|habilities|. Neste caso, eles apenas inserem o código da
habilidade em negrito, e deve ser usado quando estiver citando uma
habilidade.

