% arara: lualatex: {shell: yes, synctex: yes}
\documentclass[12pt]{article}
\usepackage{manual-colaboracao-preamble}
\title{Manual para a colaboração no livro do ensino médio do projeto
	Livro Aberto de Matemática}
\author{Tarso Boudet Caldas}
\date{\today}

\begin{document}
\maketitle

\tableofcontents

\section{Introdução}

Este é um manual para a colaboração na produção dos capítulos dos
livros do ensino médio da Associação Livro Aberto. Os capítulos
presentes no Overleaf utilizam comandos e ambientes próprios para a
sintaxe dos livros do projeto, que serão melhor detalhados aqui para
facilitar o processo colaborativo. Não nos propomos a ensinar
\LaTeX~neste manual, mas recomendamos
\emph{\href{http://alfarrabio.di.uminho.pt/~albie/lshort/pt-lshort-a5.pdf}{Uma
    não tão pequena introdução ao \LaTeXe}} como referência.

A versão do modelo utilizada é apenas um rascunho, para facilitar a
produção de conteúdo, por isto os títulos são simplificados e não
refletem o estado final do material, por isso, não se preocupe com a
formatação, pois isto será corrigido numa etapa de pós-produção.

\section{Produção do capítulo}\label{produuxe7uxe3o-do-capuxedtulo}

Nas seções a seguir, descreveremos o funcionamento do modelo para a
produção dos capítulos, tratando principalmente dos comandos
utilizados.

\subsection{Estrutura da pasta}\label{estrutura-da-pasta}

Os arquivos têm a seguinte estrutura. \dirtree{
  .1 nome-do-capitulo. 
  .2 class-options
  .2 definitions
  .2 resources.
  .3 figures. 
  .3 fonts.
  .3 tikz. 
  .4 figura-tikz.tikz.
  .2 biblatex.cfg
  .2 main.tex. 
  .2 nome-do-capitulo-preamble.sty. 
  .2 livroabertoem-preamble.stsy
  .2 livroabertoem.cls
  .2 apresentacao.tex. 
  .2 sec1-nome-da-secao1.tex. 
  .2 sec2-nome-da-secao2.tex. 
  .2 sec3-nome-da-secao3.tex. 
  .2 sec4-exercicios.tex. 
  .2 sec-encerramento.tex.
	.2 nome-do-capitulo-bibliografia.bib. 
}%

\begin{itemize}
	\item Crie um arquivo novo no início de cada seção, isto é, no começo de cada
	      explorando, ou em um comando \latexinline/\section/ de introdução.
	\item Coloque pacotes e comandos que você criar no arquivo
	      \latexinline/nome-do-capitulo-preamble.sty/. (\emph{Evite a acumulação de 
          pacotes que não estão sendo utilizado no capítulo})
	\item As figuras em \latexinline/tikz/ devem estar em arquivos separados
	      dentro na pasta \latexinline/tikz/. Mais detalhes na seção sobre
        \hyperref[figuras]{figuras}.
  \item As pastas \latexinline/definitions/ e \latexinline/class-options/ contém as
        definições da classe \latexinline/livroabertoem/,w
\end{itemize}


\subsection{Preâmbulo}\label{preuxe2mbulo}

No arquivo \latexinline/main.tex/ teremos a definição da classe e
incluiremos as seções. A classe que usaremos é
\latexinline/livroabertoem/, criada pelo projeto, e chamamos o pacote
\latexinline/livroabertoem-preamble/, que contém as definições usadas
em todos os capítulos, como para língua; tabelas; créditos;
bibliografia e Tik\emph{Z}, etc. Por fim, antes do início do documento
requisitamos o pacote com o preâmbulo do capítulo, que terá as
declarações que não são usadas em todos os capítulos. Sendo assim,
nosso preâmbulo será:

\begin{latexcode}
\documentclass{livroabertoem}

\usepackage{livroabertoem-preamble}
\usepackage{nome-do-capitulo-preamble}
\end{latexcode}

\subsection{\texorpdfstring{Classe \latexinline/livroabertoem/}{Classe livroabertoem}}%
\label{classe-livroabertoem}

A classe livroabertoem por si só gera apenas o conteúdo do aluno, para
se ter o material do professor é necessário adicionar a opção
\latexinline/professor/:

\begin{latexcode}
\documentclass[professor]{livroabertoem}
\end{latexcode}

Ela é baseada na classe \latexinline/memoir/, portanto, comandos
da \latexinline/memoir/ também funcionarão no seu documento.

\subsubsection{Comandos}

\subsection{Início do capítulo}\label{inuxedcio-do-capuxedtulo}

Logo após \latexinline/\begin{document}/, colocamos o capítulo com o
comando \latexinline/\chapter/ e adicionamos o texto de introdução com
os comandos.

\begin{itemize}
	\item \latexinline/\chapterwhat/ ``O quê?'' --- Sobre o que é o
	      capítulo?
	\item \latexinline/\chapterbecause/ ``Por quê?'' --- Motivação para
	      o estudo do capítulo.
\end{itemize}

Depois disso, iniciamos a configuração da página de créditos.

\subsection{Página de créditos}\label{puxe1gina-de-cruxe9ditos}

Além do comando \latexinline/\author/, que funciona semelhantemente ao
de outras classes de \LaTeX, temos alguns comandos para a geração da
página de créditos:

\begin{itemize}
	\item \latexinline/\revision/ Nome do(s) revisor(es). Assim como os
	      autores, múltiplos nomes podem ser separados com o comando
	      \latexinline/\and/
	\item \latexinline/\version/ Versão atual do capítulo
	\item \latexinline/\graphicsauthor/ Autor(es) das figuras técnicas
	\item \latexinline/\illustrationauthor/ Ilustrador(es)
	\item \latexinline/\collaboration/ Colaborador(es)
	\item \latexinline/\ccbysa/ Comando para capítulos que utilizam a
	      licença cc-by-sa (demais capítulos têm a licença cc-by-nc-sa)
	\item \latexinline/\creditos/ Gera a página de créditos a partir das
	      definições dos comandos anteriores. (esta página dá erro caso não haja
	      o nome de um autor)
\end{itemize}

\textbf{Observação}: Os comandos da página de créditos podem ser
colocados também no preâmbulo, apenas o comando
\latexinline/\creditos/ precisa ficar depois do início do
capítulo.

\subsection{Inclusão das seções}\label{inclusuxe3o-das-seuxe7uxf5es}

As seções podem ser inclusas com o comando
\latexinline/\include{arquivo}/ após os créditos. Para compilar as
seções separadamente, é possível utilizar o comando
\latexinline/\includeonly{arquivo}/ (\emph{isto deve ser feito antes no
	preâmbulo})

\subsection{Divisões do capítulo}\label{divisuxf5es-do-capuxedtulo}

Os livros possuem a seguinte hierarquia: \dirtree{.1 Volume. .2
	Capítulo 1. .3 Introdução ao Professor. .3 Seção 1. .4 Explorando 1. .4
	Organizando 1. .4 Praticando 1. .3 Seção 2. .4 Explorando 1. .4
	Organizando 1. .4 Praticando 1. .3 Saiba Mais. .3 Exercícios. .3
	Material Suplementar. .3 Notas. .3 Bibliografia. }%

\begin{itemize}
	\item Na edição pelo Overleaf, o Volume não será utilizado, pois, cada
	      projeto será um capítulo diferente.
	\item As seções de Saiba Mais e Exercícios podem também ser colocadas entre
	      seções.
	\item As notas não serão a princípio utilizadas na versão do Overleaf.
	\item Os comandos para a divisão de seções são:

	      \begin{itemize}
		      \item a divisão de capítulos e seções é feita normalmente com os comandos
		            \latexinline/chapter/ e \latexinline/section/;
		      \item \latexinline/\explore{titulo}/ Explorando;
		      \item \latexinline/\arrange{titulo}/ Organizando;
		      \item \latexinline/\practice{titulo}/ Praticando;
		      \item \latexinline/\know{titulo}/ Saiba Mais;
		      \item \latexinline/\exercise/ Exercícios (este não possui título)
		      \item \latexinline/\additionalmaterial/ Material Suplementar (este
		            não possui título)
		      \item a introdução ao professor é fica no ambiente
		            \latexinline/teachersection/, que também pode ser usado para material
		            do professor entre as seções. Só aparece quando usada a opção
		            \latexinline/professor/.
	      \end{itemize}
\end{itemize}

\subsection{Conteúdo do aluno}\label{conteuxfado-do-aluno}

No material do aluno, temos as seguintes possibilidades de ambientes
com título:

\begin{itemize}
	\item \latexinline/task/ Atividades;
	\item \latexinline/example/ Exemplo;
	\item \latexinline/observation/ Observação;
	\item \latexinline/reflection/ Refletindo.
\end{itemize}

Eles são da forma

\begin{latexcode}
\begin{task}{título}
conteúdo
\end{task}
\end{latexcode}

Temos também ambientes sem título, sendo eles:

\begin{itemize}
	\item \latexinline/knowledge/ Você sabia?;
	\item \latexinline/research/ Notas;
	\item \latexinline/project/ Projeto aplicado;
\end{itemize}

sendo da forma

\begin{latexcode}
\begin{knowledge}
conteúdo
\end{knowledge}
\end{latexcode}

\subsection{Conteúdo do professor}\label{conteuxfado-do-professor}

O conteúdo do professor deve estar sempre presente no ambiente
\latexinline/teacher/. Há três ambientes de conteúdo do professor:

\begin{itemize}
	\item \latexinline/objectives/ Objetivos específicos;
	\item \latexinline/sugestions/ Sugestões e discussões (pode ser
	      usado o comando \latexinline/\sugestion{titulo}/ para criar um
	      título para a sugestão);
	\item \latexinline/answer/ Solução.
\end{itemize}

Sendo assim, o conteúdo do professor fica na estrutura:

\begin{latexcode}
\begin{teacher}
  Texto

  \begin{objectives}{titulo}
  objetivos
  \end{objectives}

  \begin{sugestions}{titulo}
  \sugestion{sugestao} texto
  \end{sugestions}

  \begin{answer}{titulo}
  solucao
  \end{answer}
\end{teacher}
\end{latexcode}

Estes ambientes são mais comumente usados em atividades, por isso o
ambiente \latexinline/teacher/ deve estar no ambiente da atividade
neste caso. Se for colocar conteúdo para o professor dentre de qualquer
um dos outros ambientes do aluno, coloque antes do início do ambiente
(não haverá nenhum erro, apenas será mais fácil para a transição no
modelo final, em que isto não é permitido).

Um último ambiente para o professor é o ambiente
\latexinline/habilities/, que deve ser usado dentro de uma seção do
professor (o ambiente \latexinline/teacher/), para descrever as
habilidades da BNCC trabalhadas neste capítulo. Ele funciona da
seguinte forma:

\begin{latexcode}
\begin{habilities}
  \hability{habilidade 1} texto da habilidade 1
  \hability{habilidade 2} texto da habilidade 2
\end{habilities}
\end{latexcode}

\section{Orientações para a padronização do código}%
\label{orientauxe7uxf5es-para-a-padronizauxe7uxe3o-do-cuxf3digo}

A seguir temos algumas recomendações gerais para facilitar um código
limpo que ajude na colaboração entre muitos autores. Não são regras que
precisam ser seguidas à risca, nem recomendamos reescrever código para
adaptá-lo a essas direções, apenas tenha elas em mente quando for
adicionar conteúdo.

\subsection{Formatação do código}\label{formatauxe7uxe3o-do-cuxf3digo}

\begin{itemize}
	\item Tente deixar as linhas com cerca de 70--90 caracteres.
	\item Utilize indentação para os ambientes.
	\item Utilize \latexinline/\cref/ em vez de \latexinline/\ref/ para
	      referenciar qualquer elemento. (seções, atividades, exemplos, figuras,
	      etc.). Não é necessário escrever Figura \latexinline/\cref{label}/,
	      basta \latexinline/\cref{label}/.
	\item Adicione sempre um prefixo aos \emph{labels} para facilitar o
	      reconhecimento, por exemplo, use o label
	      \latexinline/\label{fig:figura}/ para referenciar figuras,
	      \latexinline/tab:/ para tabelas, \latexinline/eq:/ para equações
	      \latexinline/exp:/ para o Explorando, \latexinline/org:/ para o
	      Organizando, \latexinline/prac/ para Praticando, \latexinline/know/
	      para Saiba Mais. A seção de exercícios pode ter simplesmente o label
	      \latexinline/nomedocapitulo:exerc/.
	\item Use o modo matemático com \latexinline/\( \)/ em vez de
	      \latexinline/$ $/ e \latexinline/\[ \]/ em vez de \latexinline/$$ $$/.
	\item Divida o código em seções a partir do explorando ou divisões que façam
	      sentido na estrutura do capítulo (evite arquivos com muitas linhas de
	      código).
\end{itemize}

\subsection{Figuras.}\label{figuras}

\begin{itemize}
	\item Dê nome às figuras que tenham alguma relação ao que elas representam.
	\item Dê legenda às figuras sempre que possível.
	\item Priorize o uso de medidas relativas para tamanho de figuras. (uma
	      porcentagem de \latexinline/\linewidth/ pode ser usado, por exemplo,
	      \latexinline/.8\linewidth/ utiliza 80\% da linha)
	\item Não dê tanta importância ao posicionamento das figuras durante a
	      edição, o \LaTeX\ as posiciona sozinho
	\item Figuras podem ser colocadas simplesmente como (por exemplo)
\end{itemize}

\begin{latexcode}
\begin{figure}
  \includegraphics[width=.6\linewidth]{nomedafigura}
  \caption{Legenda}%
  \label{fig:nomedafigura}
\end{figure}
\end{latexcode}

\begin{itemize}
	\item Figuras em TikZ devem estar em arquivos separados na pasta
	      \latexinline/tikz/ e devem ter a extensão \latexinline/.tikz/, com o
	      preâmbulo da classe \latexinline/standalone/ e inclusas com o comando
	      \latexinline/\includetikzpicture{nome-da-figura}/. \textbf{Não inclua o
		      nome da extensão}\latexinline/.tikz/.

\begin{latexcode}
  \documentclass[tikz]{standalone}

  \begin{document}
    \begin{tikzpicture}
      codigo da figura
    \end{tikzpicture}
  \end{document}
\end{latexcode}
\end{itemize}

\subsection{Tabelas}\label{tabelas}

\begin{itemize}
	\item Crie tabelas simples a partir do pacote
	      \href{https://linorg.usp.br/CTAN/macros/latex/contrib/booktabs/booktabs.pdf}{\latexinline/booktabs/}.
	      Elas seguem o padrão:

	      \begin{latexcode}
  \begin{table}
    \begin{tabular}{ccc}
      \toprule
      A & B & C \\
      \midrule
      1 & 2 & 3 \\
      4 & 5 & 6 \\
      \bottomrule
    \end{tabular}
  \end{table}
\end{latexcode}
	      \begin{itemize}
		      \item Não utilize linhas verticais, e as horizontais devem ser uma linha no
		            topo (\latexinline/\toprule/) uma embaixo (\latexinline/\bottomrule/) e
		            uma para separar o cabeçalho da tabela (\latexinline/\midrule/). Também
		            é possível linhas horizontais em apenas algumas colunas utilizando o
		            comando \latexinline/\cmidrule{a-b}/, onde \latexinline/a/ e
		            \latexinline/b/ são as colunas que iniciam e terminam a linha,
		            respectivamente.
	      \end{itemize}
\end{itemize}

\subsection{Unidades}\label{unidades}

Utilizamos principalmente o pacote \latexinline/siunitx/ para
padronização de unidades. São os comandos que seguem:

\begin{itemize}
	\item Utilize \latexinline/\num{numero}/ para inserir números pelo texto
	      (quando lidando com quantidades). Devem seguir o formado, por exemplo,
	      de \latexinline/\num{100000,23}/.

	      \begin{itemize}
		      \item Para multiplicar para uma potência de 10 é possível escrever
		            \latexinline/\num{3e4}/, que resultará em 3×10⁴. Para potências
		            negativas, utilize \latexinline/d/ em vez de \latexinline/e/.
	      \end{itemize}
	\item Ângulos devem ser escritos o comando
	      \latexinline/\ang{angulo}/.
	\item Unidades podem ser escritas \latexinline/\unit{unidade}/. Macros para
	      unidades mais usados são definidas por
	\item comandos intuitivos (\latexinline/\mm/ para milímetros \latexinline/\s/
	      para segundos). Tabelas com todas as macros
	\item definidas para unidades do pacote estão presentes nas páginas 7 a 13 da
	      documentação.

	      \begin{itemize}
		      \item Para levar uma unidade ao quadrado utilize \latexinline/\square/,
		            \latexinline/\cubic/ para ao cubo, e \latexinline/\per/ para dividir
		            uma unidade por outra. Comandos do módo matemático também funcionam
		            (Estão disponíveis exemplos na seção 3.3 da documentação)
	      \end{itemize}
	\item Para escrever um número atrelado a uma unidade,
	      use\latexinline/\qty{numero}{unidade}/.
	\item Para valores monetários, usamos os comandos do pacote
	      \href{https://linorg.usp.br/CTAN/macros/latex/contrib/currency/currency_doc.pdf}{\latexinline/currency/}:

	      \begin{itemize}
		      \item \latexinline/\vBRL{valor}/,
		            \latexinline/\vUSD{valor}/ ou
		            \latexinline/\vEUR{valor}/ para colocar os símbolos de
		            reais, dólares e euros respectivamente (R\$, \$, €);
		      \item se quiser utilizar o nome das moedas, (1 real, por exemplo), use
		            \latexinline/\vXXX[@na]{valor}/ para singular e
		            \latexinline/\vXXX[@pl]{valor}/ para plural, onde XXX é a moeda
		            desejada;
		      \item caso queira usar só a unidade, sem valor, é possível usar
		            \latexinline/\cXXX/ (as mesmas opções são válidas).
	      \end{itemize}
\end{itemize}

\end{document}
