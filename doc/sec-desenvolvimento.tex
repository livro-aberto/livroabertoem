\section{Desenvolvimento do projeto}\label{sec:desenvolvimento-projeto}

\subsection{Classe}

A classe \latexinline/livroabertoem.cls/ é baseada na classe
\latexinline/memoir.cls/, portanto, utiliza ao máximo os comandos de
formatação da mesma, e os comandos da classe funcionam no documento. A
classe possui diversos comandos úteis para a escrita de todo tipo de
texto, e a
\href{https://linorg.usp.br/CTAN/macros/latex/contrib/memoir/memman.pdf}{documentação}
é bem completa, e pode ser consultada caso necessário\footnote{O manual
	deve servir como referência para a criação de elementos textuais, mas
	não devem ser utilizados comandos para modificar a diagramação, isto é,
	fonte, tamanho de página, etc. }. Inclusive é uma classe que emula
outros pacotes comuns do \LaTeX, portanto, é recomendado verificar se a
própria classe possui comandos nativos que façam o que você precisa,
evitando assim a inserção de novos pacotes. Além da classe
\verb|memoir|, são usados os seguintes pacotes: \texttt{comment,
	xstring, xpatch, ragged2e, xcolor, graphicx, calc, tikz, tcolorbox,
	pagecolor, anyfontsize, fontspec} e, por fim, \verb|paracol|, que
permite que tenhamos a coluna do professor em paralelo à do aluno.%

O pacote
\href{https://linorg.usp.br/CTAN/macros/latex/contrib/paracol/paracol-man.pdf}{\texttt{paracol}}
possui algumas incompatibilidades, especialmente com os pacotes
\verb|multicol| e \verb|longtable|. No segundo caso, isto pode ser
contornado usando o pacote \verb|supertabular| (ver
\cref{sec:longtable}). Todos os comandos do \verb|paracol| funcionam
nesta classe, mas também acompanha suas limitações, por isso é
recomendado ter a documentação do pacote como consulta caso haja alguma
dúvida.%

Estão disponíveis três opções, a primeira é a que define a classe, a
opção \verb|professor| (ou \verb|teacher|), que cria a coluna lateral e
mostra as páginas de conteúdo exclusivas do material do professor. As
outras duas opções foram criadas para diminuir o tempo de processamento
do documento, facilitando o processo de criação do conteúdo, sendo elas
a opção \verb|rascunho| e a opção \verb|nofonts|:
\begin{itemize}
	\item \verb|rascunho| retira todos os elementos visuais do livro que possam aumentar o tempo
	      de processamento, com exceção das fontes. Além disso, há também a opção \verb|draft|, que trará
	      os mesmos efeitos de \verb|rascunho|, mas que não insere as figuras (por ser uma opção
	      vinda da classe \verb|memoir|).
	\item \verb|nofonts|, como diz o nome, retira todas as fontes do texto e substitui para
	      fontes padrão do \LaTeX, não carregando o pacote \verb|fontspec|.
\end{itemize}

Todos os comandos definidos pela classe estão descritos na
\cref{sec:apressados}.

\subsection{Pacote de estilo}

O pacote \verb|livroabertoem-preamble.cls| serve para as configurações
de pacotes comuns a todos os capítulos. Aqui são definidas as
padronizações de listas, figuras, tabelas, página de créditos, figuras
em \TikZ, bibliografia, modo matemático, hyperlinks, etc. Os pacotes
utilizados para tal são \texttt{tikz, enumitem, makecell, xurl, float,
	caption, microtype, babel, biblatex, csquotes, amsmath, amsthm,
	mathtools, unicode-math, hyperref, cleveref, siunitx} e
\verb|currency|. Para uma discussão das padronizações e escolhas
feitas, veja a \cref{sec:padronizacao}.

