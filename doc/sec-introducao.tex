\section{Introdução}

Este é um manual para a colaboração na produção dos capítulos dos
livros do ensino médio da Associação Livro Aberto. Os capítulos
presentes no Overleaf utilizam comandos e ambientes próprios para a
sintaxe dos livros do projeto, que serão melhor detalhados aqui para
facilitar o processo colaborativo. Não nos propomos a ensinar
\LaTeX~neste manual, mas recomendamos
\emph{\href{http://alfarrabio.di.uminho.pt/~albie/lshort/pt-lshort-a5.pdf}{Uma
		não tão pequena introdução ao \LaTeXe}} como referência.

A versão do modelo utilizada é apenas um rascunho, para facilitar a
produção de conteúdo, por isto os títulos são simplificados e não
refletem o estado final do material, por isso, não se preocupe com a
formatação, pois isto será corrigido numa etapa de pós-produção.

Utilizamos como base uma classe de documento, acompanhada de um pacote,
chamados \latexinline/livroabertoem.cls/ e
\latexinline/livroabertoem-preamble.sty/, respectivamente, criados
tanto para a versão final quanto para a de rascunho, trazendo opções
para ligar e desligar comandos que aumentam o tempo de processamento do
documento.

A seguir, temos uma seção com comandos e lembretes para serem
consultados pelos leitores que já estejam familiarizados com a classe,
e precisam de uma \enquote{cola} para relembrar comandos. Caso esteja
lendo este documento pela primeira vez, vá para a
\cref{sec:desenvolvimento-projeto}.

