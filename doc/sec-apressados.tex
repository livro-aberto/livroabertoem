\section{\texttt{livroabertoem} para apressados}\label{sec:apressados}

Esta seção traz em um formato simplificado os comandos e suas
descrições diretas, para aqueles que precisam de uma consulta rápida.
Para exemplos concretos, veja as outras seções do manual.

\subsection{Comandos para divisão do documento}\label{sec:comandos-divisao}

As \cref{tab:divisoes-comandos,tab:divisoes-formatacao}, trazem os
tipos de divisão do documento, isto é, a divisão por capítulos, seções,
etc. Todos aqueles comandos que possuem \emph{título} requerem um
argumento.

\begin{table}[H]
	\centering\small
	\begin{tabular}
		{l>{\raggedright}p{.25\linewidth}cp{.40\linewidth}}
		\toprule
		Divisão            & Comando                                                                                      & Hierarquia     & Descrição                                                                                                                                            \\
		\midrule
		Volume             & \latexinline/\volume/                                                                        & \(-1\)         & Agrupamento de capítulos de um mesmo assunto (Geometria, Funcões, \ldots).                                                                           \\
		\addlinespace
		Capítulo           & \latexinline/\chapter/                                                                       & \(0\)          & Conteúdos específicos dentro de um assunto (Áreas, Função Afim, Ladrilhamento, \ldots).                                                              \\
		\addlinespace
		Seção              & \latexinline/\section/                                                                       & \(1\)          & Separa subseções e os \emph{banners}, podendo servir como uma introdução a um explorando.                                                            \\
		\addlinespace
		\emph{Banners}     & \latexinline/\explore/, \latexinline/\arrange/, \latexinline/\practice/, \latexinline/\know/ & \(2\)          & Dividem uma seção entre formatos de conteúdos (Explorando, Organizando, Praticando, Saiba Mais)                                                      \\
		Subseção           & \latexinline/\subsection/                                                                    & \(3\)          & Comando que pode ser utilizado para dividir texto dentro de uma atividade ou dentro de uma das divisões de conteúdo.                                 \\
		\addlinespace
		Exercícios         & \latexinline/\exercise/                                                                      & \(1\) ou \(2\) & É também um \emph{banner}, mas que possui alinhamento diferente dos demais, utilizando a cor principal da coleção (azul escuro), e não possui título \\
		\addlinespace
		Material Adicional & \latexinline/\additionalmaterial/                                                            & \(1\)          & Seção de conteúdos adicionais fora do texto principal (encartes, fichas, referências, \ldots)                                                        \\
		\bottomrule
	\end{tabular}
	\caption{Formatação das divisões.}\label{tab:divisoes-comandos}
\end{table}

\begin{table}[H]
	\centering\small
	\begin{tabular}{l*{2}{p{.4\linewidth}}}
		\toprule
		Divisão               & Função na versão final                                                                                                                                                                                                                                                                                                                                                            & Adaptação na versão de rascunho                                 \\
		\midrule
		Volume                & Cria a capa com ícone e nome do volume                                                                                                                                                                                                                                                                                                                                            & Em vez da capa, há um texto substituto.                         \\
		\addlinespace
		Capítulo              & Cria uma capa com o ícone vazado (a mesma para capítulos de um mesmo volume),
		com o nome do capítulo e o nome d(os) professor(es) colaboradores, cria a
		página de créditos a partir dos comandos usados previamente, junto das
		caixas \enquote{O quê?} e \enquote{Por que?}
		                      & Cria apenas um título simplificado, com o nome do capítulo e o nome do professor,
		com o texto do \enquote{O quê?} e \enquote{Por quê?} na mesma página. Não gera a página de créditos.                                                                                                                                                                                                                                                                                                                                                                        \\
		\addlinespace
		Seção                 & Cria um título numerado em uma página de rosto                                                                                                                                                                                                                                                                                                                                    & Sem modificações                                                \\
		\addlinespace
		\emph{Banners}        & Cria um \emph{Banner} com a cor da respectiva divisão, o nome da divisão e um título, modificando a cor de elementos do conteúdo.                                                                                                                                                                                                                                                 & Gera apenas um título de subseção genérico, em vez de um banner \\
		Subseção              & Título simples com a cor do banner atual (Não gera entrada na Tabela de Conteúdos).                                                                                                                                                                                                                                                                                               & Sem alterações.                                                 \\
		\addlinespace
		Exercícios            & Cria um \emph{banner} apenas com o nome \enquote{Exercícios}, sem título. É o único \emph{banner} que pode estar em dois tipos de hierarquia. Caso utilize o comando normalmente, ele possui a mesma hierarquia dos outros \emph{banners}, mas caso deseje ter exercícios apenas uma seção de exercício no capítulo, pode-se usar \latexinline/\exercise[section]/, que irá gerar
		uma quebra de página. & O mesmo que os outros \emph{banners}                                                                                                                                                                                                                                                                                                                                                                                                                \\
		\addlinespace
		Material Adicional    & Cria um \emph{banner} com o nome \enquote{Material Adicional}, sem título                                                                                                                                                                                                                                                                                                         & O mesmo que os outros \emph{banners}                            \\
		\bottomrule
	\end{tabular}
	\caption{Comandos de divisão do documento}\label{tab:divisoes-formatacao}
\end{table}


{
\renewcommand*{\thefootnote}{\fnsymbol{footnote}}

Em resumo, a hierarquia da tabela de conteúdos é a seguinte:

\dirtree{%
	.1 Volume.
	.2 Capítulo.
	.3 Introdução ao Professor\footnotemark[1].
	.3 Seção.
	.4 Para o professor da seção\footnotemark[1].
	.4 Explorando.
	.4 Organizando.
	.4 Praticando.
	.4 Exercícios da seção.
	.3 Material Suplementar.
	.3 Exercícios do capítulo.
	.3 Notas\footnotemark[1].
	.3 Bibliografia.
}%
\footnotetext[1]{Divisões apenas do material do professor. Ver \cref{subsec:comando-professor}.}
}

\newpage

\subsection{Comandos de ambientes}\label{sec:comandos-ambientes}

Todos os ambientes passam, na versão de rascunho, a usar título de
subseção, com as cores indicadas na \cref{tab:ambientes}. Vale lembrar
que todos os comandos de ambiente precisam estar entre
\latexinline/\begin{comando}/ \latexinline/\end{comando}/. Se um
ambiente tem \emph{título}, ele possui um argumento obrigatório.

\begin{table}[htp!]
	\centering\small
	\begin{tabular}
		{llp{.2\linewidth}p{.4\linewidth}}
		\toprule
		\centering Ambiente & \centering Comando & \centering Descrição                                             & \centering\arraybackslash Função na versão final                                                                                                                                                                                                                                         \\
		\midrule
		Atividade           & \verb|task|        & Auto-explicativo                                                 & Cria um banner na cor da divisão atual (diferente dos de divisão) com o título da atividade à esquerda, e a numeração da atividade à direita. Régua ao final do ambiente, para separar do resto do conteúdo, além de fonte diferenciada, para destacar que o texto pertence à atividade. \\
		\addlinespace
		Exemplo             & \verb|example|     & Auto-explicativo                                                 & Cria uma \verb|tcolorbox|, uma caixa com fundo cinza, com título acima da caixa à esquerda, e o nome \enquote{Exemplo} junto da numeração à direita em uma aba acima da caixa, na cor da numeração atual.                                                                                \\
		\addlinespace
		Observação          & \verb|observation| & Auto-explicativo                                                 & Cria uma \verb|tcolorbox| no mesmo estilo da de \enquote{Exemplo}, porém sem numeração, apenas com o nome \enquote{Observação}.                                                                                                                                                          \\
		\addlinespace
		Teorema             & \verb|theorem|     & Auto-explicativo                                                 & \verb|tcolorbox| no mesmo modelo de \enquote{Observação}                                                                                                                                                                                                                                 \\
		\addlinespace
		Definição           & \verb|definition|  & Auto-explicativo                                                 & \verb|tcolorbox| no mesmo modelo de \enquote{Observação}                                                                                                                                                                                                                                 \\
		\addlinespace
		Para refletir       & \verb|reflection|  & Caixa para reflexões adicionais relacionadas ao conteúdo         & Cria uma \verb|tcolorbox| no mesmo modelo, porém, sem título, a aba à esquerda apenas com o nome \enquote{Para refletir} e ícone de cérebro à esquerda, ao lado da caixa. A cor utilizada e a cor definida para o volume.                                                                \\
		\addlinespace
		Você sabia?         & \verb|knowledge|   & Caixa com curiosidades relacionadas ao conteúdo                  & Cria uma \verb|tcolorbox| no mesmo modelo do \enquote{Para refletir}, porém com ícone de lupa.                                                                                                                                                                                           \\
		\addlinespace
		Para pesquisar      & \verb|research|    & Caixa para dar dicas de pesquisa adicional para os alunos        & Mesmo modelo das duas anteriores, porém com ícone de bloco de notas                                                                                                                                                                                                                      \\
		\addlinespace
		Projeto aplicado    & \verb|project|     & Caixa para criação de um projeto relacionado à seção ou capítulo & Semelhante às três anteriores, porém sem ícone, e com duas abas, uma em cima da outra                                                                                                                                                                                                    \\
		\bottomrule
	\end{tabular}

	\caption{Ambientes definidos pela classe.}\label{tab:ambientes}
\end{table}


\subsection{Comandos do material do professor}\label{subsec:comando-professor}

O material do professor aparece quando se utiliza a opção
\verb|professor|, da classe. O conteúdo do professor se apresenta em
duas formas. A primeira é a inclusão de material do professor em
paralelo ao material do aluno, através dos ambientes \verb|teacher| e
\verb|teacher*|. (A diferença está na sincronização vertical com o
material do professor, e é melhor explicada na
\cref{sec:teacher-estrela}). Há alguns ambientes exclusivos da coluna
lateral, que podem ser vistos na \cref{tab:teacher-ambientes}. Todos
possuem o mesmo estilo, sendo o nome do ambiente e o título separados
por uma régua horizontal, e funcionam da mesma forma na versão de
rascunho.

\begin{table}[htp!]
	\centering\small
	\begin{tabular}{llp{.4\linewidth}}
		\toprule
		Ambiente               & Comando           & Descrição                                     \\
		\midrule
		Objetivoes Específicos & \verb|objectives| & Os objetivos do conteúdo apresentado
		(preferencialmente concisos e em forma de lista)                                           \\
		\addlinespace
		Sugestões e Discussões & \verb|sugestions| & Uma pequena conversa com o professor
		sobre o conteúdo, trazendo possibilidades de como conduzir o conteúdo e especificações
		sobre a execução de atividades. Possui um comando próprio \latexinline/\sugestion/ para
		enumerar sugestões específicas (divisão da turma, duração, possíveis dificuldades, \ldots) \\
		\addlinespace
		Solução                & \verb|answer|     & Auto-explicativo                              \\
		\bottomrule
	\end{tabular}
	\caption{Comandos da coluna lateral do material do professor.}\label{tab:teacher-ambientes}
\end{table}

É válido lembrar que não é necessário utilizar os ambientes na coluna. É possível inserir
textos simples, figuras, tabelas etc.

Se o texto que se deseja colocar é grade demais, você pode inserí-lo em
uma nota, com o comando
\latexinline/\teachernotes[ambiente]{título}{texto}/, isto é, possui
dois argumentos obrigatórios e um opcional. Caso não seja declarado
nenhum dos ambientes da \cref{tab:teacher-ambientes}, então será criado
um título com uma régua horizontal abaixo, se declarado, será criado um
título nos mesmos moldes da seção.

A outra forma de conteúdo do professor e a criação de uma nova página
utilizando a mesma coluna do material do aluno, que chamaremos de
\emph{página do professor}, que possui um rodapé na cor do volume e uma
numeração alternativa (em progresso). São criadas através dos ambientes

\begin{itemize}
	\item \verb|teacherintroduction|: Uma seção introdutória ao professor apresentando o capítulo
	      como um todo. Não possui título;
	\item \verb|teachersection|: Uma seção com título, nos mesmos moldes da seção do aluno,
	      porém não-numerada. Pode ser utilizada como uma forma de introduzir novas seções, junto
	      do comando \latexinline/\section/ do material do aluno;
	\item \verb|longteacher|: Página do professor com título, que tem a função de adicionar
	      textos para o professor que são grandes demais para a coluna lateral. Deve ser evitado,
	      pois cria disrupção em relação ao material do aluno, podendo trazer conflitos na paginação.
	      Para textos maiores, utilize as notas aos fim do capítulo. (\cref{sec:notas-professor});
	\item \verb|ExerciseAnswers|: Ambiente para inserir as respostas das seções de exercício.
\end{itemize}

Há também o ambiente \verb|habilities| para a inserção de texto de
habilidades da BNCC. O funcionamento deste é melhor explicado na
\cref{sec:habilidades}

\paragraph{Observação} O ambiente \verb|teacher| pode ser também usado nas páginas do
professor.

\subsubsection{Notas finais}\label{sec:notas-professor}

Com frequência o conteúdo desejado para o professor é grande demais
para a coluna, mas ao mesmo tempo não é suficiente para preencher uma
página com o ambiente \verb|longteacher|, por isso, usamos o a
interface de \emph{page notes} da classe \verb|memoir|, que uma seção
com as notas criadas no meio do texto (algo como notas de rodapé, mas
no final do texto). Para utilizar a página de notas, basta usar o
comando anterior para criar um \emph{banner} semelhante ao da coluna
lateral.

Para que as notas no final do texto sejam geradas no PDF, é necessário
o comando \latexinline/\printpagenotes/ no final do capítulo (antes da
bibliografia).

Importante ressaltar que as notas funcionam apenas no ambiente
\verb|teacher|, e devem estar em um parágrafo separado.

\subsection{Comandos pré-volume}\label{sec:pre-volume}

Alguns comandos devem ser utilizados antes do comando
\latexinline/\volume/, podendo ser usados no prêambulo também. São
estes

\begin{itemize}
	\item \latexinline/\volumecolor/: Define a cor utilizada para elementos gerais do volume, como as capas, algumas das caixas, título da seção do profesor, etc. Algumas cores já foram definidas para os volumes existentes, chamadas \verb|volume#|, onde \#  representa um número de 1 a 5. (Para mais informações sobre cores, consultar a \cref{tab:cores} ou a \cref{subsec:cores})
	\item \latexinline/\footericon/: Define o ícone utilizado no rodapé das páginas do livro. Não aparece na versão de rascunho.
	\item \latexinline/\volumeillustration/: Define a figura a ser utilizada na capa do volume.
	\item \latexinline/\chapterillustration/: Define a figura que será utilizada na capa dos capítulos do volume.
\end{itemize}

Após o comando \latexinline/\volume/, deve vir a Tabela de Conteúdos,
inserida com \latexinline/\tableofcontents/.

\subsection{Comandos pré-capítulo}\label{sec:pre-capitulo}

Os seguintes comandos devem ser inseridos antes de
\latexinline/\chapter/. No caso de capítulos separados ou no primeiro
capítulo de cada volume, pode ser colocado também no prêambulo do
documento.

\begin{itemize}
	\item \latexinline/\author/: nome dos autores separados por \latexinline/\and/
	\item \latexinline/\revision/: nome dos revisores separados por \latexinline/\and/
	\item \latexinline/\version/: Versão atual do documento
	\item \latexinline/\graphicsauthor/: Autor(es) das figuras técnias, separados por \latexinline/\and/ %chktex 36
	\item \latexinline/\illustrationauthor/: Author(es) das ilustrações do capítulo, separados por \latexinline/\and/. %chktex 36
	\item \latexinline/\collaboration/: Colaborador(es) do capítulo, separados por \latexinline/\and/. %chktex 36
	\item \latexinline/\chapterwhat/: Texto introdutório \enquote{O quê?}.
	\item \latexinline/\chapterbecause/: Texto introdutório \enquote{Por quê?}
\end{itemize}

