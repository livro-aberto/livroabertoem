%\begin{macrocode}
\newlength{\la@bannername@width}
\newlength{\la@bannername@height}
\newlength{\la@bannername@tip}
\newlength{\la@titlebeforeskip}
\newlength{\la@titleafterskip}
%\end{macrocode}

%\begin{macrocode}
\setlength{\la@titlebeforeskip}{1em plus 2pt minus 1pt}
\setlength{\la@titleafterskip}{2em plus 3pt minus 1pt}
%\end{macrocode}

%\begin{macrocode}
\setlength{\la@bannername@width}{130pt}
\setlength{\la@bannername@height}{20pt}
\setlength{\la@bannername@tip}{25pt}
%\end{macrocode}

%\begin{macrocode}
\newcommand{\la@default@banner@length}{
  \setlength{\la@bannername@width}{130pt}
  \setlength{\la@bannername@height}{20pt}
  \setlength{\la@bannername@tip}{25pt}
}
%\end{macrocode}

%\begin{macrocode}
\newcommand{\la@large@banner@length}{
  \setlength{\la@bannername@width}{200pt}
  \setlength{\la@bannername@height}{20pt}
  \setlength{\la@bannername@tip}{25pt}
}
%\end{macrocode}

% Defines the banner used for the sections
%\begin{macrocode}
\newcommand{\la@bannernamestyle}[1][]{
		\coordinate [#1] (tag west) at 
      ($(-1\la@bannername@tip,0pt)$);
		\coordinate (tag east) at 
      ($(tag west) + (\la@bannername@width, 0pt)$);
		\coordinate (tag upper left) at 
      ($(tag west) + (\la@bannername@tip, .5\la@bannername@height)$);
		\coordinate (tag lower left) at 
      ($(tag west) + (\la@bannername@tip, -.5\la@bannername@height)$);
		\coordinate (tag upper right) at 
      ($(tag west) + (\la@bannername@width - \la@bannername@tip, .5\la@bannername@height)$);
		\coordinate (tag lower right) at 
      ($(tag west) + (\la@bannername@width - \la@bannername@tip, -.5\la@bannername@height)$);

		\fill [rounded corners=4, fill = \currentcolor] (tag west) 
                                                 -- (tag upper left) 
                                                 -- (tag upper right) 
                                                 -- (tag east) 
                                                 -- (tag lower right) 
                                                 -- (tag lower left) 
                                                 -- cycle;

		\node (banner name) [font = \vphantom{Ag}, anchor = center] at 
      ($(tag east)!.5!(tag west) - (0,1pt)$) {{\bannernamefont \la@bannername}};
}
%\end{macrocode}

%\begin{macrocode}
\newcommand{\la@bannerstyle}[1]{
  \ifdraft
    {\bannertitlefont\LARGE \la@bannername: #1\par}
  \else
    \tikzset{external/export next=false}
    \begin{tikzpicture}[remember picture, overlay]
      \draw [\currentcolor,  line width = 3pt] (-\la@student@spinemargin,0) 
      -- (\la@student@textwidth-7pt,0);

      \la@bannernamestyle

      \node (banner title) [
          anchor = east, fill, rectangle, white, inner sep = 0pt,
          inner sep = 3pt, font = \vphantom{Ag}
        ]
        at (\la@student@textwidth,0) {{\bannertitlefont #1}};
    \end{tikzpicture}
  \fi
}
%\end{macrocode}


%\begin{macrocode}
\newcommand{\la@exercisebanner}{
  \ifdraft
  {\bannertitlefont\LARGE \la@bannername}
  \else
    \la@default@banner@length
    \tikzset{external/export next=false}
    \begin{tikzpicture}[remember picture, overlay]
      \draw [principal,  line width = 3pt] (-\la@student@spinemargin,0) 
      -- (\la@student@textwidth+\la@student@foremargin,0);

      \la@bannernamestyle[shift = {(.5\la@student@textwidth-.5\la@bannername@width+\la@bannername@tip,0)}]
    \end{tikzpicture}
  \fi
}
%\end{macrocode}

%\begin{macrocode}
\newcommand{\la@largebannerstyle}{
  \ifdraft
  {\bannertitlefont\LARGE \la@bannername}
  \else
    \la@large@banner@length
    \tikzset{external/export next=false}
    \begin{tikzpicture}[remember picture, overlay]
      \draw [principal,  line width = 3pt] (-\la@student@spinemargin,0) 
        -- (\la@student@textwidth+\la@student@foremargin,0);

      \la@bannernamestyle
    \end{tikzpicture}
  \fi
}
%\end{macrocode}


% Formatação dos comandos \section e \subsection
%\begin{macrocode}
\newif\ifaftersection
\setsecheadstyle{\sectiontitlefont\raggedright}
\setsubsecheadstyle{\subsectiontitlefont\raggedright}

\newcommand{\la@presection}{
    \ifodd\c@page
      \ifdim\pagetotal=\z@\else
        \clearpage\leavevmode\thispagestyle{empty}\clearpage
      \fi
    \else
      \clearpage
    \fi
    \ifteacherpnum
      \renewcommand{\thepage}{\arabic{page}}
      \setcounter{page}{\value{stupagecount}}
      \teacherpnumfalse
    \fi
    \ifteacherpage\teacherpagefalse\fi
    \ifteacher\la@teacher@backgroundcolor{boxbackground}\fi
}

\let\la@plainsection\section%
  \pretocmd{\section}{%
    \la@presection
    \thispagestyle{livroaberto}
    \pagestyle{livroaberto}
	\aftersectiontrue
  \color{black}
}{}{}
\setsecindent{0pt}

% \setafterparaskip{.333mu}
\setbeforeparaskip{.75\la@titlebeforeskip}


%\begin{macrocode}
\newcommand{\bannername}[1]{
	\def\la@bannername{#1}
}
\newcommand{\la@bannercounter}[1]{
	\def\la@banner@counter{#1}
}
\newcommand{\bannertype}[1]{
  \def\la@bannertype{#1}
}
%\end{macrocode}

\newcommand{\la@banner}[1]{
    \def\@currentlabelname{#1}
    \addcontentsline{toc}{banner}{\la@bannername: #1}
    \vspace{\la@titlebeforeskip}
    \par\penalty-500\la@bannerstyle{#1}\penalty500\par
    \ifdraft\else
      \vspace*{\la@titleafterskip}
    \fi
    \la@defaultlist
    \color{black}
    \par
}

\newcommand{\la@largebanner}{
  \def\@currentlabelname{\la@bannername}
  \addcontentsline{toc}{section}{\la@bannername}
  \vspace{\la@titlebeforeskip}
  \par\penalty-300\la@largebannerstyle\penalty500\par
  \vspace{\la@titleafterskip}
  \la@defaultlist
  \color{black}
}

\newcommand{\la@exercise}[1][]{
    \def\@currentlabelname{Exercícios}
    \IfStrEq{#1}{section}{
        \clearpage
        \addcontentsline{toc}{section}{Exercícios}
    }{
        \penalty-200
        \addcontentsline{toc}{banner}{Exercícios}
  }
  \vspace{\la@titlebeforeskip}
  \par\la@exercisebanner\par
  \vspace*{\la@titleafterskip}
  \la@exerciselist
  \color{black}
}

%\begin{macrocode}
\newcounter{explore}
\newcommand{\explore}[1]{%
	\sectioncolor{explore}
	\bannername{Explorando}
	\pagestyle{livroaberto}
  \refstepcounter{explore}
  \la@banner{#1}
}
%\end{macrocode}

%\begin{macrocode}
\newcounter{practice}
\newcommand{\practice}[1]{%
	\sectioncolor{practice}
	\bannername{Praticando}
  \refstepcounter{practice}
  \la@banner{#1}
}

\newcounter{know}
\newcommand{\know}[1]{%
	\sectioncolor{know}
	\bannername{Saiba Mais}
  \refstepcounter{know}
  \la@banner{#1}
}

\newcounter{arrange}
\newcommand{\arrange}[1]{%
	\sectioncolor{arrange}
	\bannername{Organizando}
  \refstepcounter{arrange}
  \la@banner{#1}
}
%\end{macrocode}

%\begin{macrocode}
\newcounter{exercise}
\newcommand{\exercise}{	
	\sectioncolor{principal}
	\bannername{\strut{Exercícios}}
  \refstepcounter{exercise}
	\la@exercise
}

\newcounter{additionalmaterial}
\newcommand{\additionalmaterial}{	
	\sectioncolor{principal}
	\bannername{Material Suplementar}
  \refstepcounter{additionalmaterial}
  \la@largebanner
}
%\end{macrocode}


% \newcommand{\expandtocdepth}[1][]{%
% 	\renewcommand{\toclevel@section}{1}
% 	\renewcommand{\toclevel@subsection}{4}
% 	\renewcommand{\toclevel@paragraph}{5}
% 	\renewcommand{\toclevel@subparagraph}{6}

% 	\renewcommand{\toclevel@exploresec}{2}
% 	\renewcommand{\toclevel@practicesec}{2}
% 	\renewcommand{\toclevel@arrangesec}{2}
% 	\setcounter{tocdepth}{2}

% 	\IfStrEqCase{#1}{%
% 		{}{
% }%
% 		{know}{
% 			\renewcommand{\toclevel@knowsec}{2}
% 		}%
% 		{exercise}{
% 			\renewcommand{\toclevel@exercisesec}{2}
% 		}%
% 		{know, exercise}{
% 			\renewcommand{\toclevel@knowsec}{2}
% 			\renewcommand{\toclevel@exercisesec}{2}
% 		}
% 	}[%
% 	Apenas os valores ``know'' e ``exercise'' são permitidos.
% 	]%
% }


