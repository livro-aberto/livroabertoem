\RequirePackage{tikz}
\RequirePackage[inline]{enumitem}

\pgfdeclareshape{enumbox}{%
	\inheritsavedanchors[from=rectangle]
	\inheritanchorborder[from=rectangle]

	\inheritanchor[from=rectangle]{center}
	\inheritanchor[from=rectangle]{south}
	\inheritanchor[from=rectangle]{west}
	\inheritanchor[from=rectangle]{north}
	\inheritanchor[from=rectangle]{east}
	\inheritanchor[from=rectangle]{base}

	\backgroundpath{%
		\southwest\pgf@xa = \pgf@x\pgf@ya = \pgf@y
		\northeast\pgf@xb = \pgf@x\pgf@yb = \pgf@y

		% Construct main path
		\pgfpathmoveto{\pgfpoint{\pgf@xa}{\pgf@ya+2pt}}
		\pgfpathlineto{\pgfpoint{\pgf@xa}{\pgf@yb}}
		\pgfpathlineto{\pgfpoint{\pgf@xb-2pt}{\pgf@yb}}
		\pgfpatharc{90}{0}{2.0pt}
		\pgfpathlineto{\pgfpoint{\pgf@xb}{\pgf@ya}}
		\pgfpathlineto{\pgfpoint{\pgf@xa+2pt}{\pgf@ya}}
		\pgfpatharc{270}{180}{2.0pt}
		\pgfpathclose%
	}
}

\pgfdeclareshape{itembox}{%
	\inheritsavedanchors[from=rectangle]
	\inheritanchorborder[from=rectangle]

	\inheritanchor[from=rectangle]{center}
	\inheritanchor[from=rectangle]{south}
	\inheritanchor[from=rectangle]{west}
	\inheritanchor[from=rectangle]{north}
	\inheritanchor[from=rectangle]{east}
	\inheritanchor[from=rectangle]{base}

	\backgroundpath{%
		\southwest\pgf@xa = \pgf@x\pgf@ya = \pgf@y
		\northeast\pgf@xb = \pgf@x\pgf@yb = \pgf@y

		% Construct main path
		\pgfpathmoveto{\pgfpoint{\pgf@xa}{\pgf@ya}}
		\pgfpathlineto{\pgfpoint{\pgf@xa}{\pgf@yb}}
		\pgfpathlineto{\pgfpoint{\pgf@xb}{\pgf@yb}}
		\pgfpathlineto{\pgfpoint{\pgf@xb}{\pgf@ya}}
		\pgfpathclose%
	}
}

\newcommand{\enumbox}[1]{%
	\tikzset{external/export next = false}
	\tikz[baseline = (numbereditem.base)]{%
		\node[shape = enumbox,text = white, font = \bfseries\fontsize{11pt}{0}\selectfont, draw = black,
			fill = cor1,
			minimum width = 6.00mm, minimum height = 6.00mm] (numbereditem) {#1};
	}
}

\newcommand{\enumdefault}[1]{%
	\tikzset{external/export next = false}
	\tikz[baseline = (numbereditem.base)]{%
		\node[shape = enumbox, font = \bfseries\fontsize{11pt}{0}\selectfont, draw = black, minimum width = 6.00mm, minimum height = 6.00mm] (numbereditem) {#1};
	}
}


% Defining an alternative shape for itemize, where the bullets are actually squares.
\newcommand{\itembox}{%
	\tikzset{external/export next = false}
	\tikz[baseline = (boxitem.base)] {%
		\node[shape = rectangle, draw = black, minimum width = 6.00mm, minimum height = 6.00mm] (boxitem) {};
	}
}

% Defining an alternative shape for itemize, where the bullets are actually colored squares.
\newcommand{\itemcoloredsquare}{%
	\tikzset{external/export next = false}
	\tikz{%
		\node[shape = rectangle, fill = \currentcolor, minimum width = 1.50mm, minimum height = 1.50mm] (boxitem) {};
	}
}

\newcommand{\la@defaultlist}{
	\setlist[enumerate]{%
		label = \protect\titem{\alph*}, topsep = \parskip, itemsep = .75\parskip
	}
	\setlist[enumerate,2]{%
		label = \titem{\roman*}, topsep = \parskip, itemsep = .75\parskip %chktex 9
	}
	\setlist[enumerate,3]{%
		label = \titem{\Alph*)}, topsep = \parskip, itemsep = .75\parskip %chktex 9
	} %chktex 10
	\setlist[itemize]{%
		label = \protect\itemcoloredsquare, topsep = \parskip, itemsep = .75\parskip%
	}
} 

\newcommand{\la@widelist}{
	\setlist[enumerate]{%
		wide, label = \protect\titem{\alph*}, topsep = 0pt, itemsep = 0pt, partopsep = 0pt
	}
	\setlist[enumerate,2]{%
		label = \titem{\roman*}, topsep = 0pt, itemsep = 0pt, left = 1em, partopsep = 0pt %chktex 9
	}
	\setlist[enumerate,3]{%
		label = \titem{\Alph*)}, topsep = 0pt, itemsep = 0pt, left = 1.5em, partopsep = 0pt %chktex 9
	} %chktex 10
	\setlist[itemize]{%
		wide, label = \protect\itemcoloredsquare, topsep = 0pt, itemsep = 0pt, partopsep = 0pt%
	}
} 

\newcommand{\la@exerciselist}{%
	\setlist[enumerate]{%
		label = {\color{blue}\arabic*.}, topsep = \parskip, itemsep = .75\parskip
	}
	\setlist[enumerate,2]{%
		label = \textit{\alph*)}, topsep = \parskip, itemsep = .75\parskip %chktex 9
	}
	\setlist[enumerate,3]{%
		label = \textit{\roman*)}, topsep = \parskip, itemsep = .75\parskip %chktex 9
	} %chktex 10
} %chktex 10

\@ifpackageloaded{tasks}{%
	\settasks{%
		label = \titem{\alph*)}, %chktex 9
		label-width = 14pt
	}
}{} % chktex 10 chktex 17


\newcommand{\titem}[1]{%
	\textcolor{\currentcolor}{%
		\textbf{#1)} % chktex 9
	}
} % chktex 10 chktex 17


