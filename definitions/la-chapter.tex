\RequirePackage{tikz}

\newlength{\la@chapterbannerheight}
\setlength{\la@chapterbannerheight}{140pt}

\newcommand{\chapterillustration}[1]{%
		\def\la@chapterillustration{#1}
}

\newcommand{\la@chapter@illustration}{
	\ifdefined\la@chapterillustration
		\begin{tikzpicture}[remember picture, overlay]
			\node [anchor = east, inner sep = 0pt] at (current page.east) {
					\includegraphics[height = \la@student@pageheight]{\la@chapterillustration}
			};
		\end{tikzpicture}
		\undef\la@chapterillustration
	\else\fi
}

% Creates the box with primary color
\newtcolorbox{la@chapterbox}[1]{
	parbox = false, blanker, enhanced, breakable, 
	before skip = \la@titlebeforeskip, after skip = \la@titleafterskip, 
	left = 3mm, right = 3mm, top = 1em, bottom = 3mm, 
	leftrule = 0pt, rightrule = 0pt, toprule = 2pt, bottomrule = 0pt, 
	colframe = primario, colback = boxbackground, 
	title = #1, fonttitle = \boxnamefont\vphantom{Ag},
	boxed title style = {%
		empty, arc = 0pt, outer arc = 0pt, boxrule = 0pt
	},
	attach boxed title to top left = {},
	underlay boxed title = {%
		 \coordinate (fnw) at (frame.north west);
		 \filldraw [draw = primario, fill = primario]
			($(fnw) + (.2pt,0)$) -- ++ (0,20pt) -- ++ (6pc,0) % Up from frame and left
			.. controls +(1.5pc,-1pt) and ($(fnw) + (10.5pc,-1pt)$) .. % curve going down
		    ($(fnw) + (12pc, -1pt)$) -- cycle;
	}
}

\newcommand{\chapterwhat}[1]{
    \def\la@chapterwhat{
    	\begin{la@chapterbox}{O que?}
    		#1
    	\end{la@chapterbox}
    }
}

\newcommand{\chapterbecause}[1]{
    \def\la@chapterbecause{
    	\begin{la@chapterbox}{Por que?}
    		#1
    	\end{la@chapterbox}
	}
}	


\newcounter{la@chapter}
\providecommand*{\toclevel@la@chapter}{1}
\newcommand{\la@chapterautores}{%	
	\ifdefined\la@autores
		\la@autores
		\undef\la@autores
	\else
		\ifdefined\la@autor
			\la@autor
			\undef\la@autor
		\else\fi
	\fi
}

% \titleclass{\la@chapter}{page}[\chapter]
% \titlespacing{\la@chapter}{0pt}{0pt}{0pt}
% \titleformat{\la@chapter}[hang]{\chaptertitlefont}{\thela@chapter}{0pt}{
% }

\renewcommand{\chapter}[1]{
    \def\@currentchaptertitle{#1}
	\refstepcounter{chapter}
  \addcontentsline{toc}{chapter}{\chaptername\space\thechapter: #1}
	\la@chapter@illustration

	\begin{tikzpicture}[remember picture, overlay]
		\fill [\currentvolcolor] (current page.south west) 
					-- ($(current page.south west) + (0,\la@chapterbannerheight)$) 
					-- ($(current page.south east) + (0,\la@chapterbannerheight)$) 
					-- (current page.south east) -- cycle;

		\node [
			align = left, font = \vphantom{Ag}, anchor = west, inner sep = 0pt
		]	at ($(current page.south west) + (30pt,.5\la@chapterbannerheight)$) {
			\parbox{.425\paperwidth}{
        \raggedright{\chapternamefont\chaptername\ \thechapter} \\ \chaptertitlefont#1}
      };

		\node [
			align = right, font = \vphantom{Ag}, anchor = east, inner sep = 0pt
		]	at ($(current page.south east) + (-30pt,.5\la@chapterbannerheight)$) {
			\parbox{.425\paperwidth}{\raggedleft\chapterauthorfont\la@chapterautores}};
	\end{tikzpicture}
	\clearpage
	\la@teacher@backgroundcolor{boxbackground}
	\ifdefined\la@chapterwhat\la@chapterwhat\else\fi
	\ifdefined\la@chapterbecause\la@chapterbecause\else\fi
	\undef\la@chapterwhat
	\undef\la@chapterbecause
}


\chapterstyle{livroaberto}
\pagestyle{livroaberto}
