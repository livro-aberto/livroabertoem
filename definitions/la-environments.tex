\RequirePackage{mfirstuc} %To capitalize just the inicial letter of each word. There is a command to exclude specific words to captalize such like a, an, the, etc.

\RequirePackage{tcolorbox}
	\tcbuselibrary{theorems, listings, breakable, most, skins}
	\tcbsetforeverylayer{shield externalize}


% Creates the banner of task environment
\newcommand{\la@contentsepbanner}[1]{
	\leavevmode\checkoddpage%
	\needspace{6em}

	\tikzset{external/export next=false}

	\begin{tikzpicture}[remember picture, overlay, shift = {(0,0)}]

		\draw [\currentcolor,  line width = 3pt] (-\la@student@leftmargin,0) -- (\la@student@textwidth+\la@student@rightmargin,0);
		\node (banner name) [
			fill, \currentcolor, rectangle, rounded corners=10, anchor = east, minimum width = 3.5cm, minimum height = 20pt] 
			at (\la@student@textwidth,0) {{\bannernamefont \la@bannername}};
		\node [anchor = west] (banner title) at (0,-1.5em) {{\boxtitlefont #1}};

	\end{tikzpicture}
	\vspace{\titleafterskip}
}



% Creates the box with section colors for examples and observations
\newtcolorbox{la@sectionbox}[2]{
	parbox = false, 
	blanker, enhanced, breakable, 
	before skip = \titlebeforeskip, after skip = \titleafterskip, 
	left = 3mm, right = 3mm, top = 1em, bottom = 3mm, 
	leftrule = 0pt, rightrule = 0pt, toprule = 2pt, bottomrule = 0pt, 
	colframe = \currentcolor, colback = boxbackground, 
	title = #1, fonttitle = \boxtitlefont\vphantom{Ag},
	attach boxed title to top left = {%
		% xshift = 3mm
	},
	boxed title style = {colback = white, opacityframe = 0,},
	underlay boxed title = {%
		\coordinate (fne) at (frame.north east);


		\filldraw [draw = \currentcolor, fill = \currentcolor]
			($(fne) + (-.2pt,0)$) -- ++ (0,20pt) -- ++ (-6pc,0) % Up from frame and left
			.. controls +(-1.5pc,-1pt) and ($(fne) + (-10.5pc,-1pt)$) .. % curve going down
			($(fne) + (-12pc, -1pt)$) -- cycle;
			
		\node [anchor = south east, font = \boxnamefont\vphantom{Ag}] at (frame.north east) {#2};
	}
}


% Creates the box with primary color
\newtcolorbox{la@chapterbox}[1]{
	parbox = false, blanker, enhanced, breakable, 
	before skip = \titlebeforeskip, after skip = \titleafterskip, 
	left = 3mm, right = 3mm, top = 1em, bottom = 3mm, 
	leftrule = 0pt, rightrule = 0pt, toprule = 2pt, bottomrule = 0pt, 
	% text width = .4\la@student@textwidth,
	colframe = primario, colback = boxbackground, 
	title = #1, fonttitle = \boxnamefont\vphantom{Ag},
	boxed title style = {%
		empty, arc = 0pt, outer arc = 0pt, boxrule = 0pt
	},
	attach boxed title to top left = {},
	underlay boxed title = {%
		 \coordinate (fnw) at (frame.north west);
		 \filldraw [draw = primario, fill = primario]
			($(fnw) + (.2pt,0)$) -- ++ (0,20pt) -- ++ (6pc,0) % Up from frame and left
			.. controls +(1.5pc,-1pt) and ($(fnw) + (10.5pc,-1pt)$) .. % curve going down
		    ($(fnw) + (12pc, -1pt)$) -- cycle;
	}
}



% Requires icon on the format iconname-volume#.pdf, where # is the number of the current volume. 
% If there are more volumes, you should specify an icon for each volume color. 
% Included icons are "notas", "vocesabia" and "refletir", for each of the 4 default volume colors.
\newcommand{\boxicon}[1]{
	\def\la@boxicon{#1}
}

\newcommand{\la@box@icon}{
	\ifdefined\la@boxicon
		\includegraphics[width=22pt]{\la@boxicon-\currentvolcolor.pdf}
		\undef\la@boxicon
	\else\fi
}

% Creates the box with current volume color and icon
\newtcolorbox{la@volumebox}[1]{
	parbox = false, blanker, enhanced, breakable, 
	before skip = \titlebeforeskip, after skip = \titleafterskip, 
	left = 3mm, right = 3mm, top = 1em, bottom = 3mm, 
	leftrule = 0pt, rightrule = 0pt, toprule = 2pt, bottomrule = 0pt, 
	colframe = \currentvolcolor, colback = boxbackground, 
	title = #1, fonttitle = \boxnamefont\vphantom{Ag},
	boxed title style = {%
		empty, arc = 0pt, outer arc = 0pt, boxrule = 0pt
	},
	attach boxed title to top left = {},
	underlay boxed title = {%
		 \coordinate (fnw) at (frame.north west);
		 \filldraw [draw = \currentvolcolor, fill = \currentvolcolor]
			($(fnw) + (.2pt,0)$) -- ++ (0,20pt) -- ++ (6pc,0) % Up from frame and left
			.. controls +(1.5pc,-1pt) and ($(fnw) + (10.5pc,-1pt)$) .. % curve going down
		    ($(fnw) + (12pc, -1pt)$) -- cycle;
		 \node [overlay, left of = fnw, node distance = 18pt, yshift = 9pt] {\la@box@icon};
	}
}



% Environments

\newcounter{task}[chapter]
\newcounter{example}[chapter]

\newenvironment{task}[1]{\par
	\refstepcounter{task}
	\bannername{Atividade \thetask}
	\la@contentsepbanner{#1}
	\par
}{
	\vspace{\titlebeforeskip}
}


\newenvironment{example}[1]{
	\refstepcounter{example}
	\begin{la@sectionbox}{#1}{Exemplo \theexample}
}{
	\end{la@sectionbox}
}

\newenvironment{observation}[1]{
	\begin{la@sectionbox}{#1}{Observação}
}{
	\end{la@sectionbox}
}


\newcommand{\la@chapterwhat}[1]{
	\begin{la@chapterbox}{O que?}
		#1
	\end{la@chapterbox}
}	

\newcommand{\la@chapterbecause}[1]{
	\begin{la@chapterbox}{Por que?}
		#1
	\end{la@chapterbox}
}	


\newenvironment{reflection}{
	\boxicon{refletir}
	\begin{la@volumebox}{Para refletir}
}{
	\end{la@volumebox}
}

\newenvironment{knowledge}{
	\boxicon{vcsabia}
	\begin{la@volumebox}{Você Sabia?}
}{
	\end{la@volumebox}
}

\newenvironment{research}{
	\boxicon{notas}
	\begin{la@volumebox}{Notas}
}{
	\end{la@volumebox}
}

	

% Creates the box with current volume color
\newtcolorbox{project}{
	parbox = false, blanker, enhanced, breakable, 
	before skip = \titlebeforeskip, after skip = \titleafterskip, 
	left = 3mm, right = 3mm, top = 1em, bottom = 3mm, 
	leftrule = 0pt, rightrule = 0pt, toprule = 2pt, bottomrule = 0pt, 
	colframe = principal, colback = boxbackground, 
	title = Projeto Aplicado, fonttitle = \boxnamefont\vphantom{Ag},
	boxed title style = {%
		empty, arc = 0pt, outer arc = 0pt, boxrule = 0pt
	},
	attach boxed title to top left = {},
	underlay boxed title = {%
		 \coordinate (fnw) at (frame.north west);
		 \coordinate (fne) at (frame.north east);

		 \filldraw [draw = \currentvolcolor, fill = \currentvolcolor]
			($(fnw) + (.2pt,0)$) -- ++ (0,20pt) -- ++ (12pc,0) % Up from frame and left
			.. controls +(1.5pc,-1pt) and ($(fnw) + (16.5pc,-1pt)$) .. % curve going down
		    ($(fnw) + (18pc, -1pt)$) -- cycle;

		 \filldraw [draw = principal, fill = principal]
			($(fnw) + (.2pt,0)$) -- ++ (0,20pt) -- ++ (8pc,0) % Up from frame and left
			.. controls +(1.5pc,-1pt) and ($(fnw) + (12.5pc,-1pt)$) .. % curve going down
		    ($(fnw) + (14pc, -1pt)$) -- cycle;

		 \draw [line width = 2pt, principal] ($(fnw) + (0,-1pt)$) -- ($(fne) + (0,-1pt)$);
	}
}
