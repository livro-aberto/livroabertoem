\ifdraft\else
	\RequirePackage{tcolorbox}
	\tcbuselibrary{theorems, listings, breakable, most, skins}
	\tcbsetforeverylayer{shield externalize}
\fi


% Creates the banner of task environment
\newcommand{\la@contentsepbanner}[1]{
	% \needspace{6em}
	\ifdraft
  {{\subsectiontitlefont\color{\currentcolor}\la@bannername: #1}}
	\else
		\tikzset{external/export next=false}

		\begin{tikzpicture}[remember picture, overlay, shift = {(0,0)}]

			\draw [\currentcolor,  line width = 3pt]
			(-\la@student@spinemargin,0) -- (\la@student@textwidth+\la@student@foremargin,0);
			\node (banner name) [
					fill, \currentcolor, rectangle, rounded corners=10, anchor = east,
					minimum width = 100pt, minimum height = 20pt
				]
      at (\la@student@textwidth,0) {{\bannernamefont\la@bannername}};
			\node [anchor = west, outer sep = 0pt, inner sep = 0pt, font = \boxtitlefont]
      (banner title) at (0,-1.5em) {#1};

		\end{tikzpicture}
	\vspace{\la@titleafterskip}
	\fi
}

% Requires icon on the format iconname-volume#.pdf, where # is the number of the current volume. 
% If there are more volumes, you should specify an icon for each volume color. 
% Included icons are "notas", "vocesabia" and "refletir", for each of the 4 default volume colors.
\newcommand{\boxicon}[1]{
	\def\la@boxicon{#1}
}

\ifdraft
	\newenvironment{la@sectionbox}[2]{%
		\subsection{{\color{\currentcolor} #2: #1}}
  }{
      \par{\color{cinza}\rule{\linewidth}{1pt}}\normalfont

  }
	\newenvironment{la@volumebox}[1]{%
		\subsection{{\color{\currentvolcolor} #1}}
	}{
	\par{\color{cinza}\rule{\linewidth}{1pt}}\normalfont
}
\else
	% Creates the box with section colors for examples and observations
	\newtcolorbox{la@sectionbox}[2]{
		parbox = false,
		blanker, enhanced, breakable,
		before skip = \la@titlebeforeskip, after skip = \la@titleafterskip,
		left = 3mm, right = 3mm, top = 1em, bottom = 3mm,
		leftrule = 0pt, rightrule = 0pt, toprule = 2pt, bottomrule = 0pt,
		colframe = \currentcolor, colback = boxbackground,
		title = #1, fonttitle = \boxtitlefont\vphantom{Ag},
		attach boxed title to top left = {%
				yshift = 1pt
			},
		boxed title style = {colback = white, opacityframe = 0},
		underlay boxed title = {%
				\coordinate (fne) at (frame.north east);

				\filldraw [draw = \currentcolor, fill = \currentcolor]
				($(fne) + (-.2pt,0)$) -- ++ (0,20pt) -- ++ (-6pc,0) % Up from frame and left
				.. controls +(-1.5pc,-1pt) and ($(fne) + (-10.5pc,-1pt)$) .. % curve going down
				($(fne) + (-12pc, -1pt)$) -- cycle;

				\node [anchor = south east, font = \boxnamefont\vphantom{Ag}] at ($(frame.north east) + (0,-1pt)$) {#2};
			}
	}

\newcommand{\la@box@icon}{%
	\ifdefined\la@boxicon
		\includegraphics[width=22pt]{\la@boxicon-\currentvolcolor.pdf}
		\undef\la@boxicon
	\else\fi
}

	% Creates the box with current volume color and icon
	\newtcolorbox{la@volumebox}[1]{
		parbox = false, blanker, enhanced, breakable,
		before skip = \la@titlebeforeskip, after skip = \la@titleafterskip,
		left = 3mm, right = 3mm, top = 1em, bottom = 3mm,
		leftrule = 0pt, rightrule = 0pt, toprule = 2pt, bottomrule = 0pt,
		colframe = \currentvolcolor, colback = boxbackground,
		title = #1, fonttitle = \boxnamefont\vphantom{Ag},
		boxed title style = {%
				empty, arc = 0pt, outer arc = 0pt, boxrule = 0pt
			},
		attach boxed title to top left = {yshift=-2pt},
		underlay boxed title = {%
				\coordinate (fnw) at (frame.north west);
				\filldraw [draw = \currentvolcolor, fill = \currentvolcolor]
				($(fnw) + (.2pt,0)$) -- ++ (0,20pt) -- ++ (6pc,0) % Up from frame and left
				.. controls +(1.5pc, -1pt) and ($(fnw) + (10.5pc,-1pt)$) .. % curve going down
				($(fnw) + (12pc, -1pt)$) -- cycle;
				\node [overlay, left of = fnw, node distance = 18pt, yshift = 9pt] {\la@box@icon};
			}
	}

\fi


% Environments
\newenvironment{task}[1]{\par
  \filbreak
	\refstepcounter{task}
	\def\@currentlabelname{#1}
	\task@toc{Atividade \thetask: #1}
	\bannername{Atividade \thetask}
	\la@contentsepbanner{#1}
	\par\taskfont
}{
	\par{\color{cinza}\rule{\linewidth}{1pt}}\normalfont
}

\newcounter{example}[chapter]
\newenvironment{example}[1]{
	\refstepcounter{example}\def\@currentlabelname{#1}
	\begin{la@sectionbox}{#1}{Exemplo \theexample}
		}{
	\end{la@sectionbox}
}

\newcounter{observation}[chapter]
\newenvironment{observation}[1]{
	\refstepcounter{observation}\def\@currentlabelname{#1}
	\begin{la@sectionbox}{#1}{Observação}
		}{
	\end{la@sectionbox}
}


\newcounter{reflection}[chapter]
\newenvironment{reflection}{
	\refstepcounter{reflection}
	\boxicon{refletir}
	\begin{la@volumebox}{Para refletir}
		}{
	\end{la@volumebox}
}

\newcounter{knowledge}[chapter]
\newenvironment{knowledge}{
	\refstepcounter{knowledge}
	\boxicon{vcsabia}
	\begin{la@volumebox}{Você Sabia?}
		}{
	\end{la@volumebox}
}

\newcounter{research}[chapter]
\newenvironment{research}{
	\refstepcounter{research}
	\boxicon{notas}
	\begin{la@volumebox}{Para pesquisar}
		}{
	\end{la@volumebox}
}

\ifdraft
	\newenvironment{project}{
		\subsection{Projeto Aplicado} }{}
\else
	% Creates the box with current volume color
	\newtcolorbox{project}{
		parbox = false, blanker, enhanced, breakable,
		before skip = \la@titlebeforeskip, after skip = \la@titleafterskip,
		left = 3mm, right = 3mm, top = 1em, bottom = 3mm,
		leftrule = 0pt, rightrule = 0pt, toprule = 2pt, bottomrule = 0pt,
		colframe = principal, colback = boxbackground,
		title = Projeto Aplicado, fonttitle = \boxnamefont\vphantom{Ag},
		boxed title style = {%
				empty, arc = 0pt, outer arc = 0pt, boxrule = 0pt
			},
		attach boxed title to top left = {},
		underlay boxed title = {%
				\coordinate (fnw) at (frame.north west);
				\coordinate (fne) at (frame.north east);

				\filldraw [draw = \currentvolcolor, fill = \currentvolcolor]
				($(fnw) + (.2pt,0)$) -- ++ (0,20pt) -- ++ (12pc,0) % Up from frame and left
				.. controls +(1.5pc,-1pt) and ($(fnw) + (16.5pc,-1pt)$) .. % curve going down
				($(fnw) + (18pc, -1pt)$) -- cycle;

				\filldraw [draw = principal, fill = principal]
				($(fnw) + (.2pt,0)$) -- ++ (0,20pt) -- ++ (8pc,0) % Up from frame and left
				.. controls +(1.5pc,-1pt) and ($(fnw) + (12.5pc,-1pt)$) .. % curve going down
				($(fnw) + (14pc, -1pt)$) -- cycle;

				\draw [line width = 2pt, principal] ($(fnw) + (0,-1pt)$) -- ($(fne) + (0,-1pt)$);
			}
	}
\fi
