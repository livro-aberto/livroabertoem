\newif\ifteacherenv\teacherenvfalse

\newcommand{\la@teachertcolorbox}{
  \iftcolorbox%
    \ClassError{livroabertoem}{%
    Ambiente do professor dentro de caixa}{%
      O ambiente teacher não pode ficar dentro das caixas criadas\MessageBreak
      usando tcolorbox. Coloque-o imediatamente antes da caixa\MessageBreak
      que se deseja referenciar.
    }
  \fi
}

\newenvironment{la@teacher}{%
  \la@teachertcolorbox
	\teacherenvtrue
	\raggedright
	\la@widelist
	\small
}{
	\par\hfill
	\la@defaultlist
	\teacherenvfalse
	\normalsize
}

\newenvironment{teacher}{
	\par\switchcolumn
	\begin{la@teacher}
		}{
	\end{la@teacher}
	\par\switchcolumn
}

\newenvironment{teacher*}{
	\par\switchcolumn*
	\begin{la@teacher}
		}{
	\end{la@teacher}
	\par\switchcolumn
}

\newcommand{\teachertitle}[3][\currentcolor]{
	\vbox{
    \color{#1}\bfseries\vspace{1em}
    {\Large\selectfont #2}\par
    \vspace{-.5\baselineskip}
    \rule{\la@teacher@textwidth}{3pt}\par
    \vspace{-.3\baselineskip}
    {\large\RaggedRight #3}\par
  }
}

\newcommand{\la@notteacherenv}{\ClassError{livroabertoem}{Fora do ambiente do professor}{%
  Você deve usar este comando somente no ambiente do professor.}
}

\newcommand{\la@teacher@sidetitle}[2]{%
    \vspace{.3\la@titlebeforeskip}
	\ifteacherenv\penalty-5000\par
    \teachertitle{#1}{#2}
    \penalty10000\par
	\else
		\la@notteacherenv
	\fi
}

\newenvironment{objectives}[1]{
	\la@teacher@sidetitle{Objetivos Específicos}{#1}
}{}

\newenvironment{sugestions}[1]{%
	\la@teacher@sidetitle{Sugestões e discussões}{#1}
}{}
\newcommand{\sugestion}[1]{%
	\paragraph{#1}
}

\newenvironment{answer}[1]{%
	\la@teacher@sidetitle{Solução}{#1}
}{}
