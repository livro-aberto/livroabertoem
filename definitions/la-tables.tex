\newcolumntype{C}[1]{>{\centering\arraybackslash}m{#1}}
\newcolumntype{e}{>{\(}c<{\)}}
\newcolumntype{E}[1]{>{\(}C{#1}<{\)}}

\@ifpackageloaded{supertabular}{%
  \newcommand{\tbltobecontinued}[1]{%
    \multicolumn{#1}{r}{\footnotesize Continua na próxima página}\\
  }
  \newcommand{\tblcontinuation}[1]{
    \multicolumn{#1}{r}{\footnotesize Continuação da página anterior}\\
  }

  \newcommand{\supertabularheading}[2]{
    \tablefirsthead{
      \toprule
      #2
      \midrule
    }
    \tablehead{
      \tblcontinuation{#1}
      \toprule
      #2
      \midrule
     }
    \tabletail{
      \bottomrule
      \tbltobecontinued{#1}
     }
    \tablelasttail{
      \bottomrule
    }
  }
}{}

\@ifclassloaded{memoir}{
  \newcommand{\listquadroname}{Lista de quadros}
  \newcommand{\quadroname}{Quadro}
  \newcommand{\quadrorefname}{Quadro}
  \newcommand{\chartautorefname}{Quadro}
  \newlistof{listofquadros}{loq}{\listquadroname}

  % código retirado de https://tex.stackexchange.com/questions/572654/how-might-the-longtable-environment-have-to-be-modified-to-create-a-longquadro-e
  \newcounter{quadro}
  \newfloat{quadro}{loq}{Quadro}
  \newlistentry{quadro}{loq}{0}
}{}
