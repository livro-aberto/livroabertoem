\newif\ifhabilities\habilitiesfalse
\newif\ifhability\habilityfalse
\newif\ifefhability\efhabilityfalse

\NewDocumentEnvironment{habilities}{ O{EM} O{Habilidades da BNCC} }{%
	\habilitiestrue
	\IfStrEq{#1}{EF}{\efhabilitytrue}{}
	\ifteacherpage
      \subsection{#2}
	\else
		\la@notteacherpage
	\fi
}{%
	\habilitiesfalse
	\ifhability
}

\newcommand{\hability}[2][]{%
	\habilitytrue
	\ifhabilities
		\ifefhability
			\paragraph{EF#1MA#2}\la@efhability{#1}{#2}
		\else
			\paragraph{EM13MAT#2}\la@hability{#2}
		\fi
	\else
		\la@nothabilities
	\fi
}

\newcommand{\la@nothabilities}{%
	\ClassError{livroaberto}{Fora do ambiente de habilidades}{%
		Você só deve utilizar este comando dentro de um ambiente de \MessageBreak
		habilidades.
	}
}

\newcommand{\la@invalidhability}{%
	\ClassError{livroaberto}{Habilidade inválida para o Ensino Médio}{%
		Esta habilidade não existe para o Ensino Médio. Você deve colocar\MessageBreak
		os últimos três números da habilidade no hambiente obrigatório.
	}
}

\newcommand{\la@invalidefhability}{%
	\ClassError{livroaberto}{Habilidade Inválida para o Ensino Fundamental}{%
		Esta habilidade não existe para o Ensino Fundamental. Você deve colocar\MessageBreak
		o ano da habilidade ('01', por exemplo) como argumento opcional e os\MessageBreak
		dois números finais da habilidade no argumento obrigatório.
	}
}

\newcommand{\la@nohability}{%
	\ClassError{livroaberto}{Habildades não inseridas}{%
		Você deve colocar habilidades usando o comando \protect\hability.
	}
}

\newcommand{\la@hability}[1]{
	\IfStrEqCase{#1}{
		{101}{%
				Interpretar criticamente situações econômicas, sociais e fatos relativos às
				Ciências da Natureza que envolvam a variação de grandezas, pela análise dos gráficos
				das funções representadas e das taxas de variação, com ou sem apoio de tecnologias
				digitais.
			}{102}{%
				Analisar tabelas, gráficos e amostras de pesquisas estatísticas
				apresentadas em relatórios divulgados por diferentes meios de
				comunicação,
				identificando, quando for o caso, inadequações que possam induzir a erros de
				interpretação, como escalas e amostras não apropriadas.
			}{103}{%
				Interpretar e compreender textos científicos ou divulgados pelas
				mídias,
				que empregam unidades de medida de diferentes grandezas e as conversões possíveis
				entre elas, adotadas ou não pelo Sistema Internacional (SI), como as de armazenamento
				e velocidade de transferência de dados, ligadas aos avanços tecnológicos.
			}{104}{%
				Interpretar taxas e índices de natureza socioeconômica (índice de
				desenvolvimento humano, taxas de inflação, entre outros), investigando os processos de
				cálculo desses números, para analisar criticamente a realidade e produzir argumentos.
			}{105}{%
				Utilizar as noções de transformações isométricas (translação, reflexão,
				rotação e composições destas) e transformações homotéticas para construir figuras e
				analisar elementos da natureza e diferentes produções humanas
				(fractais, construções
				civis, obras de arte, entre outras).
			}{106}{%
				Identificar situações da vida cotidiana nas quais seja necessário fazer
				escolhas levando-se em conta os riscos probabilísticos (usar este ou aquele método
				contraceptivo, optar por um tratamento médico em detrimento de outro etc.)
			}{201}{%
				Propor ou participar de ações adequadas às demandas da região,
				preferencialmente para sua comunidade, envolvendo medições e cálculos de perímetro,
				de área, de volume, de capacidade ou de massa.
			}{202}{%
				Planejar e executar pesquisa amostral sobre questões relevantes, usando
				dados coletados diretamente ou em diferentes fontes, e comunicar os resultados por
				meio de relatório contendo gráficos e interpretação das medidas de tendência central
				e das medidas de dispersão (amplitude e desvio padrão), utilizando ou não recursos
				tecnológicos.
			}{203}{%
				Aplicar conceitos matemáticos no planejamento, na execução e na
				análise de ações envolvendo a utilização de aplicativos e a criação de planilhas
				(para o
				controle de orçamento familiar, simuladores de cálculos de juros simples e compostos,
				entre outros), para tomar decisões.
			}{301}{%
				Resolver e elaborar problemas do cotidiano, da Matemática e de outras
				áreas do conhecimento, que envolvem equações lineares simultâneas, usando técnicas
				algébricas e gráficas, com ou sem apoio de tecnologias digitais.
			}{302}{%
				Construir modelos empregando as funções polinomiais de 1º ou 2º
				graus,
				para resolver problemas em contextos diversos, com ou sem apoio de tecnologias digitais.
			}{303}{%
				Interpretar e comparar situações que envolvam juros simples com as que
				envolvem juros compostos, por meio de representações gráficas ou análise de planilhas,
				destacando o crescimento linear ou exponencial de cada caso.
			}{304}{%
				Resolver e elaborar problemas com funções exponenciais nos quais seja
				necessário compreender e interpretar a variação das grandezas
				envolvidas, em contextos
				como o da Matemática Financeira, entre outros.
			}{305}{%
				Resolver e elaborar problemas com funções logarítmicas nos quais seja
				necessário compreender e interpretar a variação das grandezas
				envolvidas, em contextos
				como os de abalos sísmicos, pH, radioatividade, Matemática Financeira, entre outros.
			}{306}{%
				Resolver e elaborar problemas em contextos que envolvem fenômenos
				periódicos reais (ondas sonoras, fases da lua, movimentos cíclicos, entre outros) e
				comparar suas representações com as funções seno e cosseno, no plano cartesiano, com
				ou sem apoio de aplicativos de álgebra e geometria.
			}{307}{%
				Empregar diferentes métodos para a obtenção da medida da área de
				uma superfície (reconfigurações, aproximação por cortes etc.) e deduzir expressões de
				cálculo para aplicá-las em situações reais (como o remanejamento e a distribuição de
				plantações, entre outros), com ou sem apoio de tecnologias digitais.
			}{308}{%
				Aplicar as relações métricas, incluindo as leis do seno e do cosseno ou as
				noções de congruência e semelhança, para resolver e elaborar problemas que envolvem
				triângulos, em variados contextos.
			}{309}{%
				Resolver e elaborar problemas que envolvem o cálculo de áreas totais e
				de volumes de prismas, pirâmides e corpos redondos em situações reais (como o cálculo
				do gasto de material para revestimento ou pinturas de objetos cujos formatos sejam
				composições dos sólidos estudados), com ou sem apoio de tecnologias digitais.
			}{310}{%
				Resolver e elaborar problemas de contagem envolvendo agrupamentos
				ordenáveis ou não de elementos, por meio dos princípios multiplicativo e aditivo,
				recorrendo a estratégias diversas, como o diagrama de árvore.
			}{311}{%
				Identificar e descrever o espaço amostral de eventos aleatórios, realizando
				contagem das possibilidades, para resolver e elaborar problemas que envolvem o cálculo
				da probabilidade.
			}{312}{%
				Resolver e elaborar problemas que envolvem o cálculo de probabilidade
				de eventos em experimentos aleatórios sucessivos.
			}{313}{%
				Utilizar, quando necessário, a notação científica para expressar uma
				medida, compreendendo as noções de algarismos significativos e algarismos
				duvidosos,
				e reconhecendo que toda medida é inevitavelmente acompanhada de erro.
			}{314}{%
				Resolver e elaborar problemas que envolvem grandezas determinadas
				pela razão ou pelo produto de outras (velocidade, densidade demográfica, energia
				elétrica etc.).
			}{315}{%
				Investigar e registrar, por meio de um fluxograma, quando possível, um
				algoritmo que resolve um problema.
			}
			{316}{%
				Resolver e elaborar problemas, em diferentes contextos, que envolvem
				cálculo e interpretação das medidas de tendência central (média, moda, mediana) e das
				medidas de dispersão (amplitude, variância e desvio padrão).
			}
			{401}{%
				Converter representações algébricas de funções polinomiais de 1º grau
				em representações geométricas no plano cartesiano, distinguindo os casos nos quais o
				comportamento é proporcional, recorrendo ou não a softwares ou aplicativos de álgebra
				e geometria dinâmica.
			}
			{402}{%
				Converter representações algébricas de funções polinomiais de 2º grau
				em representações geométricas no plano cartesiano, distinguindo os casos nos quais
				uma variável for diretamente proporcional ao quadrado da outra, recorrendo ou não a
				softwares ou aplicativos de álgebra e geometria dinâmica, entre outros materiais.
			}
			{403}{%
				Analisar e estabelecer relações, com ou sem apoio de tecnologias
				digitais, entre as representações de funções exponencial e logarítmica expressas em
				tabelas e em plano cartesiano, para identificar as características fundamentais (domínio,
				imagem, crescimento) de cada função.
			}
			{404}{%
				Analisar funções definidas por uma ou mais sentenças (tabela do Imposto
				de Renda, contas de luz, água, gás etc.), em suas representações algébrica e gráfica,
				identificando domínios de validade, imagem, crescimento e decrescimento, e convertendo
				essas representações de uma para outra, com ou sem apoio de tecnologias digitais.
			}
			{405}{%
				Utilizar conceitos iniciais de uma linguagem de programação na
				implementação de algoritmos escritos em linguagem corrente e/ou matemática.
			}
			{406}{%
				Construir e interpretar tabelas e gráficos de frequências com base em
				dados obtidos em pesquisas por amostras estatísticas, incluindo ou não o uso de softwares
				que inter-relacionem estatística, geometria e álgebra.
			}
			{407}{%
				Interpretar e comparar conjuntos de dados estatísticos por meio de
				diferentes diagramas e gráficos (histograma, de caixa (box-plot), de ramos e folhas, entre
				outros), reconhecendo os mais eficientes para sua análise.
			}
			{501}{%
				Investigar relações entre números expressos em tabelas para
				representá-los no plano cartesiano, identificando padrões e criando conjecturas para
				generalizar e expressar algebricamente essa generalização, reconhecendo quando
				essa representação é de função polinomial de 1º grau.
			}
			{502}{%
				Investigar relações entre números expressos em tabelas para
				representá-los no plano cartesiano, identificando padrões e criando conjecturas para
				generalizar e expressar algebricamente essa generalização, reconhecendo quando
				essa representação é de função polinomial de 2º grau do tipo y = ax2.
			}
			{503}{%
				Investigar pontos de máximo ou de mínimo de funções quadráticas em
				contextos envolvendo superfícies, Matemática Financeira ou Cinemática, entre outros,
				com apoio de tecnologias digitais.
			}
			{504}{%
				Investigar processos de obtenção da medida do volume de prismas,
				pirâmides, cilindros e cones, incluindo o princípio de Cavalieri, para a obtenção das
				fórmulas de cálculo da medida do volume dessas figuras.
			}
			{505}{%
				Resolver problemas sobre ladrilhamento do plano, com ou sem apoio de
				aplicativos de geometria dinâmica, para conjecturar a respeito dos tipos ou composição
				de polígonos que podem ser utilizados em ladrilhamento, generalizando padrões
				observados.
			}
			{506}{%
				Representar graficamente a variação da área e do perímetro de
				um polígono regular quando os comprimentos de seus lados variam, analisando e
				classificando as funções envolvidas.
			}
			{507}{%
				Identificar e associar progressões aritméticas (PA) a funções afins de
				domínios discretos, para análise de propriedades, dedução de algumas fórmulas e
				resolução de problemas.
			}
			{508}{%
				Identificar e associar progressões geométricas (PG) a funções
				exponenciais de domínios discretos, para análise de propriedades, dedução de algumas
				fórmulas e resolução de problemas.
			}
			{509}{%
				Investigar a deformação de ângulos e áreas provocada pelas diferentes
				projeções usadas em cartografia (como a cilíndrica e a cônica), com ou sem suporte de
				tecnologia digital.
			}
			{510}{%
				Investigar conjuntos de dados relativos ao comportamento de duas
				variáveis numéricas, usando ou não tecnologias da informação, e, quando apropriado,
				levar em conta a variação e utilizar uma reta para descrever a relação observada.
			}
			{511}{%
				Reconhecer a existência de diferentes tipos de espaços amostrais,
				discretos ou não, e de eventos, equiprováveis ou não, e investigar implicações no cálculo
				de probabilidades.
			}
	}[\la@invalidhability]
}

\newcommand{\la@efhability}[2]{%
	\IfStrEqCase{#1}{%
		{01}{%
				\IfStrEqCase{#2}{%
					{01}{%
							Utilizar números naturais como indicador de quantidade ou de ordem em diferentes situações cotidianas e reconhecer situações em que os números não indicam contagem nem ordem, mas sim código de identificação
						}
						{02}{%
							Contar de maneira exata ou aproximada, utilizando diferentes estratégias como o pareamento e outros agrupamentos.
						}
						{03}{%
							Estimar e comparar quantidades de objetos de dois conjuntos (em torno de \num{20} elementos), por estimativa e/ou por correspondência (um a um, dois a dois) para indicar “tem mais”, “tem menos” ou “tem a mesma quantidade”.
						}
						{04}{%
							Contar a quantidade de objetos de coleções até \num{100} unidades e apresentar o resultado por registros verbais e simbólicos, em situações de seu interesse, como jogos, brincadeiras, materiais da sala de aula, entre outros.
						}
						{05}{%
							Comparar números naturais de até duas ordens em situações cotidianas, com e sem suporte da reta numérica.
						}
						{06}{%
							Construir fatos básicos da adição e utilizá-los em procedimentos de cálculo para resolver problemas
						}
						{07}{%
							Compor e decompor número de até duas ordens, por meio de diferentes adições, com o suporte de material manipulável, contribuindo para a compreensão de características do sistema de numeração decimal e o desenvolvimento de estratégias de cálculo.
						}
						{08}{%
							Resolver e elaborar problemas de adição e de subtração, envolvendo números de até dois algarismos, com os significados de juntar, acrescentar, separar e retirar, com o suporte de imagens e/ou material manipulável, utilizando estratégias e formas de registro pessoais.
						}
						{09}{%
							Organizar e ordenar objetos familiares ou representações por figuras, por meio de atributos, tais como cor, forma e medida.
						}
						{10}{%
							Descrever, após o reconhecimento e a explicitação de um padrão (ou regularidade), os elementos ausentes em sequências recursivas de números naturais, objetos ou figuras.
						}
						{11}{%
							Descrever a localização de pessoas e de objetos no espaço em relação à sua própria posição, utilizando termos como à direita, à esquerda, em frente, atrás.
						}
						{12}{%
							Descrever a localização de pessoas e de objetos no espaço segundo um dado ponto de referência, compreendendo que, para a utilização de termos que se referem à posição, como direita, esquerda, em cima, em baixo, é necessário explicitar-se o referencial.
						}
						{13}{%
							Relacionar figuras geométricas espaciais (cones, cilindros, esferas e blocos retangulares) a objetos familiares do mundo físico.
						}
						{14}{%
							Identificar e nomear figuras planas (círculo, quadrado, retângulo e triângulo) em desenhos apresentados em diferentes disposições ou em contornos de faces de sólidos geométricos.
						}
						{15}{%
							Comparar comprimentos, capacidades ou massas, utilizando termos como mais alto, mais baixo, mais comprido, mais curto, mais grosso, mais fino, mais largo, mais pesado,
						}
						{16}{%
							Relatar em linguagem verbal ou não verbal sequência de acontecimentos relativos a um dia, utilizando, quando possível, os horários dos eventos.
						}
						{17}{%
							Reconhecer e relacionar períodos do dia, dias da semana e meses do ano, utilizando calendário, quando necessário.
						}
						{18}{%
							Produzir a escrita de uma data, apresentando o dia, o mês e o ano, e indicar o dia da semana de uma data, consultando calendários.
						}
						{19}{%
							Reconhecer e relacionar valores de moedas e cédulas do sistema monetário brasileiro para resolver situações simples do cotidiano do estudante.
						}
						{20}{%
							Classificar eventos envolvendo o acaso, tais como “acontecerá com certeza”, “talvez aconteça” e “é impossível acontecer”, em situações do cotidiano.
						}
						{21}{%
							Ler dados expressos em tabelas e em gráficos de colunas simples.
						}
						{22}{%
							Realizar pesquisa, envolvendo até duas variáveis categóricas de seu interesse e universo de até 30 elementos, e organizar dados por meio de representações pessoais.
						}
				}
			}
			{02}{%
				\IfStrEqCase{#2}{01}{%
					Comparar e ordenar números naturais (até a ordem de centenas) pela
					compreensão de características do sistema de numeração decimal (valor posicional e função
					do zero).
				}
				{02}{%
					Fazer estimativas por meio de estratégias diversas a respeito da quantidade de
					objetos de coleções e registrar o resultado da contagem desses objetos (até 1000 unidades).
				}
				{03}{%
					Comparar quantidades de objetos de dois conjuntos, por estimativa e/ou por
					correspondência (um a um, dois a dois, entre outros), para indicar “tem mais”, “tem menos” ou
					“tem a mesma quantidade”, indicando, quando for o caso, quantos a mais e quantos a menos.
				}
				{04}{%
					Compor e decompor números naturais de até três ordens, com suporte de
					material manipulável, por meio de diferentes adições.
				}
				{05}{%
					Construir fatos básicos da adição e subtração e utilizá-los no cálculo mental ou
					escrito.
				}
				{06}{%
					Resolver e elaborar problemas de adição e de subtração, envolvendo números
					de até três ordens, com os significados de juntar, acrescentar, separar, retirar, utilizando
					estratégias pessoais.
				}
				{07}{%
					Resolver e elaborar problemas de multiplicação (por 2, 3, 4 e 5) com a ideia de
					adição de parcelas iguais por meio de estratégias e formas de registro pessoais, utilizando ou
					não suporte de imagens e/ou material manipulável.
				}
				{08}{%
					Resolver e elaborar problemas envolvendo dobro, metade, triplo e terça parte,
					com o suporte de imagens ou material manipulável, utilizando estratégias pessoais.
				}
				{09}{%
					Construir sequências de números naturais em ordem crescente ou decrescente a
					partir de um número qualquer, utilizando uma regularidade estabelecida.
				}
				{10}{%
					Descrever um padrão (ou regularidade) de sequências repetitivas e de sequências
					recursivas, por meio de palavras, símbolos ou desenhos.
				}
				{11}{%
					Descrever os elementos ausentes em sequências repetitivas e em sequências
					recursivas de números naturais, objetos ou figuras.
				}
				{12}{%
					Identificar e registrar, em linguagem verbal ou não verbal, a localização e os
					deslocamentos de pessoas e de objetos no espaço, considerando mais de um ponto de
					referência, e indicar as mudanças de direção e de sentido.
				}
				{13}{%
					Esboçar roteiros a ser seguidos ou plantas de ambientes familiares, assinalando
					entradas, saídas e alguns pontos de referência.
				}
				{14}{%
					Reconhecer, nomear e comparar figuras geométricas espaciais (cubo, bloco
					retangular, pirâmide, cone, cilindro e esfera), relacionando-as com objetos do mundo físico.
				}
				{15}{%
					Reconhecer, comparar e nomear figuras planas (círculo, quadrado, retângulo
					e triângulo), por meio de características comuns, em desenhos apresentados em diferentes
					disposições ou em sólidos geométricos.
				}
				{16}{%
					Estimar, medir e comparar comprimentos de lados de salas (incluindo contorno)
					e de polígonos, utilizando unidades de medida não padronizadas e padronizadas (metro,
					centímetro e milímetro) e instrumentos adequados.
				}
				{17}{%
					Estimar, medir e comparar capacidade e massa, utilizando estratégias pessoais e
					unidades de medida não padronizadas ou padronizadas (litro, mililitro, grama e quilograma).
				}
				{18}{%
					Indicar a duração de intervalos de tempo entre duas datas, como dias da semana
					e meses do ano, utilizando calendário, para planejamentos e organização de agenda.
				}
				{19}{%
					Medir a duração de um intervalo de tempo por meio de relógio digital e registrar
					o horário do início e do fim do intervalo.
				}
				{20}{%
					Estabelecer a equivalência de valores entre moedas e cédulas do sistema
					monetário brasileiro para resolver situações cotidianas.
				}
				{21}{%
					Classificar resultados de eventos cotidianos aleatórios como “pouco prováveis”,
					“muito prováveis”, “improváveis” e “impossíveis”.
				}
				{22}{%
					Comparar informações de pesquisas apresentadas por meio de tabelas de dupla
					entrada e em gráficos de colunas simples ou barras, para melhor compreender aspectos da
					realidade próxima.
				}
				{23}{%
					Realizar pesquisa em universo de até 30 elementos, escolhendo até três variáveis
					categóricas de seu interesse, organizando os dados coletados em listas, tabelas e gráficos de
					colunas simples.
				}
			}
			{03}{%
				\IfStrEqCase{#2}{%
					{01}{
							Ler, escrever e comparar números naturais de até a ordem de unidade de milhar,
							estabelecendo relações entre os registros numéricos e em língua materna.
						}
						{02}{%
							Identificar características do sistema de numeração decimal, utilizando a
							composição e a decomposição de número natural de até quatro ordens.
						}
						{03}{%
							Construir e utilizar fatos básicos da adição e da multiplicação para o cálculo
							mental ou escrito.
						}
						{04}{%
							Estabelecer a relação entre números naturais e pontos da reta numérica para
							utilizá-la na ordenação dos números naturais e também na construção de fatos da adição e da
							subtração, relacionando-os com deslocamentos para a direita ou para a esquerda.
						}
						{05}{%
							Utilizar diferentes procedimentos de cálculo mental e escrito, inclusive os
							convencionais, para resolver problemas significativos envolvendo adição e subtração com
							números naturais.
						}
						{06}{%
							Resolver e elaborar problemas de adição e subtração com os significados de
							juntar, acrescentar, separar, retirar, comparar e completar quantidades, utilizando diferentes
							estratégias de cálculo exato ou aproximado, incluindo cálculo mental.
						}
						{07}{%
							Resolver e elaborar problemas de multiplicação (por 2, 3, 4, 5 e 10) com os
							significados de adição de parcelas iguais e elementos apresentados em disposição retangular,
							utilizando diferentes estratégias de cálculo e registros.
						}
						{08}{%
							Resolver e elaborar problemas de divisão de um número natural por outro (até
							10), com resto zero e com resto diferente de zero, com os significados de repartição equitativa
							e de medida, por meio de estratégias e registros pessoais.
						}
						{09}{%
							Associar o quociente de uma divisão com resto zero de um número natural por 2,
							3, 4, 5 e 10 às ideias de metade, terça, quarta, quinta e décima partes.
						}
						{10}{%
							Identificar regularidades em sequências ordenadas de números naturais,
							resultantes da realização de adições ou subtrações sucessivas, por um mesmo número,
							descrever uma regra de formação da sequência e determinar elementos faltantes ou seguintes.
						}
						{11}{%
							Compreender a ideia de igualdade para escrever diferentes sentenças de adições
							ou de subtrações de dois números naturais que resultem na mesma soma ou diferença.
						}
						{12}{%
							Descrever e representar, por meio de esboços de trajetos ou utilizando croquis
							e maquetes, a movimentação de pessoas ou de objetos no espaço, incluindo mudanças de
							direção e sentido, com base em diferentes pontos de referência.
						}
						{13}{%
							Associar figuras geométricas espaciais (cubo, bloco retangular, pirâmide, cone,
							cilindro e esfera) a objetos do mundo físico e nomear essas figuras.
						}
						{14}{%
							Descrever características de algumas figuras geométricas espaciais (prismas
							retos, pirâmides, cilindros, cones), relacionando-as com suas planificações.
						}
						{15}{%
							Classificar e comparar figuras planas (triângulo, quadrado, retângulo, trapézio
							e paralelogramo) em relação a seus lados (quantidade, posições relativas e comprimento) e
							vértices.
						}
						{16}{%
							Reconhecer figuras congruentes, usando sobreposição e desenhos em malhas
							quadriculadas ou triangulares, incluindo o uso de tecnologias digitais.
						}
						{17}{%
							Reconhecer que o resultado de uma medida depende da unidade de medida
							utilizada.
						}
						{18}{%
							Escolher a unidade de medida e o instrumento mais apropriado para medições de
							comprimento, tempo e capacidade.
						}
						{19}{%
							Estimar, medir e comparar comprimentos, utilizando unidades de medida
							não padronizadas e padronizadas mais usuais (metro, centímetro e milímetro) e diversos
							instrumentos de medida.
						}
						{20}{%
							Estimar e medir capacidade e massa, utilizando unidades de medida não
							padronizadas e padronizadas mais usuais (litro, mililitro, quilograma, grama e miligrama),
							reconhecendo-as em leitura de rótulos e embalagens, entre outros.
						}
						{21}{%
							Comparar, visualmente ou por superposição, áreas de faces de objetos, de figuras
							planas ou de desenhos.
						}
						{22}{%
							Ler e registrar medidas e intervalos de tempo, utilizando relógios (analógico e
							digital) para informar os horários de início e término de realização de uma atividade e sua
							duração.
						}
						{23}{%
							Ler horas em relógios digitais e em relógios analógicos e reconhecer a relação
							entre hora e minutos e entre minuto e segundos.
						}
						{24}{%
							Resolver e elaborar problemas que envolvam a comparação e a equivalência de
							valores monetários do sistema brasileiro em situações de compra, venda e troca.
						}
						{25}{%
							Identificar, em eventos familiares aleatórios, todos os resultados possíveis,
							estimando os que têm maiores ou menores chances de ocorrência.
						}
						{26}{%
							Resolver problemas cujos dados estão apresentados em tabelas de dupla
							entrada, gráficos de barras ou de colunas.
						}
						{27}{%
							Ler, interpretar e comparar dados apresentados em tabelas de dupla entrada,
							gráficos de barras ou de colunas, envolvendo resultados de pesquisas significativas, utilizando
							termos como maior e menor frequência, apropriando-se desse tipo de linguagem para
							compreender aspectos da realidade sociocultural significativos.
						}
						{28}{%
							Realizar pesquisa envolvendo variáveis categóricas em um universo de até 50
							elementos, organizar os dados coletados utilizando listas, tabelas simples ou de dupla entrada
							e representá-los em gráficos de colunas simples, com e sem uso de tecnologias digitais.
						}
				}
			}
			{04}{%
				\IfStrEqCase{#2}{%
					{01}{%
							Ler, escrever e ordenar números naturais até a ordem de dezenas de milhar.
						}
						{02}{%
							Mostrar, por decomposição e composição, que todo número natural pode ser escrito
							por meio de adições e multiplicações por potências de dez, para compreender o sistema de
							numeração decimal e desenvolver estratégias de cálculo.
						}
						{03}{%
							Resolver e elaborar problemas com números naturais envolvendo adição e subtração,
							utilizando estratégias diversas, como cálculo, cálculo mental e algoritmos, além de fazer estimativas
							do resultado.
						}
						{04}{%
							Utilizar as relações entre adição e subtração, bem como entre multiplicação e divisão,
							para ampliar as estratégias de cálculo.
						}
						{05}{%
							Utilizar as propriedades das operações para desenvolver estratégias de cálculo.
						}
						{06}{%
							Resolver e elaborar problemas envolvendo diferentes significados da multiplicação
							(adição de parcelas iguais, organização retangular e proporcionalidade), utilizando estratégias
							diversas, como cálculo por estimativa, cálculo mental e algoritmos.
						}
						{07}{%
							Resolver e elaborar problemas de divisão cujo divisor tenha no máximo dois algarismos,
							envolvendo os significados de repartição equitativa e de medida, utilizando estratégias diversas,
							como cálculo por estimativa, cálculo mental e algoritmos.
						}
						{08}{%
							Resolver, com o suporte de imagem e/ou material manipulável, problemas simples
							de contagem, como a determinação do número de agrupamentos possíveis ao se combinar cada
							elemento de uma coleção com todos os elementos de outra, utilizando estratégias e formas de
							registro pessoais.
						}
						{09}{%
							Reconhecer as frações unitárias mais usuais (\(\frac{1}{2}\), \(\frac{1}{3}\), \(\frac{1}{4}\), \(\frac{1}{5}\), \(\frac{1}{10}\) e \(\frac{1}{100}\)) como
							unidades de medida menores do que uma unidade, utilizando a reta numérica como recurso.
						}
						{10}{%
							Reconhecer que as regras do sistema de numeração decimal podem ser estendidas
							para a representação decimal de um número racional e relacionar décimos e centésimos com a
							representação do sistema monetário brasileiro.
						}
						{11}{%
							Identificar regularidades em sequências numéricas compostas por múltiplos de um
							número natural.
						}
						{12}{%
							Reconhecer, por meio de investigações, que há grupos de números naturais para os
							quais as divisões por um determinado número resultam em restos iguais, identificando regularidades.
						}
						{13}{%
							Reconhecer, por meio de investigações, utilizando a calculadora quando necessário, as
							relações inversas entre as operações de adição e de subtração e de multiplicação e de divisão, para
							aplicá-las na resolução de problemas.
						}
						{14}{%
							Reconhecer e mostrar, por meio de exemplos, que a relação de igualdade existente
							entre dois termos permanece quando se adiciona ou se subtrai um mesmo número a cada um desses
							termos.
						}
						{15}{%
							Determinar o número desconhecido que torna verdadeira uma igualdade que envolve as
							operações fundamentais com números naturais.
						}
						{01}{%
							Ler, escrever e ordenar números naturais até a ordem de dezenas de milhar.
						}
						{02}{%
							Mostrar, por decomposição e composição, que todo número natural pode ser escrito
							por meio de adições e multiplicações por potências de dez, para compreender o sistema de
							numeração decimal e desenvolver estratégias de cálculo.
						}
						{03}{%
							Resolver e elaborar problemas com números naturais envolvendo adição e subtração,
							utilizando estratégias diversas, como cálculo, cálculo mental e algoritmos, além de fazer estimativas
							do resultado.
						}
						{04}{%
							Utilizar as relações entre adição e subtração, bem como entre multiplicação e divisão,
							para ampliar as estratégias de cálculo.
						}
						{05}{%
							Utilizar as propriedades das operações para desenvolver estratégias de cálculo.
						}
						{06}{%
							Resolver e elaborar problemas envolvendo diferentes significados da multiplicação
							(adição de parcelas iguais, organização retangular e proporcionalidade), utilizando estratégias
							diversas, como cálculo por estimativa, cálculo mental e algoritmos.
						}
						{07}{%
							Resolver e elaborar problemas de divisão cujo divisor tenha no máximo dois algarismos,
							envolvendo os significados de repartição equitativa e de medida, utilizando estratégias diversas,
							como cálculo por estimativa, cálculo mental e algoritmos.
						}
						{08}{%
							Resolver, com o suporte de imagem e/ou material manipulável, problemas simples
							de contagem, como a determinação do número de agrupamentos possíveis ao se combinar cada
							elemento de uma coleção com todos os elementos de outra, utilizando estratégias e formas de
							registro pessoais.
						}
						{09}{%
							Reconhecer as frações unitárias mais usuais (\(\frac{1}{2}\), \(\frac{1}{3}\), \(\frac{1}{4}\), \(\frac{1}{5}\), \(\frac{1}{10}\) e \(\frac{1}{100}\)) como
							unidades de medida menores do que uma unidade, utilizando a reta numérica como recurso.
						}
						{10}{%
							Reconhecer que as regras do sistema de numeração decimal podem ser estendidas
							para a representação decimal de um número racional e relacionar décimos e centésimos com a
							representação do sistema monetário brasileiro.
						}
						{11}{%
							Identificar regularidades em sequências numéricas compostas por múltiplos de um
							número natural.
						}
						{12}{%
							Reconhecer, por meio de investigações, que há grupos de números naturais para os
							quais as divisões por um determinado número resultam em restos iguais, identificando regularidades.
						}
						{13}{%
							Reconhecer, por meio de investigações, utilizando a calculadora quando necessário, as
							relações inversas entre as operações de adição e de subtração e de multiplicação e de divisão, para
							aplicá-las na resolução de problemas.
						}
						{14}{%
							Reconhecer e mostrar, por meio de exemplos, que a relação de igualdade existente
							entre dois termos permanece quando se adiciona ou se subtrai um mesmo número a cada um desses
							termos.
						}
						{15}{%
							Determinar o número desconhecido que torna verdadeira uma igualdade que envolve as
							operações fundamentais com números naturais.
						}
				}
			}
			{05}{%
				\IfStrEqCase{#2}{%
					{01}{%
							Ler, escrever e ordenar números naturais até a ordem das centenas de milhar com
							compreensão das principais características do sistema de numeração decimal.
						}
						{02}{%
							Ler, escrever e ordenar números racionais na forma decimal com compreensão
							das principais características do sistema de numeração decimal, utilizando, como recursos, a
							composição e decomposição e a reta numérica.
						}
						{03}{%
							Identificar e representar frações (menores e maiores que a unidade),
							associando-as ao resultado de uma divisão ou à ideia de parte de um todo, utilizando a reta
							numérica como recurso.
						}
						{04}{%
							Identificar frações equivalentes.
						}
						{05}{%
							Comparar e ordenar números racionais positivos (representações fracionária e
							decimal), relacionando-os a pontos na reta numérica.
						}
						{06}{%
							Associar as representações \qty{10}{\percent}{\percent}, \qty{25}{\percent}, \qty{50}{\percent}, \qty{75}{\percent} e \qty{100}{\percent} respectivamente à
							décima parte, quarta parte, metade, três quartos e um inteiro, para calcular porcentagens,
							utilizando estratégias pessoais, cálculo mental e calculadora, em contextos de educação
							financeira, entre outros.
						}
						{07}{%
							Resolver e elaborar problemas de adição e subtração com números naturais e
							com números racionais, cuja representação decimal seja finita, utilizando estratégias diversas,
							como cálculo por estimativa, cálculo mental e algoritmos.
						}
						{08}{%
							Resolver e elaborar problemas de multiplicação e divisão com números naturais e
							com números racionais cuja representação decimal é finita (com multiplicador natural e divisor
							natural e diferente de zero), utilizando estratégias diversas, como cálculo por estimativa,
							cálculo mental e algoritmos.
						}
						{09}{%
							Resolver e elaborar problemas simples de contagem envolvendo o princípio
							multiplicativo, como a determinação do número de agrupamentos possíveis ao se combinar
							cada elemento de uma coleção com todos os elementos de outra coleção, por meio de
							diagramas de árvore ou por tabelas.
						}
						{10}{%
							Concluir, por meio de investigações, que a relação de igualdade existente
							entre dois membros permanece ao adicionar, subtrair, multiplicar ou dividir cada um desses
							membros por um mesmo número, para construir a noção de equivalência.
						}
						{11}{%
							Resolver e elaborar problemas cuja conversão em sentença matemática seja uma
							igualdade com uma operação em que um dos termos é desconhecido.
						}
						{12}{%
							Resolver problemas que envolvam variação de proporcionalidade direta entre
							duas grandezas, para associar a quantidade de um produto ao valor a pagar, alterar as
							quantidades de ingredientes de receitas, ampliar ou reduzir escala em mapas, entre outros.
						}
						{13}{%
							Resolver problemas envolvendo a partilha de uma quantidade em duas partes
							desiguais, tais como dividir uma quantidade em duas partes, de modo que uma seja o dobro
							da outra, com compreensão da ideia de razão entre as partes e delas com o todo.
						}
						{14}{%
							Utilizar e compreender diferentes representações para a localização de objetos
							no plano, como mapas, células em planilhas eletrônicas e coordenadas geográficas, a fim de
							desenvolver as primeiras noções de coordenadas cartesianas.
						}
						{15}{%
							Interpretar, descrever e representar a localização ou movimentação de objetos no
							plano cartesiano (1º quadrante), utilizando coordenadas cartesianas, indicando mudanças de
							direção e de sentido e giros.
						}
						{16}{%
							Associar figuras espaciais a suas planificações (prismas, pirâmides, cilindros e
							cones) e analisar, nomear e comparar seus atributos.
						}
						{17}{%
							Reconhecer, nomear e comparar polígonos, considerando lados, vértices e
							ângulos, e desenhá-los, utilizando material de desenho ou tecnologias digitais.
						}
						{18}{%
							Reconhecer a congruência dos ângulos e a proporcionalidade entre os lados
							correspondentes de figuras poligonais em situações de ampliação e de redução em malhas
							quadriculadas e usando tecnologias digitais.
						}
						{19}{%
							Resolver e elaborar problemas envolvendo medidas das grandezas comprimento,
							área, massa, tempo, temperatura e capacidade, recorrendo a transformações entre as unidades
							mais usuais em contextos socioculturais.
						}
						{20}{%
							Concluir, por meio de investigações, que figuras de perímetros iguais podem
							ter áreas diferentes e que, também, figuras que têm a mesma área podem ter perímetros
							diferentes.
						}
						{21}{%
							Reconhecer volume como grandeza associada a sólidos geométricos e medir
							volumes por meio de empilhamento de cubos, utilizando, preferencialmente, objetos concretos.
						}
						{22}{%
							Apresentar todos os possíveis resultados de um experimento aleatório,
							estimando se esses resultados são igualmente prováveis ou não.
						}
						{23}{%
							Determinar a probabilidade de ocorrência de um resultado em eventos aleatórios,
							quando todos os resultados possíveis têm a mesma chance de ocorrer (equiprováveis).
						}
						{24}{%
							Interpretar dados estatísticos apresentados em textos, tabelas e gráficos (colunas
							ou linhas), referentes a outras áreas do conhecimento ou a outros contextos, como saúde e
							trânsito, e produzir textos com o objetivo de sintetizar conclusões.
						}
						{25}{%
							Realizar pesquisa envolvendo variáveis categóricas e numéricas, organizar dados
							coletados por meio de tabelas, gráficos de colunas, pictóricos e de linhas, com e sem uso de
							tecnologias digitais, e apresentar texto escrito sobre a finalidade da pesquisa e a síntese dos
							resultados.
						}
				}
			}
			{06}{%
				\IfStrEqCase{#2}{%
					{01}{%
							Comparar, ordenar, ler e escrever números naturais e números racionais cuja
							representação decimal é finita, fazendo uso da reta numérica.
						}
						{02}{%
							Reconhecer o sistema de numeração decimal, como o que prevaleceu no mundo
							ocidental, e destacar semelhanças e diferenças com outros sistemas, de modo a sistematizar suas
							principais características (base, valor posicional e função do zero), utilizando, inclusive, a composição
							e decomposição de números naturais e números racionais em sua representação decimal.
						}
						{03}{%
							Resolver e elaborar problemas que envolvam cálculos (mentais ou escritos, exatos
							ou aproximados) com números naturais, por meio de estratégias variadas, com compreensão
							dos processos neles envolvidos com e sem uso de calculadora.
						}
						{04}{%
							Construir algoritmo em linguagem natural e representá-lo por fluxograma que
							indique a resolução de um problema simples (por exemplo, se um número natural qualquer é
							par).
						}
						{05}{%
							Classificar números naturais em primos e compostos, estabelecer relações entre
							números, expressas pelos termos “é múltiplo de”, “é divisor de”, “é fator de”, e estabelecer, por
							meio de investigações, critérios de divisibilidade por \num{2}, \num{3}, \num{4}, \num{5}, \num{6}, \num{8}, \num{9}, \num{10}, \num{100} e \num{1000}.
						}
						{06}{%
							Resolver e elaborar problemas que envolvam as ideias de múltiplo e de divisor.
						}
						{07}{%
							Compreender, comparar e ordenar frações associadas às ideias de partes de
							inteiros e resultado de divisão, identificando frações equivalentes.
						}
						{08}{%
							Reconhecer que os números racionais positivos podem ser expressos nas formas
							fracionária e decimal, estabelecer relações entre essas representações, passando de uma
							representação para outra, e relacioná-los a pontos na reta numérica.
						}
						{09}{%
							Resolver e elaborar problemas que envolvam o cálculo da fração de uma
							quantidade e cujo resultado seja um número natural, com e sem uso de calculadora.
						}
						{10}{%
							Resolver e elaborar problemas que envolvam adição ou subtração com números
							racionais positivos na representação fracionária.
						}
						{11}{%
							Resolver e elaborar problemas com números racionais positivos na representação
							decimal, envolvendo as quatro operações fundamentais e a potenciação, por meio de
							estratégias diversas, utilizando estimativas e arredondamentos para verificar a razoabilidade de
							respostas, com e sem uso de calculadora.
						}
						{12}{%
							Fazer estimativas de quantidades e aproximar números para múltiplos da potência
							de 10 mais próxima.
						}
						{13}{%
							Resolver e elaborar problemas que envolvam porcentagens, com base na ideia
							de proporcionalidade, sem fazer uso da “regra de três”, utilizando estratégias pessoais, cálculo
							mental e calculadora, em contextos de educação financeira, entre outros.
						}
						{14}{%
							Reconhecer que a relação de igualdade matemática não se altera ao adicionar,
							subtrair, multiplicar ou dividir os seus dois membros por um mesmo número e utilizar essa
							noção para determinar valores desconhecidos na resolução de problemas.
						}
						{15}{%
							Resolver e elaborar problemas que envolvam a partilha de uma quantidade em
							duas partes desiguais, envolvendo relações aditivas e multiplicativas, bem como a razão entre
							as partes e entre uma das partes e o todo.
						}
						{16}{%
							Associar pares ordenados de números a pontos do plano cartesiano do 1º
							quadrante, em situações como a localização dos vértices de um polígono.
						}
						{17}{%
							Quantificar e estabelecer relações entre o número de vértices, faces e arestas
							de prismas e pirâmides, em função do seu polígono da base, para resolver problemas e
							desenvolver a percepção espacial.
						}
						{18}{%
							Reconhecer, nomear e comparar polígonos, considerando lados, vértices e
							ângulos, e classificá-los em regulares e não regulares, tanto em suas representações no plano
							como em faces de poliedros.
						}
						{19}{%
							Identificar características dos triângulos e classificá-los em relação às medidas dos
							lados e dos ângulos.
						}
						{20}{%
							Identificar características dos quadriláteros, classificá-los em relação a lados e a
							ângulos e reconhecer a inclusão e a intersecção de classes entre eles.
						}
						{21}{%
							Construir figuras planas semelhantes em situações de ampliação e de redução,
							com o uso de malhas quadriculadas, plano cartesiano ou tecnologias digitais.
						}
						{22}{%
							Utilizar instrumentos, como réguas e esquadros, ou softwares para representações
							de retas paralelas e perpendiculares e construção de quadriláteros, entre outros.
						}
						{23}{%
							Construir algoritmo para resolver situações passo a passo (como na construção
							de dobraduras ou na indicação de deslocamento de um objeto no plano segundo pontos de
							referência e distâncias fornecidas etc.).
						}
						{24}{%
							Resolver e elaborar problemas que envolvam as grandezas comprimento, massa,
							tempo, temperatura, área (triângulos e retângulos), capacidade e volume (sólidos formados
							por blocos retangulares), sem uso de fórmulas, inseridos, sempre que possível, em contextos
							oriundos de situações reais e/ou relacionadas às outras áreas do conhecimento.
						}
						{25}{%
							Reconhecer a abertura do ângulo como grandeza associada às figuras geométricas.
						}
						{26}{%
							Resolver problemas que envolvam a noção de ângulo em diferentes contextos e
							em situações reais, como ângulo de visão.
						}
						{27}{%
							Determinar medidas da abertura de ângulos, por meio de transferidor e/ou
							tecnologias digitais.
						}
						{28}{%
							Interpretar, descrever e desenhar plantas baixas simples de residências e vistas aéreas.
						}
						{29}{%
							Analisar e descrever mudanças que ocorrem no perímetro e na área de um
							quadrado ao se ampliarem ou reduzirem, igualmente, as medidas de seus lados, para
							compreender que o perímetro é proporcional à medida do lado, o que não ocorre com a área.
						}
						{30}{%
							Calcular a probabilidade de um evento aleatório, expressando-a por número
							racional (forma fracionária, decimal e percentual) e comparar esse número com a probabilidade
							obtida por meio de experimentos sucessivos.
						}
						{31}{%
							Identificar as variáveis e suas frequências e os elementos constitutivos (título, eixos,
							legendas, fontes e datas) em diferentes tipos de gráfico.
						}
						{32}{%
							Interpretar e resolver situações que envolvam dados de pesquisas sobre contextos
							ambientais, sustentabilidade, trânsito, consumo responsável, entre outros, apresentadas pela
							mídia em tabelas e em diferentes tipos de gráficos e redigir textos escritos com o objetivo de
							sintetizar conclusões.
						}
						{33}{%
							Planejar e coletar dados de pesquisa referente a práticas sociais escolhidas pelos
							alunos e fazer uso de planilhas eletrônicas para registro, representação e interpretação das
							informações, em tabelas, vários tipos de gráficos e texto.
						}
						{34}{%
							Interpretar e desenvolver fluxogramas simples, identificando as relações entre
							os objetos representados (por exemplo, posição de cidades considerando as estradas que as
							unem, hierarquia dos funcionários de uma empresa etc.).
						}
				}
			}
			{07}{%
				\IfStrEqCase{#2}{%
					{01}{%
							Resolver e elaborar problemas com números naturais, envolvendo as noções de
							divisor e de múltiplo, podendo incluir máximo divisor comum ou mínimo múltiplo comum, por
							meio de estratégias diversas, sem a aplicação de algoritmos.
						}
						{02}{%
							Resolver e elaborar problemas que envolvam porcentagens, como os que lidam
							com acréscimos e decréscimos simples, utilizando estratégias pessoais, cálculo mental e
							calculadora, no contexto de educação financeira, entre outros.
						}
						{03}{%
							Comparar e ordenar números inteiros em diferentes contextos, incluindo o
							histórico, associá-los a pontos da reta numérica e utilizá-los em situações que envolvam adição
							e subtração.
						}
						{04}{%
							Resolver e elaborar problemas que envolvam operações com números inteiros.
						}
						{05}{%
							Resolver um mesmo problema utilizando diferentes algoritmos.
						}
						{06}{%
							Reconhecer que as resoluções de um grupo de problemas que têm a mesma
							estrutura podem ser obtidas utilizando os mesmos procedimentos.
						}
						{07}{%
							Representar por meio de um fluxograma os passos utilizados para resolver um
							grupo de problemas.
						}
						{08}{%
							Comparar e ordenar frações associadas às ideias de partes de inteiros, resultado
							da divisão, razão e operador.
						}
						{09}{%
							Utilizar, na resolução de problemas, a associação entre razão e fração, como a
							fração 2/3 para expressar a razão de duas partes de uma grandeza para três partes da mesma
							ou três partes de outra grandeza.
						}
						{10}{%
							Comparar e ordenar números racionais em diferentes contextos e associá-los a
							pontos da reta numérica.
						}
						{11}{%
							Compreender e utilizar a multiplicação e a divisão de números racionais, a relação
							entre elas e suas propriedades operatórias.
						}
						{12}{%
							Resolver e elaborar problemas que envolvam as operações com números racionais.
						}
						{13}{%
							Compreender a ideia de variável, representada por letra ou símbolo, para expressar
							relação entre duas grandezas, diferenciando-a da ideia de incógnita.
						}
						{14}{%
							Classificar sequências em recursivas e não recursivas, reconhecendo que o
							conceito de recursão está presente não apenas na matemática, mas também nas artes e na
							literatura.
						}
						{15}{%
							Utilizar a simbologia algébrica para expressar regularidades encontradas em
							sequências numéricas.
						}
						{16}{%
							Reconhecer se duas expressões algébricas obtidas para descrever a regularidade
							de uma mesma sequência numérica são ou não equivalentes.
						}
						{17}{%
							Resolver e elaborar problemas que envolvam variação de proporcionalidade
							direta e de proporcionalidade inversa entre duas grandezas, utilizando sentença algébrica para
							expressar a relação entre elas.
						}
						{18}{%
							Resolver e elaborar problemas que possam ser representados por equações
							polinomiais de 1\textordmasculine grau, redutíveis à forma \(ax + b = c\), fazendo uso das propriedades da igualdade.
						}
						{19}{%
							Realizar transformações de polígonos representados no plano cartesiano,
							decorrentes da multiplicação das coordenadas de seus vértices por um número inteiro.
						}
						{20}{%
							Reconhecer e representar, no plano cartesiano, o simétrico de figuras em relação
							aos eixos e à origem.
						}
						{21}{%
							Reconhecer e construir figuras obtidas por simetrias de translação, rotação e reflexão,
							usando instrumentos de desenho ou softwares de geometria dinâmica e vincular esse estudo a
							representações planas de obras de arte, elementos arquitetônicos, entre outros.
						}
						{22}{%
							Construir circunferências, utilizando compasso, reconhecê-las como lugar
							geométrico e utilizá-las para fazer composições artísticas e resolver problemas que envolvam
							objetos equidistantes.
						}
						{23}{%
							Verificar relações entre os ângulos formados por retas paralelas cortadas por uma
							transversal, com e sem uso de softwares de geometria dinâmica.
						}
						{24}{%
							Construir triângulos, usando régua e compasso, reconhecer a condição de
							existência do triângulo quanto à medida dos lados e verificar que a soma das medidas dos
							ângulos internos de um triângulo é 180°.
						}
						{25}{%
							Reconhecer a rigidez geométrica dos triângulos e suas aplicações, como na construção
							de estruturas arquitetônicas (telhados, estruturas metálicas e outras) ou nas artes plásticas.
						}
						{26}{%
							Descrever, por escrito e por meio de um fluxograma, um algoritmo para a construção de
							um triângulo qualquer, conhecidas as medidas dos três lados.
						}
						{27}{%
							Calcular medidas de ângulos internos de polígonos regulares, sem o uso
							de fórmulas, e estabelecer relações entre ângulos internos e externos de polígonos,
							preferencialmente vinculadas à construção de mosaicos e de ladrilhamentos.
						}
						{28}{%
							Descrever, por escrito e por meio de um fluxograma, um algoritmo para a
							construção de um polígono regular (como quadrado e triângulo equilátero), conhecida a
							medida de seu lado.
						}
						{29}{%
							Resolver e elaborar problemas que envolvam medidas de grandezas inseridos em
							contextos oriundos de situações cotidianas ou de outras áreas do conhecimento, reconhecendo
							que toda medida empírica é aproximada.
						}
						{30}{%
							Resolver e elaborar problemas de cálculo de medida do volume de blocos retangulares,
							envolvendo as unidades usuais (metro cúbico, decímetro cúbico e centímetro cúbico).
						}
						{31}{%
							Estabelecer expressões de cálculo de área de triângulos e de quadriláteros.
						}
						{32}{%
							Resolver e elaborar problemas de cálculo de medida de área de figuras planas que
							podem ser decompostas por quadrados, retângulos e/ou triângulos, utilizando a equivalência
							entre áreas.
						}
						{33}{%
							Estabelecer o número como a razão entre a medida de uma circunferência e seu
							diâmetro, para compreender e resolver problemas, inclusive os de natureza histórica.
						}
						{34}{%
							Planejar e realizar experimentos aleatórios ou simulações que envolvem cálculo de
							probabilidades ou estimativas por meio de frequência de ocorrências.
						}
						{35}{%
							Compreender, em contextos significativos, o significado de média estatística como
							indicador da tendência de uma pesquisa, calcular seu valor e relacioná-lo, intuitivamente, com a
							amplitude do conjunto de dados.
						}
						{36}{%
							Planejar e realizar pesquisa envolvendo tema da realidade social, identificando a
							necessidade de ser censitária ou de usar amostra, e interpretar os dados para comunicá-los por
							meio de relatório escrito, tabelas e gráficos, com o apoio de planilhas eletrônicas.
						}
						{37}{%
							Interpretar e analisar dados apresentados em gráfico de setores divulgados pela
							mídia e compreender quando é possível ou conveniente sua utilização.
						}
				}
			}
			{08}{%
				\IfStrEqCase{#2}{%
					{01}{%
							Efetuar cálculos com potências de expoentes inteiros e aplicar esse conhecimento
							na representação de números em notação científica.
						}
						{02}{%
							Resolver e elaborar problemas usando a relação entre potenciação e radiciação,
							para representar uma raiz como potência de expoente fracionário.
						}
						{03}{%
							Resolver e elaborar problemas de contagem cuja resolução envolva a aplicação
							do princípio multiplicativo.
						}
						{04}{%
							Resolver e elaborar problemas, envolvendo cálculo de porcentagens, incluindo o
							uso de tecnologias digitais.
						}
						{05}{%
							Reconhecer e utilizar procedimentos para a obtenção de uma fração geratriz
							para uma dízima periódica.
						}
						{06}{%
							Resolver e elaborar problemas que envolvam cálculo do valor numérico de
							expressões algébricas, utilizando as propriedades das operações.
						}
						{07}{%
							Associar uma equação linear de 1\textordmasculine grau com duas incógnitas a uma reta no plano
							cartesiano.
						}
						{08}{%
							Resolver e elaborar problemas relacionados ao seu contexto próximo, que
							possam ser representados por sistemas de equações de 1\textordmasculine grau com duas incógnitas e
							interpretá-los, utilizando, inclusive, o plano cartesiano como recurso.
						}
						{09}{%
							Resolver e elaborar, com e sem uso de tecnologias, problemas que possam ser
							representados por equações polinomiais de 2\textordmasculine grau do tipo \(ax2 = b\).
						}
						{10}{%
							Identificar a regularidade de uma sequência numérica ou figural não recursiva
							e construir um algoritmo por meio de um fluxograma que permita indicar os números ou as
							figuras seguintes.
						}
						{11}{%
							Identificar a regularidade de uma sequência numérica recursiva e construir um
							algoritmo por meio de um fluxograma que permita indicar os números seguintes.
						}
						{12}{%
							Identificar a natureza da variação de duas grandezas, diretamente, inversamente
							proporcionais ou não proporcionais, expressando a relação existente por meio de sentença
							algébrica e representá-la no plano cartesiano.
						}
						{13}{%
							Resolver e elaborar problemas que envolvam grandezas diretamente ou
							inversamente proporcionais, por meio de estratégias variadas.
						}
						{14}{%
							Demonstrar propriedades de quadriláteros por meio da identificação da
							congruência de triângulos.
						}
						{15}{%
							Construir, utilizando instrumentos de desenho ou softwares de geometria
							dinâmica, mediatriz, bissetriz, ângulos de 90°, 60°, 45° e 30° e polígonos regulares.
						}
						{16}{%
							Descrever, por escrito e por meio de um fluxograma, um algoritmo para a
							construção de um hexágono regular de qualquer área, a partir da medida do ângulo central e
							da utilização de esquadros e compasso.
						}
						{17}{%
							Aplicar os conceitos de mediatriz e bissetriz como lugares geométricos na
							resolução de problemas.
						}
						{18}{%
							Reconhecer e construir figuras obtidas por composições de transformações
							geométricas (translação, reflexão e rotação), com o uso de instrumentos de desenho ou de
							softwares de geometria dinâmica.
						}
						{19}{%
							Resolver e elaborar problemas que envolvam medidas de área de figuras
							geométricas, utilizando expressões de cálculo de área (quadriláteros, triângulos e círculos), em
							situações como determinar medida de terrenos.
						}
						{20}{%
							Reconhecer a relação entre um litro e um decímetro cúbico e a relação entre litro
							e metro cúbico, para resolver problemas de cálculo de capacidade de recipientes.
						}
						{21}{%
							Resolver e elaborar problemas que envolvam o cálculo do volume de recipiente
							cujo formato é o de um bloco retangular.
						}
						{22}{%
							Calcular a probabilidade de eventos, com base na construção do espaço amostral,
							utilizando o princípio multiplicativo, e reconhecer que a soma das probabilidades de todos os
							elementos do espaço amostral é igual a 1.
						}
						{23}{%
							Avaliar a adequação de diferentes tipos de gráficos para representar um conjunto
							de dados de uma pesquisa.
						}
						{24}{%
							Classificar as frequências de uma variável contínua de uma pesquisa em classes,
							de modo que resumam os dados de maneira adequada para a tomada de decisões.
						}
						{25}{%
							Obter os valores de medidas de tendência central de uma pesquisa estatística
							(média, moda e mediana) com a compreensão de seus significados e relacioná-los com a
							dispersão de dados, indicada pela amplitude.
						}
						{26}{%
							Selecionar razões, de diferentes naturezas (física, ética ou econômica), que
							justificam a realização de pesquisas amostrais e não censitárias, e reconhecer que a seleção
							da amostra pode ser feita de diferentes maneiras (amostra casual simples, sistemática e
							estratificada).
						}
						{27}{%
							Planejar e executar pesquisa amostral, selecionando uma técnica de amostragem
							adequada, e escrever relatório que contenha os gráficos apropriados para representar os
							conjuntos de dados, destacando aspectos como as medidas de tendência central, a amplitude
							e as conclusões.
						}
				}
			}
			{09}{%
				\IfStrEqCase{#2}{%
					{01}{%
							Reconhecer que, uma vez fixada uma unidade de comprimento, existem
							segmentos de reta cujo comprimento não é expresso por número racional (como as medidas
							de diagonais de um polígono e alturas de um triângulo, quando se toma a medida de cada lado
							como unidade).
						}
						{02}{%
							Reconhecer um número irracional como um número real cuja representação
							decimal é infinita e não periódica, e estimar a localização de alguns deles na reta numérica.
						}
						{03}{%
							Efetuar cálculos com números reais, inclusive potências com expoentes
							fracionários.
						}
						{04}{%
							Resolver e elaborar problemas com números reais, inclusive em notação
							científica, envolvendo diferentes operações.
						}
						{05}{%
							Resolver e elaborar problemas que envolvam porcentagens, com a ideia de
							aplicação de percentuais sucessivos e a determinação das taxas percentuais, preferencialmente
							com o uso de tecnologias digitais, no contexto da educação financeira.
						}
						{06}{%
							Compreender as funções como relações de dependência unívoca entre duas
							variáveis e suas representações numérica, algébrica e gráfica e utilizar esse conceito para
							analisar situações que envolvam relações funcionais entre duas variáveis.
						}
						{07}{%
							Resolver problemas que envolvam a razão entre duas grandezas de espécies
							diferentes, como velocidade e densidade demográfica.
						}
						{08}{%
							Resolver e elaborar problemas que envolvam relações de proporcionalidade
							direta e inversa entre duas ou mais grandezas, inclusive escalas, divisão em partes
							proporcionais e taxa de variação, em contextos socioculturais, ambientais e de outras áreas.
						}
						{09}{%
							Compreender os processos de fatoração de expressões algébricas, com base em
							suas relações com os produtos notáveis, para resolver e elaborar problemas que possam ser
							representados por equações polinomiais do 2º grau.
						}
						{10}{%
							Demonstrar relações simples entre os ângulos formados por retas paralelas
							cortadas por uma transversal.
						}
						{11}{%
							Resolver problemas por meio do estabelecimento de relações entre arcos,
							ângulos centrais e ângulos inscritos na circunferência, fazendo uso, inclusive, de softwares de
							geometria dinâmica.
						}
						{12}{%
							Reconhecer as condições necessárias e suficientes para que dois triângulos
							sejam semelhantes.
						}
						{13}{%
							Demonstrar relações métricas do triângulo retângulo, entre elas o teorema de
							Pitágoras, utilizando, inclusive, a semelhança de triângulos.
						}
						{14}{%
							Resolver e elaborar problemas de aplicação do teorema de Pitágoras ou das
							relações de proporcionalidade envolvendo retas paralelas cortadas por secantes.
						}
						{15}{%
							Descrever, por escrito e por meio de um fluxograma, um algoritmo para a
							construção de um polígono regular cuja medida do lado é conhecida, utilizando régua e
							compasso, como também softwares.
						}
						{16}{%
							Determinar o ponto médio de um segmento de reta e a distância entre dois pontos
							quaisquer, dadas as coordenadas desses pontos no plano cartesiano, sem o uso de fórmulas, e
							utilizar esse conhecimento para calcular, por exemplo, medidas de perímetros e áreas de figuras
							planas construídas no plano.
						}
						{17}{%
							Reconhecer vistas ortogonais de figuras espaciais e aplicar esse conhecimento
							para desenhar objetos em perspectiva.
						}
						{18}{%
							Reconhecer e empregar unidades usadas para expressar medidas muito grandes
							ou muito pequenas, tais como distância entre planetas e sistemas solares, tamanho de vírus ou
							de células, capacidade de armazenamento de computadores, entre outros.
						}
						{19}{%
							Resolver e elaborar problemas que envolvam medidas de volumes de prismas e
							de cilindros retos, inclusive com uso de expressões de cálculo, em situações cotidianas.
						}
						{20}{%
							Reconhecer, em experimentos aleatórios, eventos independentes e dependentes
							e calcular a probabilidade de sua ocorrência, nos dois casos.
						}
						{21}{%
							Analisar e identificar, em gráficos divulgados pela mídia, os elementos que
							podem induzir, às vezes propositadamente, erros de leitura, como escalas inapropriadas,
							legendas não explicitadas corretamente, omissão de informações importantes (fontes e
							datas), entre outros.
						}
						{22}{%
							Escolher e construir o gráfico mais adequado (colunas, setores, linhas), com
							ou sem uso de planilhas eletrônicas, para apresentar um determinado conjunto de dados,
							destacando aspectos como as medidas de tendência central.
						}
						{23}{%
							Planejar e executar pesquisa amostral envolvendo tema da realidade social e
							comunicar os resultados por meio de relatório contendo avaliação de medidas de tendência central
							e da amplitude, tabelas e gráficos adequados, construídos com o apoio de planilhas eletrônicas.
						}
				}
			}
	}[\la@invalidefhability]
}
