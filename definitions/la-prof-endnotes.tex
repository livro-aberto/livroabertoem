\makepagenote
\notepageref

% % Redefine as configurações da página de notas para sair do paracol e passar a utilizar multicols
\renewcommand*{\notedivision}{
	\cleardoublepage
	\teachercol@off
	\par \la@section{\sectiontitlefont\notesname}
	\begin{multicols}{2}
}

%Termina o multicols da página de notas.
\apptocmd{\printpagenotes}{
	\clearpage
	\end{multicols}
}{}{}

\renewcommand*{\pagerefname}{Página}

\renewcommand*{\pagenotesubhead}[3]{}

\renewcommand*{\notenumintext}[1]{{\noindent\hspace{-2.5pt}\bannernamefont\color{linkscolor} Nota #1}}

\renewcommand*{\notenuminnotes}[1]{{\hspace{-2.5pt}\bannernamefont\color{principal} Nota #1.} \space}

\renewcommand\printpageinnoteshyperref[1]{%
(\hyperref[#1]{\pagerefname\ \pageref*{#1}})\space}


\newcommand{\la@teacher@note}[3]{
		\leavevmode\pagenote{
		{
			\color{principal}\bfseries
			{\fontsize{14pt}{0pt}\selectfont #1}\par
			\vspace{-.8\baselineskip}	
			\rule{\linewidth}{3pt}
			{\Large #2}
		}	
		#3
	}
}

\newcommand{\teachernotes}[3][]{
	\IfStrEqCase{#1}{%
		{}{
			\la@teacher@note{#2}{}{#3}
		}
		{objectives}{
			\la@teacher@note{Objetivos específicos}{#2}{#3}				
		}%
		{sugestions}{
			\la@teacher@note{Sugestões e discussões}{#2}{#3}			
		}%
		{answer}{%
			\la@teacher@note{Solução}{#2}{#3}			
		}%
	}
}


