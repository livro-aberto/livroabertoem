\makepagenote
\notepageref

% % Redefine as configurações da página de notas para sair do paracol e passar a utilizar multicols
\renewcommand*{\notedivision}{
    \la@switchteacherpnum
    \la@teacherpage{\notesname}
    \addcontentsline{toc}{section}{Notas}
    \sectioncolor{\currentvolcolor}
}

\apptocmd{\printpagenotes}{
    \endla@teacherpage%
}{}{}

\renewcommand*{\notesname}{Notas}
\renewcommand*{\pagerefname}{Página}
\renewcommand*{\pagenotesubhead}[3]{}

\renewcommand*{\notenumintext}[1]{
  {\par\bannernamefont\color{linkscolor} Nota #1}
}

\renewcommand*{\notenuminnotes}[1]{
  {\hspace{-2.5pt}\bannernamefont\color{\currentvolcolor} Nota #1.} \space
}

\renewcommand\printpageinnoteshyperref[1]{%
  (\hyperref[#1]{\pagerefname\ \pageref*{#1}})\space
}

\newcommand{\teachernotelist}{\la@defaultlist}
\newcommand{\la@teacher@note}[3]{
		\leavevmode\pagenote{
      \teachernotelist%
      \teachertitle[\currentvolcolor]{#1}{#2} #3
	}
}

\newcommand{\teachernotes}[3][]{
	\IfStrEqCase{#1}{%
		{}{
			\la@teacher@note{#2}{}{#3}
		}
		{objectives}{
			\la@teacher@note{Objetivos específicos}{#2}{#3}				
		}%
		{sugestions}{
			\la@teacher@note{Sugestões e discussões}{#2}{#3}			
		}%
		{answer}{%
			\la@teacher@note{Solução}{#2}{#3}			
		}%
	}
}


