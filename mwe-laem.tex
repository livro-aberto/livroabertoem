% arara: lualatex: {interaction: nonstopmode}

\documentclass[]{livroabertoem}

\usepackage{livroabertoem-preamble}
\usepackage{lipsum}

\footericon{estprob}
\chapterillustration{estprob-illustration}
\volumeillustration{estprob}
\volumecolor{volume2}

\author{Tarso Boudet Caldas \and Tarso Caldas}
\chapterwhat{\lipsum[5]}
\chapterbecause{\lipsum[7]}
\begin{document}

\frontmatter

\volume{Probabilidade e Estatística}
\tableofcontents

\mainmatter
\chapter{A Natureza da Estatística}
% \credits

\taskintoc

\begin{teacherintroduction}
\lipsum[2]

\begin{habilities}
  \hability{101}
\end{habilities}

\begin{habilities}[EF]
  \hability[05]{04}
\end{habilities}

\lipsum[3-5]

\end{teacherintroduction}

\section{A Natureza da Estatística}

\lipsum[1-3]

\explore{A Natureza da Estatística}

\lipsum[1]
\[
    a^2 + b^2 \in \times = c^2 + 3 + 4
\]

\begin{task}{Triângulos}
\begin{teacher}
  \objectives{Triângulos}
  \lipsum[12]
  \sugestions{Triângulos}
  \lipsum[13]
  \answer{Triângulos}
  \lipsum[13]
\end{teacher}

\lipsum[2]
\end{task}

\lipsum[3]

\arrange{A Natureza da Estatística}

\begin{example}{Triângulos}
  \lipsum[4]
\end{example}

\newpage

\begin{knowledge}
  \lipsum[6]
\end{knowledge}

\practice{A Natureza da Estatística}
\begin{observation}{Triângulos}
  \lipsum[5]
\end{observation}
\begin{reflection}
  \lipsum[6]
\end{reflection}

\begin{teachersection}{A Natureza da Estatística}

\lipsum[1-7]

\end{teachersection}
\section{A Natureza da Estatística}

\lipsum[1-3]

\explore{A Natureza da Estatística}

\lipsum[1]
\[
    a^2 + b^2 \in \times = c^2 + 3 + 4
\]

\begin{task}{Triângulos}

\begin{teacher}
  \objectives{Triângulos}
  \lipsum[12]
  \sugestions{Triângulos}
  \lipsum[13]
  \answer{Triângulos}
  \lipsum[13]
\end{teacher}

\lipsum[2]
\end{task}

\lipsum[3]

\arrange{A Natureza da Estatística}

\begin{example}{Triângulos}
  \lipsum[4]
\end{example}

\newpage

\know{A Natureza da Estatística}
\begin{research}
  \lipsum[6]
\end{research}

\begin{project}
  \lipsum[7]
\end{project}


\begin{knowledge}
\lipsum[5]
\end{knowledge}

\begin{reflection}
\lipsum[6]
\end{reflection}


\exercise

\begin{enumerate}
  \item \lipsum[7]
  \item \lipsum[8]
  \item \lipsum[9]
  \item \lipsum[10]
\end{enumerate}

% \printpagenotes

\end{document}
