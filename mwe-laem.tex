% arara: lualatex: {interaction: nonstopmode}

\documentclass[professor,nofonts]{livroabertoem}

\usepackage{livroabertoem-preamble}
\usepackage{lipsum}

% \footericon{funcoes-footericon}
% \chapterillustration{funcoes-chapillustration}
% \volumeillustration{funcoes-volillustration}
\volumecolor{volume4}

\author{Tarso Boudet Caldas \and Tarso Caldas}
\chapterwhat{\lipsum[5]}
\chapterbecause{\lipsum[7]}
\begin{document}


\volume{Probabilidade e Estatística}
\tableofcontents

% \mainmatter

\chapter{A Natureza da Estatística}


\begin{teacherintroduction}
	\lipsum[2]

	\begin{habilities}
		\hability{101}
	\end{habilities}

	\begin{habilities}
		\efhability{01}{04}
	\end{habilities}

	\lipsum[3-15]

\end{teacherintroduction}

\section{A Natureza da Estatística}

\begin{longteachersection}{A Natureza da estatística}
	\lipsum[3-15]
\end{longteachersection}

\explore{A Natureza da Estatística}

\lipsum[1]
\[
	a^2 + b^2 \in \times = c^2 + 3 + 4
\]

\begin{task}{Triângulos}
	\begin{teacher}
		\objectives{Triângulos}
		\lipsum[12]
		\sugestions{Triângulos}
		\lipsum[13]
		\answer{Triângulos}
		\lipsum[13]
	\end{teacher}

	\lipsum[2]
\end{task}

\lipsum[3]

\arrange{A Natureza da Estatística}

\begin{example}{Triângulos}
	\lipsum[4]
\end{example}

\newpage

\begin{knowledge}
	\lipsum[6]
\end{knowledge}

\practice{A Natureza da Estatística}
\begin{observation}{Triângulos}
	\lipsum[5]
\end{observation}
\begin{reflection}
	\lipsum[6]
\end{reflection}

% \section{A Natureza da Estatística}
%
% \begin{teachersection}{A Natureza da Estatística}
%
% 	\lipsum[1-7]
%
% \end{teachersection}
%
% \explore{A Natureza da Estatística}
%
% \lipsum[1]
% \[
% 	a^2 + b^2 \in \times = c^2 + 3 + 4
% \]
%
% \begin{task}{Triângulos}
%
% 	\begin{teacher}
% 		\objectives{Triângulos}
% 		\lipsum[12]
% 		\sugestions{Triângulos}
% 		\lipsum[13]
% 		\answer{Triângulos}
% 		\lipsum[13]
% 	\end{teacher}
%
% 	\lipsum[2]
% \end{task}
%
% \lipsum[3]
%
% \arrange{A Natureza da Estatística}
%
% \begin{example}{Triângulos}
% 	\lipsum[4]
% \end{example}
%
% \newpage
%
% \know{A Natureza da Estatística}
% \begin{research}
% 	\lipsum[6]
% \end{research}
%
% \begin{project}
% 	\lipsum[7]
% \end{project}
%
%
% \begin{knowledge}
% 	\lipsum[5]
% \end{knowledge}
%
% \begin{reflection}
% 	\lipsum[6]
% \end{reflection}
%
% \begin{definition}{Ácido}
% 	\lipsum[18]
% \end{definition}
% \begin{theorem}{Pitágoras}
%
% \end{theorem}
%
% \exercise
%
% \begin{enumerate}
% 	\item \lipsum[7]
% 	\item \lipsum[8]
% 	\item \lipsum[9]
% 	\item \lipsum[10]
% \end{enumerate}

% \printpagenotes

\end{document}
